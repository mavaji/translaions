\section[زود باش و صبر کن]{زود باش و صبر کن \protect\LTRfootnote{\lr{Hurry Up and Wait}} (راهنمایی \ref{hint:183}، جواب \ref{answer:65})}
\paragraph{}\label{prog:12}
کدی که بر اساس آن، این برنامه نوشته شده است، توسط یک برنامه‌نویسِ سیستم در شرکتی که مدت‌ها پیش در آن کار می‌کردم، نوشته شده است.

قرار بود این برنامه روی یک خط سریال، داده بفرستد. با این که خط سریال می‌توانست تا 960 کاراکتر در ثانیه را رد و بدل کند، ما فقط می‌توانستیم 300 کاراکتر در ثانیه داشته باشیم. چرا؟

\begin{LTR}
    %@formatter:off
        \begin{lstlisting}[style=C++Style]
             /************************************************
             * send_file -- Send a file to a remote link *
             * (Stripped down for this example.) *
             ************************************************/
             #include <iostream>
             #include <fstream>
             #include <stdlib.h>

             // Size of a block
             const int BLOCK_SIZE = 256;

             /************************************************
             * send_block -- Send a block to the output port*
             ************************************************/
             void send_block(
             	std::istream &in_file, // The file to read
             	std::ostream &serial_out // The file to write
             )
             {
             	int i; // Character counter

             	for (i = 0; i < BLOCK_SIZE; ++i) {
             		int ch; // Character to copy

             		ch = in_file.get();
             		serial_out.put(ch);
             		serial_out.flush();
             	}
             }

             int main()
             {
             	// The input file
             	std::ifstream in_file("file.in");

             	// The output device (faked)
             	std::ofstream out_file("/dev/null");

             	if (in_file.bad())
             	{
             		std::cerr <<
             			"Error: Unable to open input file\n";
             		exit (8);
             	}

             	if (out_file.bad())
             	{
             		std::cerr <<
             			"Error: Unable to open output file\n";
             		exit (8);
             	}

             	while (! in_file.eof())
             	{
             		// The original program output
             		// a block header here
             		send_block(in_file, out_file);
             		// The original program output a block
             		// trailer here. It also checked for
             		// a response and resent the block
             		// on error
             	}
             	return (0);
             }
        \end{lstlisting}
    %@formatter:on
\end{LTR}

\begin{tcolorbox}
    یک مدیر سیستم عادت دارد که دو هفته قبل ازاین که سیستم‌ها را ارتقا دهد، اعلام می‌کند که کار ارتقا انجام شده است. نوعاً اعتراض‌های عجولانه‌ای مانند «نرم‌افزار من از کار افتاده است و این نتیجهٔ ارتقای شما است» در روز اِعلان وجود خواهد داشت. مدیر می‌داند که این امر به دلیل ارتقا نیست چون واقعاً آن را انجام نداده است.

    وقتی که او واقعاً عمل ارتقا را انجام می‌دهد (به طور مخفیانه) هر اعتراضی که بعد از آن وجود داشته باشد احتمالاً برحق است.

    اپراتورهای آماتورِ رادیو از این ترفند استفاده می‌کنند. آنها یک برج رادیوییِ جدید نصب می‌کنند و آن را برای چند هفته قطع نگه می‌دارند. این کار به همسایه‌ها دو هفته فرصت می‌دهد تا به تداخل تلویزیون به خاطر وجود آنتن جدید اعتراض کنند.
\end{tcolorbox}