\section[برنامه‌نویسیِ ابلهانه]{برنامه‌نویسیِ ابلهانه \protect\LTRfootnote{\lr{Sheepish Programming}} (راهنمایی \ref{hint:165}، جواب \ref{answer:1})}
\paragraph{}\label{prog:55}
براونِ کشاورز، یک پرورش‌دهندهٔ گوسفند، همسایه‌ای داشت که می‌توانست به یک گله نگاه کند و بگوید که چند گوسفند در آن وجود دارد. او بسیار متعجب بود که چگونه دوستش به این سرعت این کار را انجام می‌دهد، لذا از او پرسید.

«یان، چطور به این سرعت می‌توانی تعداد گوسفندها را بگویی؟»

«به سادگی، پاها را می‌شمارم و بر 4 تقسیم می‌کنم.»
براونِ کشاورز آنقدر تحت تأثیر قرار گرفت که یک برنامهٔ \lr{\texttt{C++}} کوتاه نوشت تا درستی الگوریتم گوسفندشماریِ یان را بررسی کند. این برنامه برای گله‌های بزرگ جواب نمی‌دهد. چرا؟

\begin{LTR}
    %@formatter:off
    \begin{lstlisting}[style=C++Style]
         /************************************************
         * sheep -- Count sheep by counting the *
         * number of legs and dividing by 4. *
         ************************************************/
         #include <iostream>

         /*
         * The number of legs in some different
         * size herds.
         */
         const short int small_herd = 100;
         const short int medium_herd = 1000;
         const short int large_herd = 10000;

         /************************************************
         * report_sheep -- Given the number of legs, *
         * tell us how many sheep we have. *
         ************************************************/
         static void report_sheep(
         	const short int legs // Number of legs
         )
         {
         	std::cout <<
         		"The number of sheep is: " <<
         		(legs/4) << std::endl;
         }

         int main() {
         	report_sheep(small_herd*4); // Expect 100
         	report_sheep(medium_herd*4);// Expect 1000
         	report_sheep(large_herd*4); // Expect 10000
         	return (0);
         }
    \end{lstlisting}
    %@formatter:on
\end{LTR}