\section{}
\paragraph{}\label{answer:105}
ماکروی \lr{\texttt{ABORT}} به دو عبارت بسط می‌یابد. لذا نتیجهٔ عبارت \lr{\texttt{if}} برابر است با:
\begin{LTR}
    %@formatter:off
    \begin{lstlisting}[style=C++Style]
        if (value < 0)
            std::cerr << "Illegal root" << std::endl;exit (8);
    \end{lstlisting}
    %@formatter:on
\end{LTR}

یا اگر به درستی توگذاری کنیم:
\begin{LTR}
    %@formatter:off
    \begin{lstlisting}[style=C++Style]
        if (value < 0)
            std::cerr << "Illegal root" << std::endl;
        exit (8);
    \end{lstlisting}
    %@formatter:on
\end{LTR}

از این جا به راحتی می‌توان فهمید که چرا ما همیشه خارج می‌شویم. به جای ماکروهای چندعبارته از توابع \lr{\texttt{inline}} استفاده کنید:
\begin{LTR}
    %@formatter:off
    \begin{lstlisting}[style=C++Style]
        inline void ABORT(const char msg[]) {
            std::cerr << msg << std::endl;
            exit(8);
        }
    \end{lstlisting}
    %@formatter:on
\end{LTR}

اگر مجبورید از ماکروهای چندعبارته استفاده کنید، آنها را در آکولاد قرار دهید:
\begin{LTR}
    %@formatter:off
    \begin{lstlisting}[style=C++Style]
        #define ABORT(msg) \
            {std::cerr << msg << std::endl;exit(8);}
    \end{lstlisting}
    %@formatter:on
\end{LTR}
