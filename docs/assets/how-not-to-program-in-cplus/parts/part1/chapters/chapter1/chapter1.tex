\chapter[در آغاز]{در آغاز\protect\LTRfootnote{\lr{In the Beginning}}}
در آغاز، \lr{ENIAC MARK I} بود. روزی، اپراتوری متوجه شد که دستگاه درست کار نمیکند و فهمید که بیدی داخل دستگاه رفته و با برخورد به رله‌ها مرده است. او بید را بیرون انداخت و در گزارش کار خود نوشت «یک ساس در سیستم پیدا شد». و این چنین اولین ساسِ (باگ) کامپیوتری کشف شد\footnote{\rl{ با این که افراد معتقند که این واقعه، اولین کاربرد کلمه باگ در مورد ماشین‌های محاسباتی بود، ولی اینگونه نبود. اصطلاح باگ برای مدت مدیدی قبل از آن به همه گونه اشکالاتِ ماشین‌افزار اطلاق می‌شد. به هر حال چرا یک داستان خوب را با واقعیت خراب کنیم؟}}.

آشنایی من با باگ‌های کامپیوتری، مدت‌ها پس از آن واقعه انجام شد. من اولین برنامه‌ام را در سن 11 سالگی نوشتم. طول آن فقط یک دستورِ اسمبلی بود. آن برنامه 2 + 2 را با هم جمع می‌کرد. نتیجه برابر 2 می‌شد. طول آن برنامه فقط یک دستورالعمل بود و با این حال هم باگ داشت.

این فصل شامل یک سری مقدمات می‌شود: اولین باری که من تا 2 نیمه‌شب بیدار ماندم تا یک باگ را پیدا کنم (برنامه \ref{prog:3})، اولین سؤالی که در اولین آزمونِ برنامه‌نویسی \lr{\texttt{C}} طرح کردم (برنامه \ref{prog:2}) و البته اولین برنامه‌ای که در هر کتابِ برنامه‌نویسی موجود می‌باشد: \lr{"Hello World"}.

\begin{tcolorbox}
    قبل از اختراع \lr{ATM}، شما مجبور بودید به بانک بروید و به طور دستی، کارهای دریافت و پرداخت را انجام دهید. معمولاً می‌توانستید از یکی از برگه‌های چاپ شده در دفترچه حساب خود استفاده کنید. شماره حسابِ شما با مرکبِ مغناطیسی در پایینِ برگه‌ها نوشته شده بود.

    اگر برگه‌های شما تمام می‌شد، بانک یکی به شما می‌داد. در پایینِ آن، هیچ شماره‌ای نوشته نمی‌شد، لذا وقتی توسط دستگاهِ اتوماتیک بانک، پردازش می‌شد، دستگاه آن را بیرون می‌داد و یک کارمند شماره حساب را به طور دستی در آن وارد می‌کرد.

    یک کلاهبردار، برگه‌های «نوعیِ» خودش را چاپ کرد. آن شبیه برگه‌های «نوعیِ» معمولی بود به جز این که شماره حسابِ کلاهبردار با مرکب مغناطیسی در پایینِ آن نوشته شده بود. او سپس به بانک رفت و آن برگه‌ها را در سبد برگه‌های «نوعی» انداخت.

    کلاهبرداری بدین صورت بود: یک مشتری وارد بانک شد تا کار بانکی انجام دهد و یکی از آن برگه‌های دستکاری‌شده را برداشت. او آن برگه را پر کرد و پول پرداخت کرد. از آنجا که برگه، شماره حساب داشت، کامپیوتر به صورت اتوماتیک آن را پردازش کرد و پولی به حسابِ نوشته‌شده در پایینِ برگه واریز کرد. آن چه که به آن توجه نشد، شماره حسابی بود که به طور دستی روی برگه نوشته شده بود. به عبارت دیگر، کلاهبردار ما داشت پول‌ها را می‌دزدید.

    کارآگاهِ مسئول این قضیه گیج شده بود. پول‌ها ناپدید می‌شدند و کسی نمی‌دانست چگونه. او کار را به پول‌هایی که در بانک پرداخت می‌شدند محدود کرد. او تصمیم گرفت که تعداد زیادی پرداخت انجام دهد و ببیند که چه اتفاقی می‌افتد. چون او از جیب خودش داشت خرج می‌کرد، پولهایی که پرداخت می‌کرد، بسیار کم بود. بسیار بسیار کم. در حقیقت هر کدام 6 سنت بودند.

    کارآگاه یک هفته را به این کار گذراند. به بانک می‌رفت، یک برگه پر می‌کرد، در صف می‌ایستاد، 6 سنت پرداخت می‌کرد، یک برگهٔ جدید پر می‌کرد، در صف می‌ایستاد، 6 سنت پرداخت می‌کرد و الخ. کارمندان فکر می‌کردند که او دیوانه شده است. یک روز، یکی از پرداخت‌هایش ناپدید شد. او بانک را مجبور کرد که بررسی کند که آیا کسِ دیگری در آن روز، یک پرداخت 6 سنتی داشته یا نه. یکی داشت و این گونه کلاهبردار به دام افتاد.
\end{tcolorbox}

\newcommand{\@path}{parts/part1/chapters/chapter1/prog}

\foreach \n in {1,2,...,3}{
    \IfFileExists{\@path\n}{
        \input{\@path\n}
    }
}