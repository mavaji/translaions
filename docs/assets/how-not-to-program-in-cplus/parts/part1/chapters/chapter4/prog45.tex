\section[شگفتی سه‌چندان]{شگفتی سه‌چندان \protect\LTRfootnote{\lr{Triple Surprise}} (راهنمایی \ref{hint:312}، جواب \ref{answer:80})}
\paragraph{}\label{prog:45}
آیا \lr{\texttt{a, b, c}} به ترتیب نزولی هستند؟ آیا این برنامه با شما موافق است؟

\begin{LTR}
    %@formatter:off
    \begin{lstlisting}[style=C++Style]
         /************************************************
         * test to see if three variables are in order. *
         ************************************************/
         #include <iostream>

         int main()
         {
         	int a,b,c; // Three simple variables

         	a = 7;
         	b = 5;
         	c = 3;

         	// Test to see if they are in order
         	if (a > b > c)
         		std::cout << "a,b,c are in order\n";
         	else
         		std::cout << "a,b,c are mixed up\n";
         	return (o);
         }
    \end{lstlisting}
    %@formatter:on
\end{LTR}

\begin{tcolorbox}
    دیباگِرِ همهٔ کامپیوترهای \lr{DEC}، \lr{DDT} نام دارد. در دفترچه راهنمای \lr{PDP-10 DDT} پاورقی وجود دارد که این اسم از کجا آمده است:

    پاورقی تاریخی: \lr{DDT} در \lr{MIT} و برای کامپیوتر \lr{PDP-1} در 1961 ابداع شد. در آن زمان \lr{DDT} مخفف \lr{"DEC Debugging Tape"} بود. از آن موقع، ایدهٔ یک برنامهٔ دیباگ‌کنندهٔ آنلاین در کل صنایع کامپیوتر منتشر شد. برنامه‌های \lr{DDT} اکنون برای همهٔ کامپیوترهای \lr{DEC} موجود می‌باشند. از آنجا که به جای نوار، امروزه از وسائل دیگری استفاده می‌شود، اسم مناسب‌ترِ \lr{"Dynamic Debugging Technique"} به کار رفته است که مخفف آن همان \lr{DDT} می‌شود. اختلاط معنی بین \lr{DDT-10} و یک آفت‌کش معروف یعنی دیکلرو دیفنیل تریکلروتیل \lr{(C\textsubscript{14} H\textsubscript{9} C\textsubscript{15})} باید بسیار کم باشد چون هر کدام به یک نوع متفاوت و مانعة‌الجمع از آفت‌ها حمله می‌کنند.
\end{tcolorbox}