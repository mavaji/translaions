\section[اِعمال نیرو]{اِعمال نیرو \protect\LTRfootnote{\lr{Zipping Along}} (راهنمایی \ref{hint:206}، جواب \ref{answer:15})}
\paragraph{}\label{prog:62}
چه چیز ساده‌تر از انتساب یک مقدار به دو ثابت و نمایش آن است؟ با این حال در چیزی به این سادگی هم مشکل وجود دارد. چرا یکی از کدهای پُستی اشتباه است؟

\begin{LTR}
    %@formatter:off
    \begin{lstlisting}[style=C++Style]
         /************************************************
         * print_zip -- Print out a couple of zip codes.*
         ************************************************/
         #include <iostream>
         #include <iomanip>

         int main()
         {
         	// Zip code for San Diego
         	const long int san_diego_zip = 92126;

         	// Zip code for Boston
         	const long int boston_zip = 02126;

         	std::cout << "San Diego " << std::setw(5) <<
         		std::setfill('0') <<
         		san_diego_zip << std::endl;

         	std::cout << "Boston " << std::setw(5) <<
         		std::setfill('0') <<
         		boston_zip << std::endl;

         	return (0);
         }
    \end{lstlisting}
    %@formatter:on
\end{LTR}

\begin{tcolorbox}
    \centering
    قانون اُلین برای کامپیوترها

    \raggedright
    1. هیچ چیزی در علوم کامپیوتر، مهم‌تر از چنگ انداختن به چیزهای بدیهی نیست.

    2. هیچ چیزِ بدیهی در مورد کامپیوترها وجود ندارد.
\end{tcolorbox}