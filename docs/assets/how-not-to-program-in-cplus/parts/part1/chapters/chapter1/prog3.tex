\section[شگفتی در صبح زود]{شگفتی در صبح زود \protect\LTRfootnote{\lr{Early Morning Surprise}} (راهنمایی \ref{hint:34}، جواب \ref{answer:53})}
\paragraph{}\label{prog:3}
این برنامه توسط یکی از دوستانم نوشته شده وقتی که هر دو در دانشگاه بودیم. تمرینِ خانه این بود که یک روتینِ ضربِ ماتریسی بنویسیم. هر چند خودِ تابع باید به زبان اسمبلی نوشته می‌شد. برای این که سرعتِ اجرای آن را تا آنجا که می‌توانیم افزایش دهیم، دوستم باید از الگوریتمی که من طراحی کرده بودم و ماتریس را به صورت بُردار در می‌آورد، استفاده می‌کرد.

برای آزمایش سیستم، او یک تابع تست در SAIL\footnote{\lr{SAIL} یک زبان قدیمی برای برنامه‌نویسیِ سیستم \lr{PDP-10} بود. دیباگِرِ آن \lr{BAIL} نام داشت. بعدها یک نسخهٔ مستقل از ماشینِ این زبان ابداع گردید که \lr{MAIN SAIL} نام داشت. این زبان چندین سال قبل از \lr{\texttt{C}} وجود داشت.} نوشت. وقتی برنامه را تست کردیم، جواب‌های غلطی به دست آوردیم. هر دویِ ما، خط به خطِ برنامه را از 8 بعد از ظهر تا 2 نیمه‌شب موشکافی کردیم. وقتی نهایتاً خطا را پیدا کردیم، از این که مرتکب چنین اشتباه احمقانه‌ای شده بودیم، به شدت خندیدیم.

برنامهٔ زیر یک نسخهٔ ساده‌شده از آن کد است. تمام این برنامه به زبان \lr{\texttt{C}} نوشته شده است و از الگوریتمِ ساده‌تری برای ضرب استفاده می‌کند. ولی باگ اولیه کماکان وجود دارد. مشکل کجاست؟

\begin{LTR}
    %@formatter:off
        \begin{lstlisting}[style=C++Style]
             /**************************************************
             * matrix-test -- Test matrix multiply *
             **************************************************/
             #include <stdio.h>

             /**************************************************
             * matrix_multiply -- Multiple two matrixes *
             **************************************************/
             static void matrix_multiply(
             	int result[3][3], /* The result */
             	int matrixl[3][3],/* One multiplicand */
             	int matrix2[3][3] /* The other multiplicand */
             )
             {
             	/* Index into the elements of the matrix */
             	int row, col, element;

            	 for(row = 0; row < 3; ++row)
             	{
             		for(col = 0; col < 3; ++col)
             		{
             			result[row][col] = 0;
             			for(element = 0; element < 3; ++element)
             			{
             				result[row][col] +=
             				matrix1[row][element] *
             				matrix2[element][col];
             			}
             		}
             	}
             }

             /************************************************
             * matrix_print -- Output the matrix *
             ************************************************/
             static void matrix_print(
             	int matrix[3][3] /* The matrix to print */
             )
             {
             	int row, col; /* Index into the matrix */

             	for (row = 0; row < 3; ++row)
             	{
             		for (col = 0; col < 3; ++col)
             		{
             			printf("%o\t", matrix[row][col]);
             		}
             		printf("\n");
             	}
             }

             int main(void)
             {
             	/* One matrix for multiplication */
             	int matrix_a[3][3] = {
             		{45, 82, 26},
             		{32, 11, 13},
             		{89, 81, 25}
             	};
             	/* Another matrix for multiplication */
             	int matrix_b[3][3] = {
             		{32, 43, 50},
             		{33, 40, 52},
             		{20, 12, 32}
             	};
             	/* Place to put result */
             	int result[3][3];

             	matrix_multiply(result, matrix_a, matrix_b);
             	matrix_print(result);
             	return (o);
             }
        \end{lstlisting}
    %@formatter:on
\end{LTR}