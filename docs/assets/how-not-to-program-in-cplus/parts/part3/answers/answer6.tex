\section{}
\paragraph{}\label{answer:6}
مشکل خط \lr{\texttt{6 void main()}} است. تابع \lr{\texttt{main()}} یک تابع \lr{\texttt{void}} نیست. یک \lr{\texttt{int}} است. این تابع یک کد خروجی را به سیستم‌عامل برمی‌گرداند. یک \lr{\texttt{"Hello World"}} درست می‌توانست بدین شکل باشد:

\begin{LTR}
    %@formatter:off
            \begin{lstlisting}[style=C++Style]
                 /************************************************
                 * The "standard" hello world program. *
                 ************************************************/
                 #include <ostream>

                 int main()
                 {
                     std::cout << "Hello world!\n";
                     return (0);
                 }
            \end{lstlisting}
        %@formatter:on
\end{LTR}
وقتی همسر من برنامه‌نویسی را شروع کرد، این اولین برنامه‌ای بود که یاد گرفت (نسخه \lr{\texttt{void}}). من \lr{\texttt{void}} را به \lr{\texttt{int}} تغییر دادم و او تمرین خود را تحویل داد. معلم حل‌تمرین، این را غلط گرفت و به حالت اولیه برگرداند.

نیازی به گفتن نیست که من از این قضیه راضی نبودم و یک نامه خیلی تحقیرآمیز برای او نوشته و متذکر شدم که \lr{\texttt{main}} از نوع \lr{\texttt{int}} بوده است و با یادآوری استاندارد \lr{\texttt{C++}} حرف خودم را به اثبات رساندم. او به نامه من جواب داد و خیلی هم خوشحال شده بود.