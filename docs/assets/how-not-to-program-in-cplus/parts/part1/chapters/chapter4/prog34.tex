\section[سِنت‌ها چه شدند؟]{سِنت‌ها چه شدند؟ \protect\LTRfootnote{\lr{Non-Cents}} (راهنمایی \ref{hint:39}، جواب \ref{answer:107})}
\paragraph{}\label{prog:34}
این یک برنامهٔ سادهٔ دفترچه چک است. برنامه برای مدتی به درستی کار می‌کند ولی بعد از این که تعداد زیادی ورودی اضافه شدند، مجموعِ کل چند سِنت کم دارد. چه بر سر پول‌ها آمده است؟

\begin{LTR}
    %@formatter:off
    \begin{lstlisting}[style=C++Style]
         /************************************************
         * check -- Very simple checkbook program. *
         * *
         * Allows you to add entries to your checkbook *
         * and displays the total each time. *
         * *
         * Restrictions: Will never replace Quicken. *
         ************************************************/
         #include <iostream>
         #include <fstream>
         #include <string>
         #include <vector>
         #include <fstream>
         #include <iomanip>

         /************************************************
         * check_info -- Information about a single *
         * check *
         ************************************************/
         class check_info {
         	public:
         		// Date the check was written
        	 	std::string date;

         		// What the entry is about
         		std::string what;

         		// Amount of check or deposit
         		float amount;
         	public:
         		check_info():
         		date(""),
         		what(""),
         		amount(0.00)
        		 {};
         		// Destructor defaults
         		// Copy constructor defaults
         		// Assignment operator defaults
         	public:
         		void read(std::istream &in file);
         		void print(std::ostream &out_file);
         };

         // The STL vector to hold the check data
         typedef std::vector<check_info> check_vector;

         /************************************************
         * check_info::read -- Read the check *
         * information from a file. *
         * *
         * Warning: Minimal error checking *
         ************************************************/
         void check_info::read(
         	std::istream &in_file // File for input
         ) {
         	std::getline(in_file, date);
         	std::getline(in_file, what);
         	in_file >> amount;
         	in_file.ignore(); // Finish the line
         }
         /************************************************
         * check_info::print -- Print the check *
         * information to a report. *
         ************************************************/
         void check_info::print(
         	std::ostream &out_file // File for output
         ) {
         	out_file <<
         		std::setiosflags(std::ios::left) <<
        		std::setw(10) << date <<
         		std::setw(50) << what <<
         		std::resetiosflags(std::ios::left) <<
         		std::setw(8) << std::setprecision(2) <<
         		std::setiosflags(std::ios::fixed) <<
         		amount << std::endl;
         }

         int main()
         {
         	// Checkbook to test
         	check_vector checkbook;

         	// File to read the check data from
         	std::ifstream in_file("checks.txt");

         	if (in_file.bad()) {
         		std::cerr << "Error opening input file\n";
         		exit (8);
         	}
         	while (1) {
         		check_info next_info; // Current check

         		next_info.read(in_file);
         		if (in_file.fail())
         			break;

         		checkbook.push_back(next_info);
         	}
         	double total = 0.00; // Total in the bank
         	for (check_vector::iterator
         		cur_check = checkbook.begin();
         		cur_check != checkbook.end();
         		cur_check++)
         	{
         		cur_check->print(std::cout);
         		total += cur_check->amount;
         	}
         	std::cout << "Total " << std::setw(62) <<
         	std::setprecision(2) <<
         	total << std::endl;
         	return (0);
         }
    \end{lstlisting}
    %@formatter:on
\end{LTR}