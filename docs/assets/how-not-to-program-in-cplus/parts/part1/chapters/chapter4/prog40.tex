\section[دقت غیر قابل باور]{دقت غیر قابل باور \protect\LTRfootnote{\lr{Unbelievable Accuracy}} (راهنمایی \ref{hint:352}، جواب \ref{answer:73})}
\paragraph{}\label{prog:40}
این برنامه طراحی شده تا دقت اعداد اعشاری را تشخیص دهد. ایدهٔ آن ساده است. مقادیر زیر را محاسبه کن تا جایی که اعداد با هم برابر شوند:
\LTR\noindent
\lr{\texttt{1.0 == 1.5 (1 + ½ or 1+ 1/21) (1.1 binary)}}\\
\lr{\texttt{1.0 == 1.25 (1 + ¼ or 1 + 1/22) (1.01 binary)}}\\
\lr{\texttt{1.0 == 1.125 (1 + 1/8 or 1+ 1/23) (1.001 binary)}}\\
\lr{\texttt{1.0 == 1.0625 (1 + 1/16 or 1+ 1/24) (1.0001 binary)}}\\
\lr{\texttt{1.0 == 1.03125 (1 + 1/32 or 1 + 1/25) (1.00001 binary)}}
\RTL
این کار به ما دقت محاسبات را می‌دهد. این برنامه روی یک کامپیوتر \lr{PC} و اعداد اعشاری 32 بیتی اجرا شد. خب، انتظار داریم چند رقم دودویی در یک عدد اعشاری 32 بیتی باشد؟ این برنامه جواب درست را به ما نمی‌دهد. چرا؟

\begin{LTR}
    %@formatter:off
    \begin{lstlisting}[style=C++Style]
         /************************************************
         * accuracy test. *
         * *
         * This program figures out how many bits *
         * accuracy you have on your system. It does *
         * this by adding up checking the series: *
         * *
         * 1.0 == 1.1 (binary) *
         * 1.0 == 1.01 (binary) *
         * 1.0 == 1.001 (binary) *
         * .... *
         * *
         * Until the numbers are equal. The result is *
         * the number of bits that are stored in the *
         * fraction part of the floating point number. *
         ************************************************/
         #include <iostream>

         int main()
         {
         	/* two numbers to work with */
         	float number1, number2;

         	/* loop counter and accuracy check */
         	int counter;

         	number1 = 1.0;
         	number2 = 1.0;
         	counter = 0;

         	while (number1 + number2 != number1) {
         		++counter; // One more bit accurate

         		// Turn numbers like 0.1 binary
         		// into 0.01 binary.
         		number2 = number2 / 2.0;
         	}
         	std::cout << counter << " bits accuracy.\n";
         	return (0);
         }
    \end{lstlisting}
    %@formatter:on
\end{LTR}