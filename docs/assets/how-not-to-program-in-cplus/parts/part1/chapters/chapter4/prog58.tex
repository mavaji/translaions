\section[اسامی عجیب]{اسامی عجیب \protect\LTRfootnote{\lr{Weird Names}} (راهنمایی \ref{hint:176}، جواب \ref{answer:18})}
\paragraph{}\label{prog:58}
زیربرنامهٔ \lr{\texttt{tmp\_name}} طراحی شده است تا اسم یک فایل موقتی را برگرداند. ایده این است که در هر بار فراخوانی، یک اسم منحصربه‌فرد تولید شود: \lr{\texttt{/var/tmp/tmp.0, /var/tmp/tmp.1, /var/tmp/tmp.2, …}}

اسامیِ تولید شده، یقیناً منحصربه‌فرد هستند ولی نه آنگونه که برنامه‌نویس می‌خواست.

\begin{LTR}
    %@formatter:off
    \begin{lstlisting}[style=C++Style]
         /************************************************
         * tmp_test -- test out the tmp_name function. *
         ************************************************/
         #include <iostream>
         #include <cstdio>
         #include <cstring>
         #include <sys/param.h>
         /************************************************
         * tmp_name -- return a temporary file name *
         * *
         * Each time this function is called, a new *
         * name will be returned. *
         * *
         * Returns: Pointer to the new file name. *
         ************************************************/
         char *tmp_name(void) {
         	// The name we are generating
         	char name[MAXPATHLEN];

         	// The base of the generated name
         	const char DIR[] = "/var/tmp/tmp";

         	// Sequence number for last digit
         	static int sequence = 0;

         	++sequence; /* Move to the next file name */

         	sprintf(name, "%s.%d", DIR, sequence);
         	return(name);
         }
         int main() {
         	char *a_name = tmp_name(); // A tmp name
         	std::cout << "Name: " << a_name << std::endl;
         	return(o);
         }
    \end{lstlisting}
    %@formatter:on
\end{LTR}