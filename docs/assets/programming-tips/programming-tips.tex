% Default Compiler: txs:///xelatex
% Default Bibliography Tool: BibTex

\documentclass[10pt, a4paper]{book}
\usepackage[table, svgnames, usenames,dvipsnames]{xcolor} \usepackage{array}
\usepackage{titlesec}
\usepackage[linktocpage=true,colorlinks,citecolor=blue,pagebackref=true]{hyperref}
\usepackage[top=30mm, bottom=30mm, left=30mm, right=30mm]{geometry}
\usepackage[utf8]{inputenc}
\usepackage{tcolorbox}
\usepackage{listings,lstautogobble}
\usepackage{epsfig,graphicx,subfigure,amsthm,amsmath}
\usepackage{caption}
\usepackage{float}
\usepackage{hyphenat}
\usepackage{setspace}
\usepackage{pgffor}
\makeatletter

%\onehalfspacing
\doublespacing

\usepackage{xepersian}
\settextfont[Scale=1]{IRNazanin}
\defpersianfont\mfo[Scale=1]{IRNazanin}
\setlatintextfont[Scale=0.8]{Doulos SIL}

\renewcommand{\chaptername}{فصل}

\titleformat
{\chapter} % command
[display] % shape
{\bfseries\Large\itshape} % format
{فصل \ \thechapter} % label
{0.5ex} % sep
{
\rule{\textwidth}{1pt}
\vspace{1ex}
\centering
} % before-code
[
\vspace{-0.5ex}%
\rule{\textwidth}{0.3pt}
] % after-code

\DefaultMathsDigits

\definecolor{customblue}{RGB}{235,241,245}
\definecolor{light-gray}{gray}{0.95}
\lstdefinestyle{C++Style}{%
backgroundcolor=\color{customblue},
breaklines=true,
basicstyle=\scriptsize\ttfamily,
keywordstyle=\color{blue},
%commentstyle=\color{OliveGreen}\textit,
commentstyle=\color{OliveGreen},
stringstyle=\color{red},
numbers=left,
numberstyle={\tiny\lr},
showspaces = false,
showstringspaces = false,
tabsize = 2,
frame=single,
xleftmargin=5pt,
xrightmargin=3pt,
language = C++,
aboveskip = 20pt,
rulecolor=\color{black},
captiondirection=RTL,
autogobble=true
}

\lstnewenvironment{C++Code}[1][]
{%
\lstset{style=C++Style, #1}%
}{%
}

\begin{document}
    \title{نکاتی برای برنامه‌نویسان \LTRfootnote{\lr{From The Pragmatic Programmer by Andrew Hunt \& David Thomas}}}
    \author{ترجمهٔ وحید مواجی}
    \date{آذر ۱۳۸۴}
    \frontmatter                            % only in book class (roman page #s)
    \maketitle                              % Print title page.
%    \tableofcontents                        % Print table of contents
    \mainmatter

    \titleformat{\section}{\bfseries}{نکته \thesection, }{0em}{}
    \setcounter{section}{0}
    \renewcommand*\thesection{\arabic{section}}

    \section{به کار خود توجه نشان دهید}
    اگر برایتان مهم نیست که کارتان را درست انجام دهید، چرا عمر خود را با نوشتن نرم‌افزار تلف می‌کنید؟

    \section{دربارهٔ کارتان فکر کنید!}
    کار طوطی‌وار را رها کرده و آگاهانه عمل کنید. به طور مداوم کار خود را نقد و ارزیابی کنید.

    \section{راه چاره‌ای بیندیشید، عذر بدتر از گناه نیاورید}
    به جای معذرت‌خواهی، راه چاره‌ای بیندیشید. نگویید که این کار امکان‌پذیر نیست؛ بگویید که چه چیزی امکان‌پذیر است.

    \section{دست به عصا راه نروید}
    طراحی‌های بد، تصمیمات نادرست و برنامه‌های ضعیف را به محض مشاهده درست کنید.

    \section{تسریع‌دهندهٔ تغییرات باشید}
    نمی‌توانید بقیه را وادار به تغییرات کنید. بلکه به آنها نشان دهید که آینده چگونه خواهد بود و به آنها در ساختن آن آینده کمک کنید.

    \section{اصل قضیه را فراموش نکنید}
    آن قدر در جزئیات غرق نشوید که فراموش کنید در اصل می‌خواستید چه کار کنید.
    
    \section{کیفیت را سرلوحهٔ کار خود قرار دهید}
    کاربرانِ خود را در تعیینِ نیازمندی‌های کیفیتیِ واقعیِ سیستم سهیم سازید.

    \section{همواره برای افزایش دانش خود هزینه کنید}
    به آموختن عادت کنید.

    \section{آن چه می‌خوانید و می‌شنوید را با نگاه نقادانه تحلیل کنید}
    فریفته و سرگردانِ فروشنده‌ها، رسانه‌ها و یا اسیر تعصبات نشوید. تمامیِ اطلاعات را از نقطه‌نظرِ خود و پروژهٔ خود تحلیل کنید.

    \section{هم چیزی که می‌گویید مهم است، هم این که آن را چگونه می‌گویید}
    داشتن ایده‌های بزرگ فایده‌ای ندارد اگر ندانید چگونه آنها را به دیگران انتقال دهید.

    \section{در ورطهٔ تکرار نیفتید}
    هر دانشی باید یک نمودِ عینیِ واحد، واضح و معتبر در سیستم داشته باشد.

    \section{استفادهٔ مجدد را آسان سازید}
    فقط اگر استفادهٔ مجدد آسان باشد، مردم این کار را می‌کنند. محیطی برای استفادهٔ مجدد فراهم سازید.
    
    \section{چیزهای نامرتبط نباید بر هم تأثیری بگذارند}
    اجزایی را طراحی کنید که متکی به خود و مستقل باشند و هدفِ واحد و تعریف‌شده‌ای داشته باشند.

    \section{هیچ تصمیمی نهایی نیست}
    هیچ تصمیمی برای همیشه پابرجا نیست. بلکه فکر کنید همهٔ تصمیم‌ها را روی شنِ دریا نوشته‌اید و برنامه‌ای برای تغییراتِ مناسب داشته باشید.

    \section{برای پیدا کردنِ هدف از منوّر استفاده کنید}
    منوّرها به شما اجازه می‌دهند که با سعی‌های مکرر در تاریکی ببینید چقدر از هدف خود فاصله دارید.

    \section{از مدلِ اولیه برای یادگیری استفاده کنید}
    مدل‌سازیِ اولیه، یک روش یادگیری است. ارزش آن فقط در برنامه‌ای که می‌نویسید نیست بلکه در درس‌هایی است که از آن می‌آموزید.

    \section{برنامهٔ خود را به مسألهٔ واقعی نزدیک سازید}
    به زبان کاربران خود طراحی و برنامه‌نویسی کنید.

    \section{برای غافلگیر نشدن، کارها را ارزیابی کنید}
    قبل از این که شروع کنید، همه چیز را ارزیابی کنید. در این صورت مشکلاتِ احتمالی را همان اول مدّ نظر خواهید داشت.

    \section{زمان‌بندیِ برنامه را بررسی کنید}
    از تجربه‌ای که هنگام پیاده‌سازی برنامه به دست می‌آورید برای ویرایشِ زمان‌بندیِ پروژه استفاده نمایید.

    \section{دانشِ خود را به متنِ ساده نگهداری کنید}
    متن ساده هیچ‌گاه منسوخ نخواهد شد. بلکه کمک می‌کند تا در کارِ خود مسلط شوید و همچنین باگ‌زدایی و آزمایش برنامه را ساده‌تر خواهد ساخت.

    \section{از قدرتِ کنسول‌های فرمان بهره گیرید}
    وقتی که واسط‌های گرافیکی جواب‌گو نمی‌باشند، از کنسول‌های متنی استفاده کنید.

    \section{به خوبی از یک ویرایشگر متن استفاده کنید}
    ویرایشگر متنِ موردِ نظر باید مثل موم در دستان شما باشد؛ مطمئن شوید که ویرایشگر شما قابل‌تنظیم و قابل‌گسترش است و قابلیتِ برنامه‌نویسی دارد.

    \section{همیشه از سیستم‌های کنترل کُد منبع استفاده کنید}
    سیستم کنترل کُد منبع به مثابه یک ماشین زمان می‌ماند که می‌توانید تغییرات کارِ خود را در زمان نگهدارید.

    \section{مشکل را حل کنید، سرزنش نکنید}
    واقعاً اهمیتی ندارد که ایراد برنامه تقصیر شماست یا کَسِ دیگری. این مسأله کماکان مشکل شماست و باید توسط شما رفع گردد.

    \section{موقع باگ‌زداییِ برنامه، پریشان نشوید}
    یک نفس عمیق بکشید و دربارهٔ علت مشکل فکر کنید!

    \section{فرافکنی نکنید}
    بسیار به ندرت اتفاق می‌افتد که مشکلی در سیستم‌عامل یا کامپایلر یا حتی محصولِ جانبی که خریده‌اید وجود داشته باشد. اغلب اوقات مشکل از برنامهٔ خودتان است.

    \section{نظریه‌پردازی نکنید، ادعای خود را اثبات کنید}
    نظریه‌های خود را در محیط واقعی با داده‌ها و محدودیت‌های واقعی اثبات کنید.

    \section{یک زبان کار با متن بیاموزید}
    قسمت عمده‌ای از هر روزِ خود را به کار با متن می‌گذرانید. چرا نمی‌گذارید که قسمتی از این کار را کامپیوتر برای شما انجام دهد؟

    \section{برنامه‌ای بنویسید که خودش برنامه بنویسد}
    تولیدکننده‌های خودکارِ برنامه، بازدهی شما را افزایش داده و مانعِ دوباره‌کاری می‌شوند.

    \section{نمی‌توانید نرم‌افزاری عالی بنویسید}
    نرم‌افزار نمی‌تواند عالی و بی‌عیب باشد. کاری کنید که برنامهٔ شما و کاربرانتان از خطاهای اجتناب‌ناپذیر محافظت شوند.

    \section{طبق قرارداد طراحی کنید}
    برای مستندسازی از قرارداد استفاده کنید و مطمئن شوید که برنامهٔ شما چیزی بیشتر یا کمتر از چیزی که ادعا می‌کند، انجام ندهد.

    \section{هر چه زودتر برنامه را تا حد مرگ تحت فشار بگذارید}
    یک برنامهٔ مُرده معمولاً بسیار کمتر از یک برنامهٔ چلاق، آسیب می‌رساند.

    \section{از روش‌های مطمئن‌سازی برای جلوگیری از روز مبادا استفاده نمایید}
    روش‌های مطمئن‌سازی، اعتبار فرضیات شما را می‌سنجند. از آنها برای محافظتِ برنامهٔ خود در برابر دنیای نامطئن استفاده کنید.

    \section{برای مواردِ استثنایی از استثنائات استفاده نمایید}
    استثنائات می‌توانند از همهٔ مشکلاتِ مربوط به خوانایی و نگهداری که در برنامه‌های قدیمی وجود داشت رنج ببرند. استثنائات را برای مواردِ استثنایی بگذارید.
    
    \section{آن چه را که شروع می‌کنید به پایان برسانید}
    تا آنجا که ممکن است، هر رویه یا شیئ‌ای که منبعی را در اختیار می‌گیرد باید بعد از اتمامِ کارش آن را آزاد سازد.

    \section{اتصال بین ماژول‌ها را به حداقل برسانید}
    از نوشتن برنامه‌های «کم‌رو» که متکی بر دیگر برنامه‌ها هستند اجتناب کنید.

    \section{پیکربندی کنید نه این که همه چیز را ثابت قرار دهید}
    دست کاربر را در انتخاب تنظیمات مختلف برای نرم‌افزار باز بگذارید نه این که همه چیز را به طور ثابت در نرم‌افزار تعبیه سازید.

    \section{تجرید را در برنامه قرار دهید، جزئیات را در فایل‌های جانبی}
    برای حالت کلی برنامه بنویسید و جزئیاتِ حالت‌های خاص را در فایل‌های جانبی و بیرون از برنامه قرار دهید.

    \section{جریان‌کار را برای بهبود هم‌زمانی مورد بررسی قرار دهید}
    از جریان‌کارِ کاربرانِ خود، هم‌زمانی‌های ممکن را استخراج کنید.

    \section{با استفاده از سرویس‌ها طراحی کنید}
    با استفاده از سرویس‌ها، پیاده‌سازی‌ها را پُشتِ واسط‌هایی که به خوبی طراحی و تعریف شده‌اند، پنهان سازید.

    \section{همیشه هم‌زمانی را هنگام طراحی درنظر داشته باشید}
    اگر فرض را بر هم‌زمانی بگذارید، واسط‌های بهتری طراحی خواهید کرد و فرضیات کمتری خواهید داشت.

    \section{ویو‌ها را از مدل جدا سازید}
    با طراحی بر مبنای مدل-ویو، با هزینه‌ای کم، انعطاف‌پذیریِ زیادی به نرم‌افزارِ خود می‌بخشید.

    \section{برای هماهنگ کردن جریان‌کارها از تخته‌سیاه استفاده نمایید}
    از تخته‌سیاه برای هماهنگیِ فَکت‌ها و عواملِ مختلف استفاده نمایید و درعین‌حال استقلالِ موجود بین همکاران را حفظ کنید.

    \section{الله‌بختکی برنامه ننویسید}
    فقط به چیزهای مطمئن اتکا کنید. از پیچیدگیِ تصادفی آگاه باشید و یک اتفاقِ خوشحال‌کننده را در یک برنامه‌ریزیِ هدفمند وارد نسازید.

    \section{مرتبهٔ الگوریتمِ خود را تخمین بزنید}
    قبل از این که برنامه‌ای بنویسید، ارزیابی کنید که آن برنامه چقدر طول می‌کشد تا اجرا شود.

    \section{تخمین‌های خود را بیازمایید}
    تحلیل ریاضیِ الگوریتم‌ها همه چیز نیست. حتماً برنامهٔ خود را در محیط واقعی اجرا کرده و زمانِ اجرای آن را در نظر بگیرید.
    
    \section{برنامهٔ خود را همیشه تمیز نگه دارید}
    همان طور که ممکن است باغچهٔ خود را وِجین کرده و علف‌های هرز را قطع کنید، هر زمانی که برنامهٔ شما نیاز دارد، آن را دوباره بنویسید، دوباره انجام دهید و در معماریِ آن تجدیدنظر کنید. همیشه سعی کنید ریشهٔ مشکلات را قطع کنید.
    
    \section{موقع طراحی، تست را فراموش نکنید}
    قبل از این که حتی یک خط برنامه بنویسید، به روشِ تست و آزمایشِ آن بیندیشید.

    \section{یا خودتان نرم‌افزار را تست کنید یا این که کاربرانِ عصبانی این کار را خواهند کرد}
    برنامهٔ خود را بی‌رحمانه تست کنید تا به کاربران اجازه ندهید اشکالاتِ آن را برایتان پیدا کنند.

    \section{از برنامه‌های ویزاردی که عملکردِ آنها را نمی‌فهمید استفاده نکنید}
    ویزاردها می‌توانند یک مثنویِ هفتادمن‌کاغذ از برنامه تولید کنند. قبل از این که برنامهٔ تولیدشده را وارد نرم‌افزارِ خود کنید مطمئن شوید که روشِ کارِ آن را فهمیده‌اید.

    \section{نیازمندی‌ها را به طور عمقی مورد بررسی قرار دهید}
    به‌ندرت پیش می‌آید که نیازمندی‌های کاربر همان چیزی باشد که می‌گوید، نیازهای واقعی او زیر یک خروار فرضیات، سوتفاهمات و رِندی‌ها پوشیده شده است.

    \section{با یک کاربر کار کنید تا مثل یک کاربر بیندیشید}
    این بهترین روشی است که درخواهیدیافت نرم‌افزارِ شما چگونه مورد استفاده قرار خواهد گرفت.

    \section{تجریدها بیشتر از جزئیات عمر می‌کنند}
    روی تجرید هزینه کنید نه روی پیاده‌سازی. تجریدها زیرِ رگبارِ تغییراتِ پیاده‌سازی و تکنولوژی‌های جدیدی که می‌رسند به حیاتِ خود ادامه می‌دهند.

    \section{برای پروژهٔ خود یک واژه‌نامه فراهم سازید}
    تمام اصطلاحات و واژگانِ خاصّی که در پروژهٔ شما مورد استفاده قرار می‌گیرند باید در یک مستند شرح داده شوند.

    \section{خارج از گود ننشینید، وارد گود شوید}
    وقتی با یک مسألهٔ لاینحل مُواجِه می‌شوید، محدودیت‌های اصلی را مشخص سازید. از خود بپرسید: آیا باید این طور انجام شود؟ آیا اصلاً باید انجام شود؟
    
    \section{وقتی شروع کنید که آماده هستید}
    همیشه از تجربه درس بگیرید. هیچ وقت شک‌های آزاردهنده را کم‌اهمیت تلقی نکنید.

    \section{بعضی از چیزها را نباید توضیح داد، باید انجام داد}
    وقتی لازم است که شروع به برنامه‌نویسی کنید، در منجلابِ مستندسازی نیفتید.

    \section{بردهٔ تقلید نباشید}
    تا زمانی که یک تکنولوژی را در محیط پروژهٔ خود و تحت قابلیت‌های نرم‌افزارِ خود مورد آزمایش قرار نداده‌اید، هیچ‌گاه کورکورانه از آن استفاده نکنید.
    
    \section{ابزارهای گران‌قیمت منجر به طراحی‌های بهتری نمی‌شوند}
    فریفتهٔ شعارهای تبلیغاتی، قیمت‌های گزاف و قضاوت دیگران نشوید. ابزارهای طراحی خود را با معیارهای مناسب و مفید به پروژهٔ خود بسنجید.
    
    \section{تیم‌های خود را بر اساس عملکرد، سازماندهی کنید}
    طراح‌ها را از برنامه‌نویسان و تست‌کننده‌ها را از مدل‌گر‌ها جدا نسازید .تیم‌ها را براساس کاری که انجام می‌دهند و هدف نهایی که به آن خواهند رسید بچینید.

    \section{از رویه‌های دستی استفاده نکنید}
    یک اسکریپت یا فایلِ دسته‌ای همان دستورات را به همان ترتیب و پشتِ سرِ هم انجام می‌دهد.

    \section{زود تست کنید، همیشه تست کنید. به طور خودکار تست کنید}
    تست‌هایی که در اوایل کار انجام می‌شوند بسیار موثرتر از برنامه‌های تست عریض و طویلی هستند که گوشه‌ای خاک می‌خورند.

    \section{تا زمانی که تست‌ها اجرا نشده باشند، برنامه‌نویسی به پایان نرسیده است}
    یعنی همین!

    \section{از خرابکاران برای تست کردنِ تستِ خود استفاده کنید}
    برای این که مطمئن شوید ابزارِ تست شما درست کار می‌کند، ایرادهایی عمدی در برنامهٔ خود وارد کنید تا ببینید آیا کشف می‌شوند یا نه.
    
    \section{خطوط برنامه را تست نکنید، حالت‌های مختلف برنامه را تست نمایید}
    حالت‌های مهمی که برنامه با آنها مُواجِه می‌شود را مشخص کرده و تست کنید. فقط تست کردنِ خطوط برنامه دردی را دوا نمی‌کند.

    \section{ایرادها را فقط یک بار بیابید}
    وقتی که شخصِ تست‌کننده ایرادی را پیدا می‌کند، باید آخرین باری باشد که آن را می‌یابد. تست‌کننده‌های خودکار باید از این به بعد آن ایراد را پیگیری کنند.

    \section{انگلیسی فقط یک زبان برنامه‌نویسی است}
    مستندات را همان طور بنویسید که برنامه‌ها را می‌نویسید. از \lr{MVC}، تولیدکننده‌های خودکار و فایل‌های کمکی بهره گیرید.

    \section{مستندات را روی هوا نسازید}
    مستنداتی که جداگانه از برنامه ایجاد شده‌اند بسیار نامحتمل است که به‌روز و صحیح باشند.

    \section{به مرورِ زمان از انتظارات کاربران خود فراتر روید}
    سعی کنید انتظاراتِ کاربرانِ خود را درک کنید و فقط کافیست کمی بیشتر از آن چه که می‌خواستند انجام دهید.

    \section{کارِ خود را امضا کنید}
    هنرمندانِ گذشته به امضای کارهای خود افتخار می‌کردند. شما هم باید برنامه‌ای بنویسید که به امضای خود در پای آن افتخار کنید.
\end{document}