% Default Compiler: txs:///xelatex
% Default Bibliography Tool: BibTex

\documentclass[12pt]{book}
\usepackage[x11names]{xcolor}
\usepackage{titlesec}
\usepackage[linktocpage=true,colorlinks,citecolor=blue,pagebackref=true]{hyperref}
\usepackage[top=30mm, bottom=30mm, left=30mm, right=30mm]{geometry}
\usepackage[utf8]{inputenc}
%\newcommand{\noun}[1]{\textit{\textcolor{black!60}{#1}}}
\newcommand{\noun}[1]{«{#1}»}

\usepackage{setspace}
%\onehalfspacing
\doublespacing

\usepackage{xepersian}
\settextfont[Scale=1.2]{IRNazanin}
\defpersianfont\mfo[Scale=1.2]{IRNazanin}
\setlatintextfont[Scale=1]{Doulos SIL}

%\titleformat{\chapter}
%{\normalfont\huge\bfseries}{\chaptertitlename\ \thechapter.}{20pt}{\huge}
%\renewcommand \chaptertitlename {اپیزود}
\renewcommand{\chaptername}{اپیزود}

\titleformat
{\chapter} % command
[display] % shape
{\bfseries\Large\itshape} % format
{اپیزود \ \thechapter} % label
{0.5ex} % sep
{
\rule{\textwidth}{1pt}
\vspace{1ex}
\centering
} % before-code
[
\vspace{-0.5ex}%
\rule{\textwidth}{0.3pt}
] % after-code

\begin{document}
    \title{افسانهٔ سیزیف}
    \author{آلبر کامو\\
    ترجمهٔ وحید مواجی\\
    ترجمه از فرانسه: ژاستین اوبراین، ۱۹۵۵
    }
    \date{مرداد ۱۳۸۵}
    \frontmatter                            % only in book class (roman page #s)
    \maketitle                              % Print title page.
    \tableofcontents                        % Print table of contents
    \mainmatter


    \part{افسانهٔ سیزیف}
    \paragraph{}
    خدایان سیزیف را محکوم کرده بودند که دائما سنگی را به بالای کوهی بغلتاند، تا جائی که سنگ به خاطر وزنش به پایین می‌افتاد. آنها به دلائلی فکر می‌کردند که تنبیهِ وحشتناک‌تری از کارِ عبث و بی‌امید وجود ندارد.

    \paragraph{}
    اگر کسی به هومر معتقد باشد، سیزیف خردمندترین و محتاط‌ترین موجوداتِ فانی بود. هر چند، بنا بر روایتی دیگر، او مایل بود تا حرفهٔ راهزنی را بیازماید. من تضادی در این امر نمی‌بینم. عقایدِ مختلفی وجود دارد که چرا او کارگرِ پوچ و عبثِ جهانِ زیرین شد. اولاً، او متهم به سبک‌سری در رفتار با خدایان است. او اسرارِ آنان را دزدید. اگینا، دخترِ اسوپوس، توسطِ ژوپیتر ربوده شد. پدرش از ناپدید شدنِ او به هراس آمده و به سیزیف شکایت برد. او که از جریانِ ربودن باخبر بود، به این شرط حاضر شد ماجرا را بگوید که اسوپوس به قلعهٔ کورینث، آب برساند. به خاطرِ آذرخش‌های آسمانی، او برکتِ آب را ترجیح داد. به همین دلیل او در جهانِ زیرین تنبیه شد. هومر همچنین می‌گوید که سیزیف، مرگ را در زنجیر کرده بود. پلاتو نمی‌توانست منظرهٔ فرمانرواییِ ویران و ساکتِ او را تحمل کند. او خدایِ جنگ را گسیل داشت تا مرگ را از دستانِ اشغالگر آزاد سازد.
    
    \paragraph{}
    گفته می‌شود که سیزیف وقتی نزدیک به مرگ بود، به طور بی‌ملاحظه‌ای می‌خواست عشقِ زنِ خود را بیازماید. او به زنش فرمان داد که بدنِ دفن نشدهٔ خود (سیزیف) را در وسطِ میدانِ عمومی قرار دهد. سیزیف در جهانِ زیرین بیدار شد و آنجا در حالی که از فرمانبرداریِ بسیار متضاد با عشقِ انسانی رنج می‌برد، این اجازه را از پلاتو گرفت که به زمین برگردد تا زنِ خود را ملامت کند. ولی وقتی دوباره چشمش به دنیا باز شد، از آب و خورشید، سنگ‌های گرم و دریا لذت برد، دیگر نمی‌خواست به آن تاریکیِ دوزخ‌وار برگردد. فراخوانی‌ها، علائمِ خشم و اخطارها هیچ کدام کارگر نیفتاد. سال‌هایِ زیادِ دیگری را هم با انحنایِ خلیج، دریایِ تابان و لبخندِ زمین زندگی کرد. حکمی از جانب خدایان ضروری به نظر می‌رسید. عطارد آمد و گریبانِ مرد گستاخ را گرفت، او را از لذاتِ خود جدا ساخت و به زور به جهانِ زیرین برد، جایی که سنگش انتظارِ او را می‌کشید.
    
    \paragraph{}
    تا حالا دریافته‌اید که سیزیف، قهرمانِ پوچی است. او همانقدر که لذت می‌برد، عذاب می‌کشد. تمسخرِ خدایان از جانبِ او، نفرتِ او از مرگ و اشتیاق او برایِ زندگی، آن مجازاتِ وصف‌ناشدنی را برای او به ارمغان آورد که تمامِ وجودش باید برای انجام دادنِ هیچ، به کار رود. این هزینه‌ای است که باید برای اشتیاق به زندگی پرداخته شود. چیزی از دنیای زیرین دربارهٔ سیزیف به ما گفته نشده است. افسانه‌ها برای تصورات به وجود می‌آیند تا در آنها روح بدمند. در موردِ این افسانه، تمامِ تلاشِ یک شخص برای بالا بردنِ یک سنگِ عظیم، چرخاندنِ آن و هل دادنِ آن به سمتِ بالا روی یک سطحِ شیب‌دار برای صدها بار را می‌توان دید؛ صورتِ رنجور، گونهٔ چسبیده به سنگ، شانه‌هایی زیرِ توده‌ای از خاک، پاهای از هم وارفته، شروعِ دوباره با بازوان گشاده و تمامِ امنیتِ انسانیِ دستانِ پینه‌بسته را می‌توان دید. در پایانِ تلاشِ بی‌پایانِ او در فضا و زمانِ بی‌نهایت، هدف برآورده می‌شود. آنگاه سیزیف می‌بیند که سنگ در چند لحظه به سمتِ دنیای پائین‌تر می‌غلتد و به جایی می‌رود که دوباره باید آن را به سمتِ قلّه راند. او دوباره به پایین برمی‌گردد.
    
    \paragraph{}
    در طیِ این بازگشت، این وقفه، است که سیزیف نظرِ مرا به خود جلب می‌کند. صورتی که بدین حد به سنگ‌ها نزدیک است و رنج می‌کشد، خودش اکنون سنگ شده است! مردی را ببینید که با گام‌هایی سنگین و در عین حال شمرده به عذابی بازمی‌گردد که هیچ‌گاه پایانِ آن را نخواهد دانست. آن موقع مثلِ مکثی که با رنج او فرا می‌رسد، زمانِ هوشیاری است. در تک‌تکِ آن لحظات که او ارتفاعات را ترک می‌کند و تدریجاً به کنامِ خدایان کشیده می‌شود، مافوقِ سرنوشتِ خود قرار دارد. او از سنگِ خود سخت‌تر است.
    
    \paragraph{}
    اگر این افسانه غم‌انگیز است، به خاطر اینست که قهرمانِ آن هشیار است. در واقع اگر در هر قدم، امیدِ موفقیت به او دلگرمی می‌دهد، شکنجه‌ای وجود دارد؟ کارگرِ امروزی، در هر روزِ زندگی‌اش کارِ یکسانی می‌کند و سرنوشتِ او کمتر از سرنوشتِ سیزیف، پوچ نیست. ولی فقط در لحظاتِ نادری، غم‌انگیز می‌شود که هشیاری وجود دارد. سیزیف، کارگرِ خدایان، ناتوان و سرکش، تمامِ جزئیاتِ وضعیتِ تأسف‌آورِ خود را می‌داند: این چیزی است که در طیِ هبوطِ خود به آن می‌اندیشد. روشن‌بینی که قرار بود شکنجهٔ او باشد، به تاجِ پیروزیِ او تبدیل می‌شود. هیچ سرنوشتی وجود ندارد که نتوان با استهزاء بر آن فائق آمد.
    
    \paragraph{}
    بنابراین اگر هبوط، بعضی مواقع با غصّه همراه بود، می‌تواند با شادی نیز همراه باشد. این حرفِ زیادی نیست.
    
    \paragraph{}
    باز هم من سیزیف را تصور می‌کنم که به سوی سنگِ خود برمی‌گردد و غصّه دوباره آغاز می‌گردد. وقتی تصوراتِ زمین در حافظه حک می‌شوند، وقتی خاطراتِ شادی دست از سرِ آدم برنمی‌دارند، قلبِ انسان، سودازده می‌شود: این پیروزیِ سنگ است، اصلاً خودِ سنگ است. اندوهِ بیکران را نمی‌توان تحمل کرد. این‌ها، شبهای گتسمانِ ما هستند. ولی حقایقِ نابودکننده، اگر تصدیق شوند، کشنده خواهند بود. بنابراین ادیپ در آغاز بدون این که بداند تسلیمِ قسمت است. ولی از لحظه‌ای که آگاه می‌گردد، تراژدیِ او آغاز می‌شود. در همان لحظه، کور و بی‌امید، درمی‌یابد که تنها حلقهٔ متصل‌کنندهٔ او به جهان، دستانِ آرامِ دختری است. آنگاه نکتهٔ شگرفی طنین‌انداز می‌شود: «علی‌رغم این همه کارهای شاق، سنِ زیاد و اصالتِ روحم مرا به این نتیجه می‌رساند که همه چیز خوب است». لذا ادیپِ سوفوکل، مثل کیریلفِ داستایفسکی نسخهٔ پیروزیِ پوچ و بی‌معنی را می‌پیچد. خردِ باستان، شجاعتِ مدرن را تأیید می‌کند.
    
    \paragraph{}
    کسی، پوچی را درنمی یابد مگر اینکه وسوسه شود تا دستورالعملی برای شادی بنویسد. «چی! با این روش‌های مبتذل؟» با این حال دنیایی وجود دارد. شادی و پوچی، دو پسرِ یک زمین هستند. آنها جدایی‌ناپذیرند. اشتباه خواهد بود اگر بگوییم که لزوماً شادی از کشفِ پوچی سرچشمه می‌گیرد. همچنین اگر بگوییم از بین رفتنِ پوچی، ناشی از شادی است. ادیپ می‌گوید «من نتیجه می‌گیرم که همه چیز خوب است» و این جمله مقدس است. این جمله در جهانِ وحشی و محدودِ انسان طنین‌انداز می‌شود. به ما می‌آموزد که «همه»، خسته‌کننده نیست و نبوده است. از این دنیا، خدایی را بیرون می‌اندازد که با ناخشنودی و ترجیحِ رنجِ بیهوده به آن وارد شده بود. از قسمت، یک مسألهٔ انسانی می‌سازد که باید با انسان‌ها همنشین شود.
    
    \paragraph{}
    تمام شادیِ خاموشِ سیزیف، در این امر نهفته است. قسمتِ او، مالِ خودش است. سنگش نیز همینطور. مردِ ناامید وقتی به شکنجهٔ خود می‌اندیشد، تمامِ خدایانِ دروغین را سرِ جای خود می‌نشاند. در جهانی که ناگهان به سکوتِ خود بازگشته است، ده‌ها هزار صدای کوچکِ سرگردان برمی‌خیزد. نداهای ناخودآگاه و مخفی و دعوت‌ها از هر طرف، بازگشتِ ضروری و هزینهٔ پیروزی هستند. بدونِ سایه، خورشیدی نخواهد بود و شناختنِ شب واجب است. انسانِ ناامید می‌گوید «آری» و لذا تلاش‌هایش ازین پس بی‌پایان خواهد بود. اگر قسمتِ شخصی وجود داشته باشد، سرنوشتِ بالاتری وجود نخواهد داشت یا حداقل سرنوشتی وجود دارد که او نتیجه می‌گیرد ناگزیر و پست است. اما در موردِ باقیِ مطالب، او درمی‌یابد که خداوندگارِ روزگارِ خود است. در آن لحظهٔ ظریف، نظری به عقب بر زندگیِ خود می‌اندازد، سیزیف که به سوی سنگِ خود برمی‌گردد، در آن چرخشِ ناچیز، او به آن اعمالِ نامرتبطی که سرنوشتِ او را تشکیل داده‌اند، توسطِ او به وجود آمده‌اند و در سایهٔ حافظهٔ او ترکیب شده‌اند و با مرگِ او مُهر و موم شده‌اند می‌اندیشد. بنابراین، بشر، متقاعد به این که تمامِ سرچشمه‌های این اتفاقات، انسان است، انسانِ نابینایی که مشتاقِ دانستنِ این است که چه کسی می‌داند شب، انتهایی ندارد، همچنان در حرکت است. سنگ همچنان می‌چرخد.
    
    \paragraph{}
    من سیزیف را در پایینِ کوه رها می‌کنم! انسان همیشه راهِ خود را می‌یابد. ولی سیزیف صداقتِ بالاتری را آموزش می‌دهد که خدایان را نفی کرده و سنگ‌ها را می‌چرخاند. او همچنین نتیجه می‌گیرد که همه چیز خوب است. این جهان از این پس بدونِ خدا، به نظرِ او نه بی‌حاصل است و نه پوچ. هر اتمِ آن سنگ، هر تکهٔ آن کوهستانِ غرق در شب، برای خود دنیایی است. فقط تلاش برای غلبه بر ارتفاع، برای ارضای قلبِ انسان کافی است. باید سیزیف را شاد بپنداریم.
\end{document}