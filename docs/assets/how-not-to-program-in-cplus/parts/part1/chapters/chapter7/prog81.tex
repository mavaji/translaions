\section[سپاس برای حافظه]{سپاس برای حافظه \protect\LTRfootnote{\lr{Thanks for the Memory}} (راهنمایی \ref{hint:56}، جواب \ref{answer:32})}
\paragraph{}\label{prog:81}
چرا این برنامه، حافظه کم می‌آورد؟

\begin{LTR}
    %@formatter:off
        \begin{lstlisting}[style=C++Style]
             /************************************************
             * play with a variable size stack class. *
             ************************************************/

             /************************************************
             * stack -- Simple stack class *
             * *
             * Member functions: *
             * push -- Push data on to the stack *
             * pop -- remove an item from the stack. *
             ************************************************/
             class stack
             {
             	private:
             		int *data; // The data
             		const int size; // The size of the data

             		// Number of items in the data
             		int count;
             	public:
             		// Create the stack
             		stack(
             			// Max size of the stack
             			const int _size
             		):size(_size), count(0)
             		{
             			data = new int[size];
             		}
             		~stack(void) {}
             	private:
             		// No copy constructor
             		stack(const stack &);

             		// No assignment operator
             		stack & operator = (const stack &);
             	public:
             		// Push something on the stack
             		void push(
            			// Value to put on stack
             			const int value
             		)
             		{
             			data[count] = value;
             			++count;
             		}
            		// Remove an item from the stack
             		int pop(void)
             		{
             			--count;
             			return (data[count]);
             		}
             };

             int main()
             {
             	stack a_stack(30);

             	a_stack.push(1);
             	a_stack.push(3);
             	a_stack.push(5);
             	a_stack.push(7);
             	return (0);
             }
        \end{lstlisting}
    %@formatter:on
\end{LTR}