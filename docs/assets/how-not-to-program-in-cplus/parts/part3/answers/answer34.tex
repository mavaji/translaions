\section{}
\paragraph{}\label{answer:34}
در انتهای عبارت \lr{\texttt{for}} یک نقطه‌ویرگول اضافی وجود دارد:
\begin{LTR}
    %@formatter:off
    \begin{lstlisting}[style=C++Style]
        for (index = 1; index <= 10; ++index);
    \end{lstlisting}
    %@formatter:on
\end{LTR}

این بدین معناست که \lr{\texttt{for}} مطلقاً کاری نمی‌کند. برنامهٔ درست‌توگذاری‌شده این است:
\begin{LTR}
    %@formatter:off
    \begin{lstlisting}[style=C++Style]
        for (index = 1; index <= 10; ++index);
        std::cout << index << " squared " <<
                (index * index) << '\n';
    \end{lstlisting}
    %@formatter:on
\end{LTR}
یا اگر کمی توضیح به آن اضافه کنیم، این شکلی می‌شود:
\begin{LTR}
    %@formatter:off
    \begin{lstlisting}[style=C++Style]
        for (index = 1; index <= 10; ++index)
            /* Do nothing */;
        std::cout << index << " squared " <<
            (index * index) << '\n';
    \end{lstlisting}
    %@formatter:on
\end{LTR}
می‌توانیم دریابیم که \lr{\texttt{std::cout}} درونِ حلقهٔ \lr{\texttt{for}} نیست.