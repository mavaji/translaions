\section{}
\paragraph{}\label{answer:84}
 مشکل، بهینه‌ساز است. بهینه‌ساز می‌داند که متغیر \lr{\texttt{debugging}} برابر صفر است. همیشه صفر است. حالا که این را می‌دانیم، بگذارید نگاهی به عبارت \lr{\texttt{if (debugging)}} بیندازیم. این همیشه غلط است، چون \lr{\texttt{debugging}} همیشه صفر است. بنابراین این بلوک هیچ‌گاه اجرا نمی‌شود. یعنی می‌توانیم کد زیر را:
\begin{LTR}
    %@formatter:off
        \begin{lstlisting}[style=C++Style]
            13 if (debugging)
            14 {
            15 	    dump_variables();
            16 }
        \end{lstlisting}
    %@formatter:on
\end{LTR}

 به این صورت بهینه‌سازی کنیم:
 \begin{LTR}
    %@formatter:off
        \begin{lstlisting}[style=C++Style]
            // Nothing
        \end{lstlisting}
    %@formatter:on
\end{LTR}

حال بگذارید ببینیم تعداد دفعاتی که \lr{\texttt{debugging}} مورد استفاده قرار می‌گیرد، چقدر است. در خط 11 مقداردهی اولیه می‌شود و در خط 13 استفاده می‌شود. خط 13 مورد بهینه‌سازی قرار گرفته است لذا \lr{\texttt{debugging}} هیچ‌گاه استفاه نمی‌شود. اگر از متغیری هیچ‌گاه استفاده نشود، می‌توان آن را با بهینه‌سازی حذف کرد. نتیجه، برنامهٔ بهینه‌سازی‌شدهٔ زیر است:
\begin{LTR}
    %@formatter:off
        \begin{lstlisting}[style=C++Style]
            9   void do_work()
            10 {
            11 	// Declaration optimized out
            12
            13 	// Block optimized out
            14 	//
            15 	//
            16 	// End of block that was removed
            17 	// Do real work
            18 }
        \end{lstlisting}
    %@formatter:on
\end{LTR}

حالا برنامه‌نویس ما می‌خواهد از متغیر \lr{\texttt{debugging}} برای کمک کردن در امر دیباگ استفاده کند. مشکل این جاست که بعد از بهینه‌سازی، متغیر \lr{\texttt{debugging}}ای وجود ندارد. مشکل این جاست که \lr{\texttt{C++}} نمی‌دانست که برنامه‌نویس می‌خواست از جادو (یک دیباگِر) برای تغییر متغیرها استفاده کند. اگر می‌خواهید کاری این‌چنین انجام دهید، باید به کامپایلر بگویید. این کار با اعلان متغیر \lr{\texttt{debugging}} به صورت \lr{\texttt{volatile}} انجام‌پذیر است.
\begin{LTR}
    %@formatter:off
        \begin{lstlisting}[style=C++Style]
            static volatile int debugging = 0;
        \end{lstlisting}
    %@formatter:on
\end{LTR}

کلمه کلیدی \lr{\texttt{"volatile"}} به \lr{\texttt{C++}} می‌گوید که «چیزی عجیب و غریب مثل یک روتین وقفه، یک دستور دیباگر، یا چیز دیگری ممکن است این متغیر را تغییر دهد. نمی‌توانی هیچ فرضی در مورد مقدار آن داشته باشی».