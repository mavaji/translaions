\section[نقطهٔ بدون بازگشت]{نقطهٔ بدون بازگشت \protect\LTRfootnote{\lr{Point of No Return}} (راهنمایی \ref{hint:349}، جواب \ref{answer:5})}
\paragraph{}\label{prog:101}
چرا برنامهٔ زیر یک فایلِ درست را در یونیکس می‌نویسد و یک فایلِ اشتباه را در ویندوز؟ برنامه، 128 کاراکتر را می‌نویسد ولی نسخهٔ ویندوز شامل 129 بایت است. چرا؟

\begin{LTR}
    %@formatter:off
        \begin{lstlisting}[style=C++Style]
             /*************************************************
             * Create a test file containing binary data. *
             *************************************************/
             #include <iostream>
             #include <fstream>
             #include <stdlib.h>

             int main()
             {
            	// current character to write
             	unsigned char cur_char;

             	// output file
             	std::ofstream out_file;

             	out_file.open("test.out", std::ios::out);
             	if (out_file.bad())
             	{
             		std::cerr << "Can not open output file\n";
             		exit (8);
             	}

             	for (cur_char = 0;
             		cur_char < 128;
             		++cur_char)
             	{
             		out_file << cur_char;
             	}
             	return (0);
             }
        \end{lstlisting}
    %@formatter:on
\end{LTR}


\begin{tcolorbox}
    انسان جایزالخطاست. برای این که واقعاً کارها را خراب کنید به یک کامپیوتر نیاز دارید. برای این که به خراب کردن ادامه دهید به بوروکراسی احتیاج دارید.
\end{tcolorbox}