\section[تجدید نظر در ریاضیات دبستانی]{تجدید نظر در ریاضیات دبستانی \protect\LTRfootnote{\lr{Kindergarten Arithmetic Revision}} (راهنمایی \ref{hint:113}، جواب \ref{answer:54})}
\paragraph{}\label{prog:39}
همهٔ ما می‌دانیم که 2 = 1 + 1 و 3 = 1 + 1 + 1. همچنین 1/3 + 1/3 + 1/3 برابر 3/3 یا 1 است. برنامهٔ کامپیوتری زیر این مسأله را به تصویر می‌کشد. ولی به دلایلی درست کار نمی‌کند. چرا؟

\begin{LTR}
    %@formatter:off
    \begin{lstlisting}[style=C++Style]
         /************************************************
         * test out basic arithmetic that we learned in *
         * first grade. *
         ************************************************/
         #include <iostream>

         int main()
         {
         	float third = 1.0 / 3.0; // The value 1/3
         	float one = 1.0; // The value 1

         	if ((third+third+third) == one)
         	{
         		std::cout <<
         		"Equal 1 = 1/3 + 1/3 + 1/3\n";
         	}
         	else
         	{
         		std::cout <<
         		"NOT EQUAL 1 != 1/3 + 1/3 + 1/3\n";
         	}
         	return (0);
         }
    \end{lstlisting}
    %@formatter:on
\end{LTR}

\begin{tcolorbox}
    دانش‌آموزی اولین برنامهٔ بیسیک خود را نوشته بود و با دستور \lr{\texttt{RUN}} می‌خواست آن را اجرا کند. کامپیوتر یک سری عدد نشان داد و به سرعت صفحه بالا می‌رفت و اعداد بیشتری نمایش داده می‌شد و دانش‌آموزِ بیچاره نمی‌توانست آنها را بخواند.

    دانش‌آموز دقیقه‌ای فکر کرد و از خود پرسید: «آیا اگر می‌نوشتم \lr{\texttt{WALK}}، آرام‌تر اجرا می‌شد؟».
\end{tcolorbox}