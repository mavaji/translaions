\section{}
\paragraph{}\label{answer:93}
متغیرهای عضو به ترتیب اعلان، مقداردهی اولیه می‌شوند. در این صورت عبارات:
\begin{LTR}
    %@formatter:off
    \begin{lstlisting}[style=C++Style]
        ) : width(i_width),
                height(i_height),
                area(width*height)
    \end{lstlisting}
    %@formatter:on
\end{LTR}

به ترتیب اعلان اجرا می‌شوند. \lr{\texttt{1) area، 2) width، 3) height}}. این بدان معناست که \lr{\texttt{area}} با مقادیر نامعلوم \lr{\texttt{width}} و \lr{\texttt{height}} مقداردهی می‌شود و سپس \lr{\texttt{width}} و \lr{\texttt{height}} مقداردهی اولیه می‌شوند.

توابع سازنده‌ای بنویسید که متغیرها به ترتیبی که اعلان می‌شوند، مقداردهی اولیه شوند. (اگر این کار را نکنید، کامپایلر این کار را برای شما می‌کند و باعث درهم‌ریختگی می‌شود). هیچ‌گاه از یک متغیر عضو برای مقداردهی اولیه دیگر متغیرهای عضو استفاده نکنید.