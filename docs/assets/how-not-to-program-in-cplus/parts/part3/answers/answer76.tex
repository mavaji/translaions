\section{}
\paragraph{}\label{answer:76}
مشکل این است که \lr{\texttt{strcmp}} مقدار صفر را در صورت برابر بودن رشته‌ها برمی‌گرداند و در غیر اینصورت مقدار غیرصفر برمی‌گرداند. این بدان معناست که اگر عبارت \lr{\texttt{if(strcmp(x,y))}} را داشته باشید، عبارت \lr{\texttt{if}} فقط وقتی اجرا می‌شود که دو رشته با هم برابر نباشند. برای بررسی این که دو رشته با هم برابر هستند از \lr{\texttt{if(strcmp(x,y) != 0)}} استفاده کنید. این از \lr{\texttt{if(strcmp(x,y))}} واضح‌تر است و کار می‌کند.

تا آن جا که ممکن است از کلاس \lr{\texttt{string}}ِ \lr{\texttt{C++}} به جای رشته‌های قدیمی \lr{\texttt{C}} استفاده کنید. در این صورت می‌توانید به جای \lr{\texttt{strcmp}} از عملگرهای رابطه‌ای \lr{\texttt{(<, >, ==, …)}} استفاده کنید.