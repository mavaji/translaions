\section[دوباره سلام]{دوباره سلام \protect\LTRfootnote{\lr{Hello Again}} (راهنمایی \ref{hint:172}، جواب \ref{answer:69})}
\paragraph{}\label{prog:17}
باز هم این کار را انجام می‌دهیم. \lr{"Hello World"} را خراب کردیم. مشکل چیست؟

\begin{LTR}
    %@formatter:off
        \begin{lstlisting}[style=C++Style]
             #include <iostream>

             int main()
             {
             	std::cout << "Hello World!/n";
             	return (0);
             }
        \end{lstlisting}
    %@formatter:on
\end{LTR}

\begin{tcolorbox}
    برنامه‌نویسان واقعی به زبان کوبول برنامه نمی‌نویسند. کوبول برای برنامه‌نویسانِ بی‌مایه مناسب است.

    برنامه‌های برنامه‌نویسان واقعی هیچ وقت در اولین بار کار نمی‌کند. ولی اگر آنها را روی ماشین قرار دهید، می‌توانند کار کنند البته بعد از «تعداد بسیار کمی» جلسه 30 ساعته برای دیباگ کردن.

    برنامه‌نویسان واقعی هیچ وقت از 9 تا 5 کار نمی‌کنند. اگر برنامه‌نویس واقعی را حول و حوش 9 صبح دیدید، به خاطر این است که تمامِ شب بیدار بوده است.

    برنامه‌نویسان واقعی هیچ وقت مستندسازی نمی‌کنند. مستندسازی برای ابلهانی است که نمی‌توانند کُدِ برنامه را بخوانند.

    برنامه‌نویسان واقعی به پاسکال، \lr{BLISS} یا \lr{Ada} یا هر کدام از آن زبان‌های ریشه‌ایِ علوم کامپیوتر برنامه نمی‌نویسند. انواع دادهٔ قوی فقط به درد افراد کم‌حافظه می‌خورد.
\end{tcolorbox}