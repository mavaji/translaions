\section{}
\paragraph{}\label{answer:71}
\begin{enumerate}
    \item تنظیم

    در برخی ماشین‌ها باید اعداد صحیح به صورت 2 بایتی یا 4 بایتی باشند. در برخی ماشین‌ها لازم نیست. \lr{\texttt{C++}} برای تنظیم کردن این قاعده، از لایی‌گذاری استفاده می‌کند. لذا روی یک ماشین، ساختار به صورت زیر خواهد بود:
    \begin{LTR}
    %@formatter:off
        \begin{lstlisting}[style=C++Style]
        struct data {
            char flag; // 1 byte
            long int value; // 4 bytes
        };
        \end{lstlisting}
    %@formatter:on
    \end{LTR}

    که کلاً 5 بایت می‌شود. روی یک ماشین دیگر، ممکن است این گونه باشد:
    \begin{LTR}
    %@formatter:off
        \begin{lstlisting}[style=C++Style]
            struct data {
                char flag; // 1 byte
                char pad[3]; // 3 bytes (automatic padding)
                long int value; // 4 bytes
            };
        \end{lstlisting}
    %@formatter:on
    \end{LTR}

    که کلاً 8 بایت می‌شود.

    \item ترتیب بایت‌ها

    برخی ماشین‌ها، اعداد صحیح بزرگ را با ترتیب بایتی \lr{\texttt{ABCD}} می‌نویسند. برخی دیگر از \lr{\texttt{DCBA}} استفاده می‌کنند. این امر، مانعِ قابلیت‌حمل می‌شود.

    \item اندازهٔ عدد صحیح

    ماشین‌های 64 بیتی هم وجود دارند. یعنی روی برخی سیستم‌ها یک \lr{\texttt{long int}} برابر 64 بیت است نه 32 بیت.
\end{enumerate}