\section[غلطِ انباشته شده]{غلطِ انباشته شده \protect\LTRfootnote{\lr{Stacked Wrong}} (راهنمایی \ref{hint:296}، جواب \ref{answer:72})}
\paragraph{}\label{prog:91}
در برنامهٔ زیر، یک کلاس خطرناک  \lr{\texttt{stack}} و یک کلاس ایمن‌ترِ \lr{\texttt{safe\_stack}} تعریف می‌کنیم. برنامهٔ آزمونِ ما یک آرایه از پنج پشته درست می‌کند و تعدادی دادهٔ آزمون در آن قرار می‌دهد. برنامه، اندازهٔ پشته را می‌نویسد. ولی نتایج، آن چه که انتظار می‌رفت نیستند.

\begin{LTR}
    %@formatter:off
        \begin{lstlisting}[style=C++Style]
             /************************************************
             * stack_test -- Test the use of the classes *
             * stack and safe_stack. *
             ************************************************/
             #include <iostream>

             // The largest stack we can use
             // (private to class stack and safe_stack)
             const int STACK_MAX = 100;
             /************************************************
             * stack -- Class to provide a classic stack. *
             * *
             * Member functions: *
             * push -- Push data on to the stack. *
             * pop -- Return the top item from the *
             * stack. *
             * *
             * Warning: There are no checks to make sure *
             * that stack limits are not exceeded. *
             ************************************************/
             class stack {
             	protected:
             		int count; // Number of items in the stack
             		int *data; // The stack data
             	public:
             		// Initialize the stack
             		stack(void): count(0)
             		{
             			data = new int[STACK_MAX];
             		}
             		// Destructor
             		virtual ~stack(void) {
             			delete data;
             			data = NULL;
             		}
             	private:
             		// No copy constructor
             		stack(const stack &);

             		// No assignment operator
             		stack & operator = (const stack &);
             	public:
             		// Push an item on the stack
             		void push(
             			const int item // Item to push
             		) {
             			data[count] = item;
             			++count;
             		}
             		// Remove the an item from the stack
             		int pop(void) {
             			--count;
             			return (data[count]);
             		}

             		// Function to count things in
             		// an array of stacks
             		friend void stack_counter(
             			stack stack_array[],
             			const int n_stacks
             		);
             };

             /***********************************************
             * safe_stack -- Like stack, but checks for *
             * errors. *
             * *
             * Member functions: push and pop *
             * (just like stack) *
             ***********************************************/
             class safe_stack : public stack {
             	public:
             		const int max; // Limit of the stack
             	public:
             		safe_stack(void): max(STACK_MAX) {};
             		// Destructor defaults
             	private:
             		// No copy constructor
             		safe_stack(const safe_stack &);

             		// No assignment operator
             		safe_stack & operator =
             			(const safe_stack &);
             	public:
             		// Push an item on the stack
             		void push(
             			// Data to push on the stack
             			const int data
             		) {
             			if (count >= (STACK_MAX-1)) {
             				std::cout << "Stack push error\n";
             				exit (8);
             			}
             			stack::push(data);
             		}
             		// Pop an item off the stack
             		int pop(void) {
             			if (count <= o) {
             				std::cout << "Stack pop error\n";
             				exit (8);
             			}
             			return (stack::pop());
             		}
             };


             /************************************************
             * stack_counter -- Display the count of the *
             * number of items in an array of stacks. *
             ************************************************/
             void stack_counter(
             	// Array of stacks to check
             	stack *stack_array,

             	// Number of stacks to check
             	const int n_stacks
             )
             {
             	int i;

             	for (i = 0; i < n_stacks; ++i)
             	{
             		std::cout << "Stack " << i << " has " <<
             		stack_array[i].count << " elements\n";
             	}
             }

             // A set of very safe stacks for testing
             static safe_stack stack_array[5];

             int main()
             {

             	stack_array[0].push(0);

             	stack_array[1].push(0);
             	stack_array[1].push(1);

             	stack_array[2].push(0);
             	stack_array[2].push(1);
             	stack_array[2].push(2);

             	stack_array[3].push(0);
             	stack_array[3].push(1);
             	stack_array[3].push(2);
             	stack_array[3].push(3);

             	stack_array[4].push(0);
             	stack_array[4].push(1);
             	stack_array[4].push(2);
             	stack_array[4].push(3);
             	stack_array[4].push(4);

             	stack_counter(stack_array, 5);
             	return (0);
             }
        \end{lstlisting}
    %@formatter:on
\end{LTR}

\begin{tcolorbox}
    مشکلی وجود ندارد که با کودنی و جهل کافی نتوان آن را حل کرد.
\end{tcolorbox}