\section{}
\paragraph{}\label{answer:103}
مسأله این جاست که در برخی سیستم‌ها، \lr{\texttt{long}}ها باید در یک فضای چهاربایتی قرار داشته باشند. لذا بگذارید نگاهی به ساختارِ خود بیندازیم:
\begin{LTR}
    %@formatter:off
    \begin{lstlisting}[style=C++Style]
        struct end_block_struct
        {
            unsigned long int next_512_pos; // [0123]
            unsigned char next_8k_pos1; // [4]
            unsigned char next_8k_pos2; // [5]
            unsigned long int prev_251_pos; // [6789]
    \end{lstlisting}
    %@formatter:on
\end{LTR}

6 بر 4 بخش‌پذیر نیست، لذا کامپایلر دو بایت دیگر اضافه می‌کند تا به 8 برسد. بنابراین چیزی که در واقع داریم این است:
\begin{LTR}
%@formatter:off
\begin{lstlisting}[style=C++Style]
struct end_block_struct
{
    unsigned long int next_512_pos; // [0123]
    unsigned char next_8k_pos1; // [4]
    unsigned char next_8k_pos2; // [5]
    unsigned char pad1, pad2; // [67]
    unsigned long int prev_251_pos; // [89 10 11]
\end{lstlisting}
%@formatter:on
\end{LTR}

این چیزی نیست که ما می‌خواستیم. عباراتی مانند \lr{\texttt{assert(sizeof(end\_block\_struct) == 16);}} را در برنامهٔ خود قرار دهید تا مراقب کامپایلرهایی که این مشکل را به وجود می‌آورند باشید.