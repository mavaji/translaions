\section[خطای صفر]{خطای صفر \protect\LTRfootnote{\lr{Zero Error}} (راهنمایی \ref{hint:50}، جواب \ref{answer:20})}
\paragraph{}\label{prog:28}
برنامهٔ زیر طراحی شده است تا یک آرایه را صفر کند. پس چرا کار نمی‌کند؟ آیا \lr{\texttt{memset}} خراب شده است؟

\begin{LTR}
    %@formatter:off
    \begin{lstlisting}[style=C++Style]
         /************************************************
         * zero_array -- Demonstrate how to use memset *
         * to zero an array. *
         ************************************************/
         #include <iostream>
         #include <cstring>

         int main()
         {
         	// An array to zero
         	int array[5] = {1, 3, 5, 7, 9};

         	// Index into the array
         	int i;

         	// Zero the array
         	memset(array, sizeof(array), '\0');

         	// Print the array
         	for (i = 0; i < 5; ++i)
         	{
         		std::cout << "array[" << i << "]= " <<
         		array[i] << std::endl;
         	}
         	return (0);
         }
    \end{lstlisting}
    %@formatter:on
\end{LTR}

\begin{tcolorbox}
    از یک راهنمای فرترن برای کامپیوترهای \lr{Xerox}:

    هدف اصلی عبارت \lr{\texttt{DATA}} این است که به ثوابت، اسم اختصاص دهد. به جای این که هر بار بجای \lr{π} بنویسیم \lr{\texttt{3.141592653589793}}، می‌توانیم با عبارت \lr{\texttt{DATA}}، این مقدار را به متغیر \lr{\texttt{PI}} اختصاص دهیم و به جای آن عدد طولانی از آن استفاده کنیم. این کار همچنین تغییر دادن برنامه را آسان‌تر می‌کند چون ممکن است مقدار \lr{π} عوض شود.
\end{tcolorbox}