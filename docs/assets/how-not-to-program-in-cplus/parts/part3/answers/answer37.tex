\section{}
\paragraph{}\label{answer:37}
مسأله این است که کاراکتر \lr{\texttt{\textbackslash}} به عنوان یک کاراکتر گریز\LTRfootnote{\lr{Escape Character}} استفاده می‌شود. لذا \lr{\texttt{\textbackslash n}} یک خط جدید است. \lr{\texttt{\textbackslash new}} نیز برابر \lr{\texttt{<newline>ew}} است. بنابراین رشته \lr{\texttt{\textbackslash root\textbackslash new\textbackslash table}} در واقع بدین صورت است:
\LTR\noindent
\lr{\texttt{"<return>oot<newline>ew<tab>able"}}
\RTL
آن چه که برنامه‌نویس واقعاً می‌خواست این بود:
ٰ\begin{LTR}
     %@formatter:off
     \begin{lstlisting}[style=C++Style]
         const char name[] = "\\root\\new\\table"; // DOS path
     \end{lstlisting}
     %@formatter:on
\end{LTR}

این قاعده به \lr{\texttt{\#include}} اسم فایل‌ها اعمال نمی‌شود لذا \lr{\texttt{\#include "\textbackslash usr\textbackslash include\textbackslash table.h"}} درست کار می‌کند.