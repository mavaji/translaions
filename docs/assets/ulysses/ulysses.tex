% Default Compiler: txs:///xelatex
% Default Bibliography Tool: BibTex

\documentclass[12pt]{book}
\usepackage[x11names]{xcolor}
\usepackage{titlesec}
\usepackage[linktocpage=true,colorlinks,citecolor=blue,pagebackref=true]{hyperref}
\usepackage[top=30mm, bottom=30mm, left=30mm, right=30mm]{geometry}
\usepackage[utf8]{inputenc}

\definecolor{Mulberry}{HTML}{72243D}
\colorlet{accent}{black!80!white}
\newcommand{\noun}[1]{{\textbf{#1}}}
%\newcommand{\noun}[1]{«{#1}»}

\usepackage{setspace}
%\onehalfspacing
\doublespacing

\usepackage{xepersian}
\usepackage[T1]{fontenc}
\settextfont[Scale=1.2]{IRNazanin}
\defpersianfont\mfo[Scale=1.2]{IRNazanin}
\setlatintextfont[Scale=1]{Doulos SIL}

%\titleformat{\chapter}
%{\normalfont\huge\bfseries}{\chaptertitlename\ \thechapter.}{20pt}{\huge}
%\renewcommand \chaptertitlename {اپیزود}
\renewcommand{\chaptername}{اپیزود}

\titleformat
{\chapter} % command
[display] % shape
{\bfseries\Large\itshape} % format
{اپیزود \ \thechapter} % label
{0.5ex} % sep
{
\rule{\textwidth}{1pt}
\vspace{1ex}
\centering
} % before-code
[
\vspace{-0.5ex}%
\rule{\textwidth}{0.3pt}
] % after-code

\begin{document}
    \title{خلاصهٔ اولیس \LTRfootnote{\lr{\url{https://www.sparknotes.com/lit/ulysses/}}}}
    \author{جیمز جویس\\
    ترجمهٔ وحید مواجی
    }
    \date{مهر 1391}
    \frontmatter                            % only in book class (roman page #s)
    \maketitle                              % Print title page.
    \tableofcontents                        % Print table of contents
    \mainmatter


    \part{شخصیت‌ها}
    \paragraph{\noun{لئوپلد بلوم}\protect\LTRfootnote{\lr{Leopold Bloom}}} -
    مردی سی و هشت ساله و مسئول تبلیغات در دوبلین. \noun{بلوم} در دوبلین با \noun{رودلف}\LTRfootnote{\lr{Rudolph}}، پدر یهودیِ مجارستانی‌اش‌ و \noun{الن}\LTRfootnote{\lr{Ellen}}، مادر کاتولیکِ ایرلندی‌اش بزرگ شده است. او از مطالعه و تفکر دربارهٔ علوم و اختراعات و شرح معلوماتش به دیگران لذت می‌برد. \noun{بلوم}، عاطفی و کنجکاو است و عاشق موسیقی می‌باشد. او ذهنش درگیر روابط سردش با زنش \noun{مالی} است.
    \paragraph{\noun{مریان (مالی) بلوم}\protect\LTRfootnote{\lr{Marion (Molly) Bloom}}} -
    همسر \noun{لئوپلد بلوم}. \noun{‫مالی بلوم‬} سی‌ساله است، کمی تپل و سبزه، خوش بر و رو و اهل لاس‌زدن می‌باشد. او تحصیلات زیادی ندارد ولی به هر تقدیر باهوش و صاحب‌نظر است. او خواننده‌ای حرفه‌ای است که توسط پدر ایرلندی‌اش، سرگرد \noun{برایان توییدی}\LTRfootnote{\lr{Brian Tweedy}} در \noun{جبل‌الطارق}\LTRfootnote{\lr{Gibraltar}} بزرگ شده است. \noun{مالی} حوصلهٔ \noun{بلوم} را ندارد مخصوصاً به این دلیل که از مرگ یازده سال پیشِ پسرشان \noun{رودی}\LTRfootnote{\lr{Rudy}} به این طرف، \noun{بلوم} دیگر با او صمیمی (در رابطهٔ جنسی) نیست.
    \paragraph{\noun{استیوِن ددالوس}\protect\LTRfootnote{\lr{Stephen Dedalus}}} -
    شاعری پرالهام و بیست و چندساله. \noun{استیون} باهوش و فوق‌العاده کتابخوان و علاقه‌مند به موسیقی است. به نظر می‌رسد که بیشتر در دنیای ذهنی خودش زندگی می‌کند تا اینکه عضو انجمنی یا حتی گروه دانشجویان پزشکی که همقطارانش هستند باشد. \noun{استیون} در کودکی، بسیار مذهبی بوده است ولی بر اثر مرگ مادرش که کمتر از یک سال پیش رخ داده، اکنون با مسائل مربوط به شک و ایمان دست و پنجه نرم می‌کند.
    \paragraph{\noun{مالاکی (باک)  مالیگان}\protect\LTRfootnote{\lr{Malachi (Buck) Mulligan}}} -
    دانشجوی پزشکی و دوست \noun{استیون}. \noun{باک مالیگان‬} کمی چاق و اهل مطالعه است و تقریباً همه چیز را دست می‌اندازد. او به خاطر لطیفه‌های بی‌ادبی و بامزه‌ای که تعریف می‌کند تقریباً مورد علاقه همه به جز \noun{استیون}، \noun{سایمون}  و \noun{بلوم} است.
    \paragraph{\noun{هینز}\protect\LTRfootnote{\lr{Haines}}} -
    دانشجوی فرهنگ فولکلور که به خصوص علاقه‌مند به مطالعهٔ قوم و فرهنگ ایرلندی است. \noun{هینز}  اغلب اوقات ناخواسته مغرور و خودبین است. او در قلعه \noun{مارتلو}\LTRfootnote{\lr{Martello}} اقامت دارد جایی که \noun{استیون} و \noun{باک}  هم در آنجا زندگی می‌کنند.
    \paragraph{\noun{هیو (بلیزس) بویلان}\protect\LTRfootnote{\lr{Hugh (``Blazes'') Boylan}}} -
    مدیر کنسرت قریب‌الوقع \noun{مالی} در بلفاست. \noun{بلیزس بویلان‬} در شهر مشهور و محبوب است علی‌رغم اینکه کمی هرزه به نظر می‌رسد، مخصوصاً نسبت به زنان. \noun{بویلان} به \noun{مالی} علاقه‌مند شده است و آنها در بعدازظهرِ داستان رابطه‌ای با هم برقرار می‌کنند.
    \paragraph{\noun{میلیسنت (میلی) بلوم}\protect\LTRfootnote{\lr{Millicent (Milly) Bloom}}} -
    دختر پانزده‌سالهٔ \noun{مالی} و \noun{لئوپلد بلوم} که فی‌الواقع در اولیس ظاهر نمی‌شود. خانوادهٔ \noun{بلوم} اخیراً \noun{میلی} را برای زندگی و یادگیری عکاسی به \noun{مالینگار}\LTRfootnote{\lr{Mullingar}} فرستاده‌اند. \noun{میلی}، بلوند و زیبا و علاقه‌مند به پسرها است - او با \noun{الک بانون}\LTRfootnote{\lr{Alec Bannon}} در \noun{مالینگار} قرار و مدار می‌گذارد.
    \paragraph{\noun{سایمون ددالوس}\protect\LTRfootnote{\lr{Simon Dedalus}}} -
    پدرِ \noun{استیون ددالوس}. \noun{سایمون  ددالوس} در \noun{کورک}\LTRfootnote{\lr{Cork}} بزرگ شده و بعداً به دوبلین آمده و تا کنون مرد نسبتاً موفقی بوده است. مردان دیگر، او را سرلوحه خود قرار می‌دهند، هرچند که بعد از مرگ زنش، خانه و زندگی‌اش بی‌نظم و نامرتب شده است. \noun{سایمون}  دارای صدایی خوب و استعداد لطیفه‌گویی است و اگر عادت مشروب‌خوری نداشت می‌توانست از این همه استعداد سود ببرد. \noun{سایمون}  به شدت منتقد \noun{استیون} است.
    \paragraph{\noun{اِی.ای (جرج راسل)}\protect\LTRfootnote{\lr{A.E. (George Russell)}}} -
    \noun{اِی.ای} نام مستعار \noun{جرج راسل‬}، شاعر معروفِ احیای ادبیات ایرلندی است که در کانونِ حلقه‌های ادبی ایرلند می‌باشد - حلقه‌های ادبی که \noun{استیون} را به خود راه نمی‌دهند. او عمیقاً به عرفان اسرارآمیز علاقه‌مند است. بقیه مردها چنان با او مشورت می‌کنند که انگار حرفش وحی منزل است.
    \paragraph{\noun{ریچارد بِست}\protect\LTRfootnote{\lr{Richard Best}}} -
    کتابداری در کتابخانهٔ ملی. \noun{بست} شخص مشتاق و علاقه‌مندی است، با این حال بخش عمده‌ای از مشارکتش در بحثِ هملت در اپیزود~\ref{ep:9}، نشانه‌هایی از باورهای غلطی دارد که به خیال خودش درست می‌باشند.
    \paragraph{\noun{ادی بوردمن}\protect\LTRfootnote{\lr{Edy Boardman}}} -
    یکی از دوستان \noun{گرتی مک‌داول‬}\LTRfootnote{\lr{Gerty MacDowell}}. رفتار مغرورانهٔ \noun{گرتی}، \noun{ادی} را که می‌خواهد او را با گوشه و کنایه ضایع کند، می‌رنجاند.
    \paragraph{\noun{جوسی (نام خانهٔ پدری: پاول) و دنیس برین}\protect\LTRfootnote{\lr{Josie (née Powell) and Denis Breen}}} -
    \noun{جوسی پاول} و \noun{بلوم} وقتی جوان‌تر بودند به هم علاقه داشتند. \noun{جوسی} زیبا و اهل لاس‌زدن بود. بعد از اینکه \noun{بلوم} با \noun{مالی} ازدواج کرد، \noun{جوسی} هم با \noun{دنیس} ازدواج کرد. \noun{دنیس برین} کمی دیوانه است و پارانوید به نظر می‌رسد. مراقبت از چنین شوهر ابلهی اثر خود را روی \noun{جوسی} گذاشته است و اکنون نحیف و خسته به نظر می‌رسد.
    \paragraph{\noun{سیسی، جکی و تامی کافری}\protect\LTRfootnote{\lr{Cissy, Jacky, and Tommy Caffrey}}} -
    \noun{سیسی کافری} یکی از بهترین دوستان \noun{گرتی مک‌داول‬} است. او دختری با رفتار پسرانه و کمی رُک است. او مراقب برادران نوپای کوچکش، \noun{جکی} و \noun{تامی} است.
    \paragraph{\noun{شهروند}\protect\LTRfootnote{\lr{The citizen}}} -
    یک میهن‌پرست ایرلندی مسن که از نهضت ناسیونالیست دفاع می‌کند. با اینکه به نظر نمی‌رسد \noun{شهروند‬} هیچ ارتباط رسمی با نهضت داشته باشد ولی بقیه افراد، اخبار و اطلاعات را از او می‌پرسند. او سابقاً یکی از ورزشکاران و پهلوانان ایرلند بوده است. او ماجراجو و بیگانه‌هراس است.
    \paragraph{\noun{مارتا کلیفورد}\protect\LTRfootnote{\lr{Martha Clifford}}} -
    زنی که \noun{بلوم} با او تحت نام مستعار \noun{هنری فلاور}\LTRfootnote{\lr{Henry Flower}} مکاتبه می‌کند. نامه‌های \noun{مارتا} پر از غلطهای نگارشی است و تمایلات جنسی‌اش، غیرخلاقانه و ملال‌آورند.
    \paragraph{\noun{بلا کوهن}\protect\LTRfootnote{\lr{Bella Cohen}}} -
    زن فاحشه‌ای خلافکار. \noun{بلا کوهن} گنده، سبزه و دارای رفتاری مردانه است. او تا حدی طالب احترام از جانب بقیه است و پسری در آکسفورد دارد که شهریه‌اش را یکی از مشتریانش می‌پردازد.
    \paragraph{\noun{مارتین کانینگهام}\protect\LTRfootnote{\lr{Martin Cunningham}}} -
    یکی از اعضای اصلی حلقهٔ دوستان \noun{بلوم}. \noun{مارتین کانینگهام}  نسبت به دیگران مهربان و باشفقت است و در لحظات مختلفی از روز (کل داستان در یک روز اتفاق می‌افتد) از \noun{بلوم} دفاع می‌کند با این حال با \noun{بلوم} مثل یک بیگانه رفتار می‌کند. قیافه او، شکسپیر را تداعی می‌کند.
    \paragraph{\noun{گرت دیزی}\protect\LTRfootnote{\lr{Garrett Deasy}}} -
    مدیر مدرسهٔ پسرانه‌ای که \noun{استیون} در آنجا تدریس می‌کند. \noun{دیزی}، پروتستانی از شمال ایرلند و به دولت انگلیس پایبند است. \noun{دیزی} نسبت به \noun{استیون} با تکبر برخورد می‌کند و شنوندهٔ خوبی نیست. نامه پر و پیمان او به ویراستار دربارهٔ تب برفکیِ احشام، موضوع استهزاء مردان دوبلینی در طی روز است.
    \paragraph{\noun{دیلی، کیتی، بودی و مگی ددالوس}\protect\LTRfootnote{\lr{Dilly, Katey, Boody, and Maggy Dedalus}}} -
    خواهران جوان‌تر \noun{استیون}. آنها بعد از مرگ مادرشان سعی در رتق و فتق امور منزلِ \noun{ددالوس} دارند. به نظر می‌رسد که \noun{دیلی} علائق و آرزوهایی مثل یادگیری زبان فرانسه دارد.
    \paragraph{\noun{پاتریک دیگنام، خانم دیگنام و پاتریک دیگنام جونیور}\protect\LTRfootnote{\lr{Patrick Dignam, Mrs. Dignam, and Patrick Dignam, Jr.}}} -
    \noun{پاتریک دیگنام} یکی از آشنایان \noun{بلوم} بود که خیلی زود بر اثر شرابخواری درگذشت. مراسم خاکسپاری او امروز است و \noun{بلوم} و بقیه جمع می‌شوند تا برای بیوهٔ \noun{دیگنام} و بچه‌هایش مقادیری پول جمع کنند چرا که \noun{پدی}\LTRfootnote{\lr{Paddy}} همه بیمهٔ عمرش را صرف پرداخت دیونش کرده بود و برای بچه‌هایش چیزی باقی نگذاشته است.
    \paragraph{\noun{بن دالرد}\protect\LTRfootnote{\lr{Ben Dollard}}} -
    مردی که در دوبلین به خاطر صدای بمِ عالی‌اش شهره است. کسب و کار \noun{بن دالرد}  مدتی پیش از رونق افتاده است. آدم خوش‌طینتی به نظر می‌رسد ولی احتمالاً به خاطر عادت شرابخواری گذشته‌اش، عصبی و پریشان است.
    \paragraph{\noun{جان اگلینتون}\protect\LTRfootnote{\lr{John Eglinton}}} -
    مقاله‌نویسی که وقتش را در کتابخانهٔ ملی می‌گذراند. \noun{جان اگلینتون} ، اعتماد به نفس و غرور جوانیِ \noun{استیون} را تحقیر می‌کند و نسبت به تئوری هملت \noun{استیون} با دیده تردید می‌نگرد.
    \paragraph{\noun{ریچی، سارا (سالی) و والتر گولدینگ}\protect\LTRfootnote{\lr{Richie, Sara (Sally), and Walter Goulding}}} -
    \noun{ریچی گولدینگ}، دایی \noun{استیون ددالوس} است؛ او برادرِ \noun{ماری}، مادرِ \noun{استیون} بوده است. \noun{ریچی} کارمند دادگستری است که اخیراً به خاطر مشکل کمرش کمتر توانسته کار کند - مسأله‌ای که به خاطر آن، موضوعِ خندهٔ \noun{سایمون  ددالوس} شده است. \noun{والتر}، پسر \noun{ریچی} و \noun{سارا}، لوچ است و لکنت زبان دارد.
    \paragraph{\noun{زو هیگینز}\protect\LTRfootnote{\lr{Zoe Higgins}}} -
    فاحشه‌ای در فاحشه‌خانهٔ \noun{بلا کوهن}. \noun{زو} بی‌پروا و در زخم زبان زدن استاد است.
    \paragraph{\noun{جو هاینز}\protect\LTRfootnote{\lr{Joe Hynes}}} -
    گزارشگری از روزنامهٔ دوبلین که اغلب اوقات بی‌پول است - او از \noun{بلوم}، سه پوند قرض گرفته است و تا کنون آن را پس نداده. \noun{هاینز}، \noun{بلوم} را درست نمی‌شناسد و در اپیزود \ref{ep:12} به نظر می‌رسد که دوست خوبی برای \noun{شهروند‬} است.
    \paragraph{\noun{کورنی کله‌هر}\protect\LTRfootnote{\lr{Corny Kelleher}}} -
    مسئول کفن و دفن که روابط خوبی با پلیس دارد.
    \paragraph{\noun{مینا کندی و لیدیا دوس}\protect\LTRfootnote{\lr{Mina Kennedy and Lydia Douce}}} -
    دختران پیشخدمت هتل \noun{اورموند}\LTRfootnote{\lr{Ormond}}. \noun{مینا} و \noun{لیدیا} اهل لاس‌زنی هستند و با مردانی که به نوشگاه می‌آیند گرم می‌گیرند، با این حال در خلوت خود از جنس مخالف به بدی یاد می‌کنند. دوشیزه \noun{دوس} که موهای برنز رنگی دارد، بی‌پرواتر از آن یکی به نظر می‌رسد و با \noun{بلیزس بویلان‬} درگیری داشته است. دوشیزه \noun{کندی} که موهایی طلایی دارد، خوددارتر است.
    \paragraph{\noun{ند لمبرت}\protect\LTRfootnote{\lr{Ned Lambert}}} -
    یکی از دوستان \noun{سایمون ددالوس‬} و بقیه مردان در دوبلین. \noun{ند لمبرت} اغلب در حال لطیفه‌گویی و خنده است. او در انبار غله و حبوبات در مرکز شهر کار می‌کند، در جایی که زمانی صومعهٔ مریم مقدس بوده است.
    \paragraph{\noun{لنه‌هان}\protect\LTRfootnote{\lr{Lenehan}}} -
    ویراستار مسابقات در روزنامهٔ دوبلین؛ با این حال اسب مورد نظر او، \noun{سپتر}\LTRfootnote{\lr{Sceptre}} در مسابقات \noun{گلدکاپ} می‌بازد. \noun{لنه‌هان} آدم بذله‌گویی است و با زنان لاس می‌زند. او \noun{بلوم} را مسخره می‌کند ولی به \noun{سایمون}  و \noun{استیون ددالوس} احترام می‌گذارد.
    \paragraph{\noun{لینچ}\protect\LTRfootnote{\lr{Lynch}}} -
    دانشجوی پزشکی و دوست قدیمی \noun{استیون} (او در «چهرهٔ هنرمند در جوانی» هم حضور دارد). \noun{لینچ} به شنیدن نظریات پرمدعا و فوقِ زیباشناسانهٔ \noun{استیون} عادت دارد و با سرسختی و لجاجتِ \noun{استیون} آشناست. او با \noun{کیتی ریکتس}\LTRfootnote{\lr{Kitty Ricketts}} قرار می‎‌گذارد.
    \paragraph{\noun{تامس دابلیو لیستر}\protect\LTRfootnote{\lr{Thomas W. Lyster}}} -
    کتابداری در کتابخانهٔ ملی دوبلین و عضو فرقهٔ \noun{کویکر}\LTRfootnote{\lr{Quaker}}. \noun{لیستر} بیشترین علاقه را به صحبت‌های \noun{استیون} در اپیزود \ref{ep:9} نشان می‌دهد.
    \paragraph{\noun{گرتی مک‌داول}\protect\LTRfootnote{\lr{Gerty MacDowell}}} -
    زنی در اوان بیست سالگی و از خانواده‌ای از طبقهٔ متوسطِ رو به پایین. \noun{گرتی} از لنگی دائمی پایش رنج می‌برد که احتمالاً بر اثر تصادف با دوچرخه بوده است. او با دقت بسیاری به لباس پوشیدن و رژیمش اهمیت می‌دهد و آرزوی عاشق شدن و ازدواج دارد. او به ندرت به خودش اجازه می‌دهد راجع به معلولیتش فکر کند.
    \paragraph{\noun{جان هنری منتون}\protect\LTRfootnote{\lr{John Henry Menton}}} -
    مشاور حقوقی در دوبلین که توسط \noun{پدی دیگنام} استخدام شده است. وقتی \noun{بلوم} و \noun{مالی} عاشق هم بودند، \noun{منتون} تحت تأثیر علاقه به \noun{مالی}، رقیبی عشقی برای \noun{بلوم} بود. او نسبت به \noun{بلوم} با بی‌احترامی رفتار می‌کند.
    \paragraph{\noun{راوی بی‌نام اپیزود \ref{ep:12}}} -
    راویِ بی‌نام اپیزود \ref{ep:12}، در حال حاضر کارگزار وصول طلب است و این جدیدترین شغلش از بین شغل‌های بسیاری است که داشته. او از اینکه «بااطلاع» به نظر برسد لذت می‌برد و بخش عمدهٔ شایعاتی که دربارهٔ خانواده \noun{بلوم} می‌داند را از دوستش \noun{پیسر بورک}\LTRfootnote{\lr{``Pisser'' Burke}} شنیده که آنها را وقتی در هتل \noun{سیتی آرمز}\LTRfootnote{\lr{City Arms}} زندگی می‌کردند می‌شناخته است.
    \paragraph{\noun{عضو شورای شهر، نانتی}\protect\LTRfootnote{\lr{City Councillor Nannetti}}} -
    مسئول ارشد چاپ در روزنامه دوبلین و عضو پارلمان. \noun{نانتی} یک دورگهٔ ایتالیایی-ایرلندی است.
    \paragraph{\noun{جی.جی اُمالوی}\protect\LTRfootnote{\lr{J. J. O'Molloy}}} -
    وکیلی که اکنون بیکار و بی‌پول است. \noun{اُمالوی}، امروز در قرض گرفتن پول از دوستانش ناکام است. او در اپیزود \ref{ep:12} در میخانهٔ \noun{بارنی کیرنان}\LTRfootnote{\lr{Barney Kiernan}}، از \noun{بلوم} دفاع می‌کند.
    \paragraph{\noun{جک پاور}\protect\LTRfootnote{\lr{Jack Power}}} -
    یکی از دوستان \noun{سایمون ددالوس‬} و \noun{مارتین کانینگهام}  و دیگر مردان شهر. \noun{پاور} احتمالاً در اجرای احکام کار می‌کند. او زیاد با \noun{بلوم} خوب نیست.
    \paragraph{\noun{کیتی ریکتس}\protect\LTRfootnote{\lr{Kitty Ricketts}}} -
    یکی از فاحشه‌هایی که در فاحشه‌خانهٔ \noun{بلا کوهن} کار می‌کنند. به نظر می‌رسد که \noun{کیتی} با \noun{لینچ} رابطه دارد و بخشی از روز را با او گذرانده است. او لاغر است و طرز لباس پوشیدنش، تمایلاتش به طبقهٔ بالای جامعه را نشان می‌دهد.
    \paragraph{\noun{فلوری تالبوت}\protect\LTRfootnote{\lr{Florry Talbot}}} -
    یکی از فاحشه‌های فاحشه‌خانهٔ \noun{بلا کوهن}. \noun{فلوری} چاق است و کودن به نظر می‌رسد ولی به راحتی خوشحال می‌شود.

    \part{طرح کلی داستان}
    \noun{استیون ددالوس} در حال گذراندن ساعات اولیهٔ صبح 16 ژوئن 1904 است و از دوست مسخره‌کننده‌اش، \noun{باک مالیگان} و \noun{هینز}، آشنای انگلیسی \noun{باک} دوری می‌جوید. وقتی \noun{استیون} می‌خواهد سر کار برود، \noun{باک} به او می‌گوید که کلید خانه را با خود نبرد و آنها را در ساعت 12:30 در میخانه ببیند. \noun{استیون} از \noun{باک} می‌رنجد.

    حدود ساعت 10 صبح، \noun{استیون} در کلاس درسش در مدرسهٔ پسرانهٔ \noun{گرت دیزی} دارد تاریخ درس می‌دهد. بعد از کلاس، \noun{استیون} با \noun{دیزی} ملاقات می‌کند تا حقوقش را بگیرد. \noun{دیزیِ} کوته‌فکر و متعصب، راجع به زندگی برای \noun{استیون} موعظه می‌کند. \noun{استیون} قبول می‌کند که نوشتهٔ \noun{دیزی} دربارهٔ بیماری احشام را به آشنایانش در روزنامه بدهد.

    \noun{استیون} بقیهٔ صبحش را به تنها قدم زدن در ساحل \noun{سندی‌مونت} می‌گذراند و منتقدانه به خودِ جوان‌ترش و به ادراک و الهام فکر می‌کند. او در ذهنش شعری می‌گوید و آن را روی تکه کاغذی که از نوشتهٔ \noun{دیزی} پاره کرده است می‌نویسد.

    در ساعت 8 صبح همان روز، \noun{لئوپلد بلوم} مشغول درست کردن صبحانه است و نامه و صبحانهٔ زنش را برای او به رختخواب می‌برد. یکی از نامه‌های زنش از طرف مدیر تور کنسرتِ \noun{مالی}، \noun{بلیزس بویلان} است (\noun{بلوم} به این مظنون است که او عاشق زنش نیز هست) - \noun{بویلان} ساعت 4 بعدازظهر امروز قرار ملاقات دارد. \noun{بلوم} به طبقهٔ پایین می‌رود و نامهٔ دخترش \noun{میلی} را می‌خواند و سپس از خانه خارج می‌شود.

    در ساعت 10 صبح، \noun{بلوم} نامه‌ای عاشقانه از ادارهٔ پست دریافت می‌کند - او با زنی به نام \noun{مارتا کلیفورد} تحت نام مستعار هنری \noun{فلاور} نامه‌نگاری می‌کند. او نامه را می‌خواند، مختصری در کلیسایی می‌ماند و سپس لوسیونِ \noun{مالی} را به داروخانه‌چی سفارش می‌دهد. او با \noun{بانتام لاینز} برخورد می‌کند که اشتباهاً فکر می‌کند \noun{بلوم} دارد به او راجع به اسب \noun{ثرواوی} در مسابقهٔ بعدازظهر \noun{گلدکاپ} راهنمایی می‌کند.

    حوالی ساعت 11 صبح، \noun{بلوم} همراه با \noun{سایمون ددالوس} (پدر \noun{استیون})، \noun{مارتین کانینگهام} و \noun{جک پاور} به مراسم خاکسپاری \noun{پدی دیگنام} می‌رود. مردها با \noun{بلوم} مثل یک غریبه برخورد می‌کنند. در مراسم خاکسپاری، \noun{بلوم} به مرگ پسرش و پدرش فکر می‌کند.

    ظهر، \noun{بلوم} را در دفتر روزنامهٔ \noun{فریمن} می‌بینیم که دارد دربارهٔ یک آگهی برای \noun{کیزِ} مشروب‌فروش مذاکره می‌کند. چندین مردِ علاف منجمله \noun{مایلس کرافوردِ} ویراستار، در دفتر می‌چرخند و بحث‌های سیاسی می‌کنند. \noun{بلوم} برای حتمی‌کردن آگهی خارج می‌شود. \noun{استیون} با نامهٔ \noun{دیزی} وارد دفتر روزنامه می‌شود. \noun{استیون} و بقیهٔ مردها همزمان با بازگشت \noun{بلوم} دارند به سمت میخانه می‌روند. مذاکرات آگهی \noun{بلوم} توسط \noun{کرافورد} که دارد بیرون می‌رود، رد می‌شود.

    در ساعت 1 بعدازظهر، \noun{بلوم} با \noun{جوسی برین}، عشق قدیمی‌اش برخورد می‌کند و راجع به \noun{مینا پیورفوی} که در زایشگاه بستری است، صحبت می‌کنند. \noun{بلوم} وارد رستوران \noun{برتون} می‌شود ولی تصمیم می‌گیرد به سمت \noun{دیوی برن} برود تا نهاری سبک بخورد. \noun{بلوم} به یاد بعدازظهری عاشقانه با \noun{مالی} در \noun{هاوث} می‌افتد. \noun{بلوم} خارج شده و دارد به سمت کتابخانهٔ ملی می‌رود که \noun{بویلان} را در خیابان می‌بیند و به داخل موزهٔ ملی پناه می‌برد.

    در ساعت 2 بعدازظهر، \noun{استیون} دارد «تئوری هملت» خود را به طور غیررسمی برای \noun{اِی.ای} شاعر و \noun{جان اگلینتون}، \noun{بست} و \noun{لیستر} کتابدار توضیح می‌دهد. \noun{اِی.ای}، تئوری \noun{استیون} را سبک می‌شمارد و خارج می‌شود. \noun{باک} وارد می‌شود و با تمسخر، \noun{استیون} را به خاطر قال گذاشتن او و \noun{هینز} در میخانه سرزنش می‌کند. در راه خروجی، \noun{باک} و \noun{استیون} از کنار \noun{بلوم} می‌گذرند که آمده تا رونوشتی از آگهی \noun{کیز} بردارد.

    در ساعت 4 بعدازظهر، \noun{سایمون ددالوس}، \noun{بن دالرد}، \noun{لنه‌هان} و \noun{بلیزس بویلان} در نوشگاه هتل \noun{اورموند} گرد هم می‌آیند. \noun{بلوم} متوجه ماشین \noun{بویلان} در بیرون هتل می‌شود و تصمیم می‌گیرد او را زیر نظر بگیرد. \noun{بویلان} خیلی زود برای قرارش با \noun{مالی} خارج می‌شود و \noun{بلوم} با عصبانیت در رستوران \noun{اورموند} می‌نشیند - او موقتاً با آوازخوانی \noun{ددالوس} و \noun{دالرد} آرام می‌شود. \noun{بلوم} جواب نامهٔ \noun{مارتا} را می‌نویسد و می‌رود که نامه را پست کند.

    در ساعت 5 بعدازظهر، \noun{بلوم} به میخانهٔ \noun{بارنی کیرنان} می‌رود تا با \noun{مارتین کانینگهام} درباره مسائل مالی خانوادهٔ \noun{دیگنام} صحبت کند، ولی \noun{کانینگهام} هنوز نرسیده است. \noun{شهروند}، یک میهن‌پرست خشونت‌گرای ایرلندی، سیاه مست می‌شود و به یهودی بودنِ \noun{بلوم} می‌تازد. \noun{بلوم} جلوی \noun{شهروند} می‌ایستد و از صلح و عشق دربرابر خشونت بیگانه‌هراسانه دفاع می‌کند. \noun{بلوم} و \noun{شهروند}، قبل از اینکه کالسکهٔ \noun{کانینگهام}، \noun{بلوم} را از صحنه دور کند، با هم در خیابان مشاجره می‌کنند.

    حوالی غروب آفتاب، \noun{بلوم} بعد از رفتن به خانهٔ خانم \noun{دیگنام}، در ساحل \noun{سندی‌مونت} استراحت می‌کند. زنی جوان که \noun{گرتی مک‌داول} نام دارد متوجه می‌شود که \noun{بلوم} دارد از ساحل به او نگاه می‌کند. \noun{گرتی} عمداً پایش را بیشتر و بیشتر نشان \noun{بلوم} می‌دهد در حالیکه \noun{بلوم} دارد یواشکی استمنا می‌کند. \noun{گرتی} می‌رود و \noun{بلوم} چرت می‌زند.

    ساعت 10 شب، \noun{بلوم} به زایشگاه می‌رود تا به \noun{مینا پیورفوی} سر بزند. \noun{استیون} و چند نفر از دوستانش که دانشجوی پزشکی‌اند نیز در بیمارستان هستند و مشغول نوشیدن و وراجی با صدای بلند راجع به مسائل مرتبط با تولد هستند. \noun{بلوم} قبول می‌کند که به آنها ملحق شود، هر چند که درنهان به خاطر تقلای خانم \noun{پیورفوی} در طبقهٔ بالا، با عیاشی آنها مخالف است. \noun{باک} از راه می‌رسد و مردها به میخانهٔ \noun{بورک} می‌روند. موقع تعطیل شدن میخانه، \noun{استیون}، دوستش \noun{لینچ} را راضی می‌کند که به فاحشه‌خانه بروند و \noun{بلوم} آنها را تعقیب می‌کند تا مراقبشان باشد.

    \noun{بلوم} بالاخره، \noun{استیون} و \noun{لینچ} را در فاحشه‌خانهٔ \noun{بلا کوهن} می‌یابد. \noun{استیون} مست است و فکر می‌کند که دارد روح مادرش را می‌بیند - مملو از خشم و دیوانگی، چراغی را با چوبدستی‌اش خرد می‌کند. \noun{بلوم} دنبال \noun{استیون} می‌رود و او را در بحث با یک سرباز انگلیسی که \noun{استیون} را کتک می‌زند، می‌یابد.

    \noun{بلوم}، \noun{استیون} را به هوش می‌آورد و او را به استراحتگاه رانندگان تاکسی می‌برد تا قهوه‌ای بخورد و سر حال بیاید. \noun{بلوم}، \noun{استیون} را به خانه‌اش دعوت می‌کند.

    بعد از نیمه‌شب، \noun{استیون} و \noun{بلوم} به خانهٔ \noun{بلوم} می‌روند. آنها کاکائوی داغ می‌خورند و دربارهٔ گذشته‌شان صحبت می‌کنند. \noun{بلوم} از \noun{استیون} می‌خواهد که شب را بماند. \noun{استیون} مودبانه تقاضای او را رد می‌کند. \noun{بلوم} او را بدرقه می‌کند و به داخل برمی‌گردد تا شواهد حضور \noun{بویلان} را پیدا کند. \noun{بلوم} هنوز حالش خوب است و به رختخواب می‌رود، و داستان روزش را برای \noun{مالی} تعریف کرده و از او می‌خواهد صبحانه‌اش را به رختخواب بیاورد.

    بعد از اینکه \noun{بلوم} خوابش می‌برد، \noun{مالی} بیدار می‌ماند و از تقاضای \noun{بلوم} برای آوردن صبحانه به رختخواب در تعجب است. ذهن او به دوران کودکی‌اش در \noun{جبل‌الطارق}، سکس بعدازظهرش با \noun{بویلان}، حرفهٔ خوانندگی‌اش و \noun{استیون ددالوس} مشغول است. تفکراتش راجع به \noun{بلوم} در طی مونولوگی که با خود دارد به تندی تغییر می‌کند ولی در انتها با یادآوری لحظات عاشقانه‌ای که در \noun{هاوث} داشتند و با دیدی مثبت به پایان می‌رسد.
    
    \part{اپیزودها}
    \chapter[تلماخوس]{تلماخوس\protect\footnote{\lr{Telemachus}-\rl{ در اسطوره‌های یونان پسر اولیس و پنلوپه است. در کودکی پسری ترسو بود. اما آتنه به او شجاعت بخشید. در دوران سرگردانی پدر به جست‌وجویش رفت.}}}\label{ep:1}
    ساعت حدود 8 صبح است و \noun{باک مالیگان‬}، در حال تقلید و مسخرهٔ مراسم عشاء ربانی با کاسهٔ ریش‌تراشی‌اش، \noun{استیون ددالوس} را صدا می‌زند تا بالای سقف قلعهٔ \noun{مارتلو} که مشرف بر خلیج دوبلین است، بیاید.  \noun{استیون} به مسخره‌بازیِ پرخاشگرانهٔ \noun{باک}  بی‌توجه است - او حوصلهٔ \noun{هینز}  را ندارد، فرد انگلیسی‌ای که \noun{باک}  دعوتش کرده تا در قلعه بماند. \noun{استیون} با ناله‌های \noun{هینز}  دربارهٔ کابوسی که در آن یک پلنگ سیاه دیده بود، از خواب شبانه بیدار شده است.

    \noun{مالیگان} و \noun{استیون} به دریا نگاه می‌کنند که \noun{باک}  از آن به \noun{مادر کبیر} یاد می‌کند. این کار، \noun{مالیگان} را به یاد غضب عمه‌اش نسبت به \noun{استیون} می‌اندازد چرا که \noun{استیون} قبول نکرده بود کنار بستر مرگ مادر خودش دعا کند. \noun{استیون} که هنوز لباس عزا به تن دارد به دریا می‌نگرد و به مرگ مادرش فکر می‌کند، در حالی که \noun{باک} ، \noun{استیون} را به خاطر لباس‌‌های دست دوم و ظاهر کثیفش دست می‌اندازد. \noun{باک}  یک آینهٔ شکسته را جلوی \noun{استیون} می‌گیرد تا خودش را در آن ببیند. \noun{استیون} همدردی \noun{باک}  را رد می‌کند و اظهار می‌کند که چنین «آینهٔ شکسته‌ای از یک نوکر» نشانه‌ای از هنر ایرلندی است. \noun{باک}  بازویش را به نشانه همدردی دور \noun{استیون} حلقه می‌کند و می‌گوید که آنها با هم می‌توانند ایرلند را به سطحی از فرهنگ، همتای یونان باستان برسانند. \noun{باک}  پیشنهاد می‌دهد در صورتی که \noun{هینز} ، دوباره \noun{استیون} را برنجاند، او را تهدید کنند و \noun{استیون} به یاد «تحقیر و آزار» یکی از همکلاسی‌هایشان به نام \noun{کلایو کمپتورپ}\LTRfootnote{\lr{Clive Kempthorpe}} توسط \noun{باک}  می‌افتد.

    \noun{باک}  از \noun{استیون} درباره خشم خاموشش می‌پرسد و \noun{استیون} سرآخر غضبش نسبت به \noun{باک}  را تأیید می‌کند - ماه‌ها قبل، \noun{استیون} شنیده بود که \noun{باک}  مادرش را «مثل سگ، مرده» خطاب کرده بود. \noun{باک}  سعی می‌کند که از خودش دفاع کند، سپس تسلیم می‌شود و \noun{استیون} را ترغیب می‌کند که از خشمگین بودن نسبت به تفاخر و غرور خودش دست بردارد.

    \noun{باک}  وارد قلعه می‌شود و بدون این که بداند، آوازی را می‌خواند که \noun{استیون} برای مادر در حال مرگش خوانده بود. \noun{استیون} احساس می‌کند که توسط مادر مرده‌اش یا خاطره او تسخیر شده است. \noun{باک} ، \noun{استیون} را به طبقه پایین برای صبحانه فرا می‌خواند. او \noun{استیون} را ترغیب می‌کند تا از \noun{هینز} ، که تحت تأثیر طبع ایرلندی \noun{استیون} قرار دارد، درخواست پول کند، ولی \noun{استیون} قبول نمی‌کند. \noun{استیون} به آشپزخانه می‌رود و به \noun{باک}  برای صبحانه کمک می‌کند. \noun{هینز}  اعلام می‌کند که زن شیرفروش دارد می‌آید. \noun{باک}  لطیفه‌ای می‌گوید درباره «مادر پیر، گروگن\LTRfootnote{\lr{Old mother Grogan}}» که چای درست می‌کند و آب (ادرار) درست می‌کند و \noun{هینز}  را تشویق می‌کند که از آن لطیفه در کتابی دربارهٔ زندگی مردم ایرلند استفاده کند.

    زن شیرفروش وارد می‌شود، و \noun{استیون} او را به شکل نمادی از ایرلند تصور می‌کند. \noun{استیون} از این که زن شیرفروش به \noun{باک} ، دانشجوی پزشکی، بیشتر از او احترام می‌گذارد در نهان ناراحت است. \noun{هینز}  با او (زن) به ایرلندی صحبت می‌کند ولی او (زن) حرفش را نمی‌فهمد و فکر می‌کند که (هینز) دارد فرانسوی صحبت می‌کند. \noun{باک}  پول او را پرداخت می‌کند و زن می‌رود.

    \noun{هینز}  می‌گوید که تمایل دارد تا از گفته‌های \noun{استیون} کتابی بنویسد، ولی \noun{استیون} می‌پرسد آیا از آن پولی به دست می‌آید یا نه. \noun{هینز}  بیرون می‌رود و \noun{باک} ، \noun{استیون} را به خاطر گستاخ بودن و از بین بردن فرصتشان برای گرفتن پول میگساری از \noun{هینز}  سرزنش می‌کند. \noun{باک}  لباس می‌پوشد و هر سه مرد به سمت آب می‌روند. در راه، \noun{استیون} توضیح می‌دهد که قلعه را از وزیر جنگ اجاره کرده است. \noun{هینز}  از \noun{استیون} دربارهٔ تئوری هملت‌اش می‌پرسد ولی \noun{باک}  اصرار می‌کند آن را به بعد از میگساری به تأخیر بیاندازند. \noun{هینز}  توضیح می‌دهد که قلعهٔ \noun{مارتلو}ی آنها، او را به یاد \noun{ال-سینور}\LTRfootnote{\lr{El-sinore}} هملت می‌اندازد. \noun{باک} ، حرف \noun{هینز}  را قطع می‌کند تا پیش بیفتد، برقصد و «تصنیف عیسی بذله‌گو» را بخواند.

    \noun{هینز}  و \noun{استیون} با هم راه می‌روند. همزمان با صحبت \noun{هینز} ، \noun{استیون} حدس می‌زند که \noun{باک}  کلید قلعه را بخواهد - قلعه‌ای که \noun{استیون} پول اجاره‌اش را می‌دهد. \noun{هینز}  از \noun{استیون} راجع به عقاید مذهبی‌اش می‌پرسد. \noun{استیون} توضیح می‌دهد که دو ارباب، انگلستان و کلیسای کاتولیک، بر سر راه تفکر آزادش ایستاده اند و ارباب سوم، ایرلند، از او، «کارهای عجیب و غریب» می‌خواهد. \noun{هینز} ، در حالی که سعی می‌کند درباره بردگی ایرلند نسبت به بریتانیا دوستانه رفتار کند، به نرمی می‌گوید «به نظر باید تاریخ را سرزنش کرد». \noun{هینز}  و \noun{استیون} به خلیج خیره می‌شوند و \noun{استیون} مردی را به خاطر می‌آورد که به تازگی غرق شده است.

    \noun{هینز}  و \noun{استیون} به سمت آب می‌روند، جایی که \noun{باک}  دارد لباس‌هایش را در می‌آورد و دو نفر دیگر، شامل یکی از دوستان \noun{باک} ، دارند شنا می‌کنند. \noun{باک}  با دوستش راجع به دوست مشترکشان، \noun{بانون} که در \noun{وستمیث}\LTRfootnote{\lr{Westmeath}} است، صحبت می‌کند - \noun{بانون} ظاهراً دوست‌دختری دارد (که بعداً می‌فهمیم \noun{میلی بلوم} است). \noun{باک}  به آب می‌زند، در حالی که \noun{هینز}  سیگار می‌کشد. \noun{استیون} اعلام می‌کند که دارد می‌رود و \noun{باک}  از او درخواست کلید قلعه و دو پنی برای یک پینت\footnote{\lr{Pint}-\rl{واحد حجم معادل يک هشتم گالن}} آبجو می‌کند. \noun{باک}  با \noun{استیون} در ساعت 12:30 در میخانهٔ \noun{کشتی}\LTRfootnote{\lr{The Ship}} قرار می‌گذارد. \noun{استیون} می‌رود و عهد می‌کند که امشب به قلعه باز نگردد چرا که \noun{باک} ِ «غاصب»، آن را به چنگ آورده است.

    \chapter[نستور]{نستور\protect\footnote{\lr{Nestor}-\rl{در اسطوره‌های یونان، شاه پولوس است. پسر نرئوس و خلوریس بود. از میان دوازده پسر نرئوس تنها او بود که از حملهٔ هراکلس به پولوس جان سالم در برد. در جنگ تروا سالخورده‌ترین و عاقل‌ترین جنگاور محسوب می‌شد.}}}\label{ep:2}
    \noun{استیون} در حال تدریس در کلاس تاریخ درباره پیروزی \noun{پیروس}\footnote{\lr{Pyrrhus}-\rl{ژنرال و سیاستمدار یونانی}} است - کلاس خیلی نظم و ترتیب ندارد. او به دانش‌آموزان تمرین می‌دهد و پسری به نام \noun{آرمسترانگ}\LTRfootnote{\lr{Armstrong}} حدس می‌زند که از لحاظ آواشناختی، \noun{پیروس} یک «اسکله\LTRfootnote{\lr{Pier}}» بود. \noun{استیون} با او مخالفت نمی‌کند و پیرو جواب \noun{آرمسترانگ} می‌گوید که یک اسکله، «یک پل ناتمام» است. او خودش را تصور می‌کند که بعداً چاپلوسانه این لطیفه را برای خوشایندِ \noun{هینز}  تعریف می‌کند. متفکر درباره قتل \noun{پیروس} و سزار، \noun{استیون} به ناگزیریِ فلسفی برخی وقایع تاریخی می‌اندیشد - آیا تاریخ، به سرانجام رسیدن تنها حالت ممکنِ سلسله وقایع است یا یکی از حالات بیشمارِ آن می‌باشد؟

    \noun{استیون} بحث کلاس را به سمت \noun{لیسیداسِ} \noun{میلتون}\LTRfootnote{\lr{Milton's \protect\textit{Lycidas}}} می‌برد و همچنان به تفکر درباره سوالات خودش درباره تاریخ ادامه می‌دهد، سوالاتی که هنگام خواندن ارسطو در کتابخانهٔ پاریس به آنها فکر کرده بود. تصویری از شعر \noun{میلتون}، \noun{استیون} را به تفکر درباره تأثیر خدا روی همه آدمیان وا می‌دارد. \noun{استیون} به خطوط یک معمای پیش پا افتاده فکر می‌کند و سپس تصمیم می‌گیرد به دانش‌آموزان که دارند وسائلشان را جمع می‌کنند که بروند در زمین هاکی بازی کنند، معمای خودش را بگوید. \noun{استیون} در تنهایی به معمای حل‌نشدنی خودش درباره «روباهی که مادربزرگش را زیر یک بوته به خاک می‌سپارد\LTRfootnote{\lr{A fox burying his grandmother under a bush}}» می‌خندد.

    دانش‌آموزان کلاس را ترک می‌کنند به غیر از \noun{سارجنت}\LTRfootnote{\lr{Sargent}} که به کمک در درس حساب نیاز دارد. \noun{استیون} به \noun{سارجنتِ} زشت نگاه می‌کند و عشق مادر \noun{سارجنت} نسبت به خودش را تصور می‌کند. \noun{استیون} به \noun{سارجنت}، حاصلجمع‌ها را نشان می‌دهد، و کمی به لطیفهٔ \noun{باک}  فکر می‌کند که می‌گفت تئوریِ هملتِ \noun{استیون} را می‌توان با جبر اثبات کرد. با فکر دوباره به \lr{amor matris} یا عشق مادر، \noun{استیون} خودش را به شکل کودکی به یاد می‌آورد که مانند \noun{سارجنت} بدترکیب بود. \noun{سارجنت} بیرون می‌رود تا به بازی هاکی بپیوندد. \noun{استیون} بیرون می‌رود، سپس می‌رود تا در دفتر کار \noun{دیزی} منتظر بماند در حالی که \noun{دیزی}، مدیر مدرسه، در حال رفع و رجوع دعوای بچه‌ها سر بازی هاکی است.

    آقای \noun{دیزی}، دستمزد \noun{استیون} را پرداخت می‌کند و صندوق ذخیره‌اش را به رخ می‌کشد. \noun{دیزی} برای \noun{استیون} خطابه‎‌ای راجع به ارضا شدن با پولِ به دست آمده و اهمیت نگهداری دقیق پول و ذخیره کردن آن ایراد می‌کند. \noun{دیزی} خاطرنشان می‌کند که بزرگترین افتخار یک انگلیسی این است که می‌تواند ادعا کند که هزینه‌هایش را خودش پرداخت کرده و هیچ بدهی ندارد. \noun{استیون} قرض‌های فراوان خودش را ذهنی جمع می‌زند.

    \noun{دیزی} می‌پندارد که \noun{استیون}، که \noun{دیزی} فکر می‌کند \noun{فنیان}\LTRfootnote{\lr{Fenian}} (ملی‌گرای کاتولیک ایرلندی) است، به \noun{دیزی} که \noun{توری}\LTRfootnote{\lr{Tory}} (پروتستان وفادار به انگلستان) است بی‌احترامی می‌کند. \noun{دیزی} درباره اعتبار ایرلندی‌اش بحث می‌کند - او شاهد اغلب ماجراهای ایرلند بوده است. \noun{دیزی} سپس از \noun{استیون} می‌خواهد که از نفوذش استفاده کند و نوشته‌ای از او را در روزنامه به چاپ برساند. در حالی که او دارد تایپ آن را به پایان می‌رساند، \noun{استیون} نگاهی به تصاویر اسب‌های مسابقه در دفتر کار او می‌اندازد و یاد گردشی به پیست مسابقه همراه دوست قدیمی‌اش، \noun{کرانلی}\LTRfootnote{\lr{Cranly}} می‌افتد.

    \noun{استیون} فریادهایی را می‌شنود که به خاطر گلی در مسابقه هاکی سرداده شده است. \noun{دیزی}، نوشتهٔ تکمیل شده‌اش را به \noun{استیون} می‌دهد و \noun{استیون} آن را به سرعت می‌قاپد. نوشته، خطرات بیماری تب برفکی احشام را گوشزد می‌کند و اظهار می‌دارد که آن را می‌شود درمان نمود. به نظر می‌رسد که \noun{دیزی} از تأثیر افرادی که در حال حاضر روی قضیه احاطه دارند متنفر است. همچنین به نظر می‌رسد که یهودیان را برای فساد مالی و نابودی اقتصاد ملی سرزنش می‌کند. \noun{استیون} بحث می‌کند که بازرگانانِ حریص می‌توانند یهودی یا غیر یهودی باشند، ولی \noun{دیزی} اصرار دارد که یهودی‌های نسبت به «نور\LTRfootnote{\lr{The light}}» گناه کرده‌اند.

    \noun{استیون}، بازرگانان یهودی را که بیرونِ بازارِ بورسِ پاریس می‌ایستادند به خاطر می‌آورد. \noun{استیون} دوباره با \noun{دیزی} بحث می‌کند، و می‌پرسد چه کسی نسبت به نور گناه نکرده است. \noun{استیون}، تعبیر \noun{دیزی} از گذشته را رد می‌کند و می‌گوید: «تاریخ، کابوسی است که می‌خواهم از آن بیدار شوم». به طور طعنه آمیزی، همان موقع که \noun{دیزی} دارد دربارهٔ تاریخ به مثابه حرکت به سمت «هدف» جلوهٔ خدا حرف می‌زند، گلی در بازی هاکی به هدف می‌نشیند\footnote{ گل (ورزش) و هدف در زبان انگلیسی هر دو معادل کلمه \lr{goal} هستند.}. \noun{استیون} جواب می‌دهد که خدا چیزی بیش از «فریادی در خیابان» نیست. \noun{دیزی} ابتدا بحث می‌کند که همه گناه کرده‌اند، سپس زنان را برای آوردن گناه به این دنیا سرزنش می‌کند. او فهرستی از زنان را بیان می‌کند که در طول تاریخ باعث نابودی و تباهی شده‌اند.

    \noun{دیزی} پیش‌بینی می‌کند که \noun{استیون} زیاد در مدرسه باقی نخواهد ماند، چرا که یک معلم بالفطره نیست. \noun{استیون} می‌گوید که او بیشتر یک یادگیرنده است یا یاددهنده. \noun{استیون} با بازگشت به موضوع نوشتهٔ \noun{دیزی}، خاتمه بحث را پیش می‌کشد. \noun{استیون} سعی خواهد کرد که آن را در دو روزنامه به چاپ برساند. \noun{استیون} از مدرسه بیرون می‌رود و به خوش‌خدمتی خودش نسبت به \noun{دیزی} فکر می‌کند. \noun{دیزی} به دنبال او می‌رود تا آخرین ضربه را به یهودی‌ها بزند - ایرلند هیچ وقت در حق یهودی‌ها جفا نکرده است چرا که آنها هیچ وقت اجازه ورود به کشور را نداشته‌اند.

    \chapter[پروتئوس]{پروتئوس\protect\footnote{\lr{Proteus}-\rl{در اسطوره‌های یونان، یکی از خدایان دریا که اشکال مختلف به خود می‌گرفته است. زمانی او را پسر پوزئیدون دانسته‌اند و زمانی ملازم و همنشین وی. او هم از قدرت پیشگویی برخوردار بود و هم از قدرت تغییر شكل در هر زمان كه اراده می كرد.}}}\label{ep:3}
    \noun{استیون} در ساحل قدم می‌زند، و به تفاوت بین دنیای مادی، آنگونه که وجود دارد و آنگونه که در چشمانش ثبت می‌شود فکر می‌کند. \noun{استیون} چشم‌هایش را می‌بندد و خود را به حس شنوایی‌اش می‌سپارد - ریتم‌هایی پدیدار می‌شوند.

    وقتی چشمانش را باز می‌کند متوجه دو قابله می‌شود، خانم \noun{فلورنس مک‌کاب}\LTRfootnote{\lr{Florence MacCabe}} و زنی دیگر. \noun{استیون} تجسم می‌کند که یکی از آنها جنینی سقط شده را در کیفش دارد. او بندناف را به مثابهٔ خط تلفنی تصور می‌کند که به اعماق تاریخ می‌رود و از طریق آن می‌تواند تماسی با «باغ عدن» برقرار کند. \noun{استیون} شکم بی‌ناف حوا را تجسم می‌کند. او به گناه نخست زن و سپس انعقاد نطفهٔ خودش فکر می‌کند. \noun{استیون} انعقاد نطفهٔ خودش را با مسیح مقایسه می‌کند. بنا بر \noun{نایسن کرید}\LTRfootnote{\lr{Nicene Creed}}، قسمتی از مراسم عشاء کاتولیک، مسیح «به وجود آمد، ساخته نشد\LTRfootnote{\lr{Begotten, not made}}» یعنی او قسمتی از جوهر خدای پدر است و توسط خدای پدر از عدم ساخته نشد. در مقابل، \noun{استیون} «ساخته شد، به وجود نیامد» چرا که با وجود داشتن والدین واقعی، روحش از عدم ساخته شد و هیچ ربطی به روح پدرش ندارد. \noun{استیون} دوست داشت در مورد ویژگی‌های انعقاد نطفهٔ الهی با دانشمندان مرتد گذشته بحث کند (آیا پدر و پسر یک چیزند یا نه؟).

    هوای دریا به سمت او می‌وزد و \noun{استیون} به یاد می‌آورد که باید نوشتهٔ \noun{دیزی} را به روزنامه ببرد، سپس \noun{باک}  را در میخانهٔ \noun{کشتی} در ساعت 12:30 ملاقات کنند. به این فکر می‌کند که از ساحل برگردد و به ملاقات زن‌دایی‌اش \noun{سارا} برود. او عکس‌العمل تمسخرآمیز پدرش را نسبت به چنین ملاقاتی تصور می‌کند (پدرش از برادرزنش، \noun{ریچی}، که شوهر \noun{سارا} می‌باشد متنفر است). \noun{استیون} چنین صحنه‌ای را هنگام ملاقات تصور می‌کند: \noun{والتر}، پسر \noun{ریچی} او را به دورن راه می‌دهد و عمو \noun{ریچی}، که مشکل کمر دارد، از رختخواب به او خوشامد می‌گوید.

    وقتی \noun{استیون} از خیال‌پردازی‌هایش بیرون می‌آید، به یاد می‌آورد که وقتی بچه بود از داشتن چنین خانواده‌ای شرمسار بود. این تنفر از خانواده‌اش، \noun{جاناتان سویفت}\LTRfootnote{\lr{Jonathan Swift}} را به ذهن متبادر می‌کند - تنفر \noun{سویفت} از مردم در داستان سفرهای گالیورش در اسب‌های اصیل نژاد \noun{هوینم}\LTRfootnote{\lr{Houyhnhnm}} و آدم‌های ددمنش نژاد \noun{یاهو}\LTRfootnote{\lr{Yahoo}} آشکار است. او به \noun{سویفت} فکر می‌کند که سرش را مانند یک کشیش تراشیده است و از روی حصار مسیر اسب‌دوانی می‌پرد تا از مردم فرار کند. \noun{استیون} به کشیش‌هایی که در همه جای شهر هستند و به تقوی و تظاهرهای روشنفکرمآبانهٔ دوران جوانی‌اش فکر می‌کند.

    \noun{استیون} متوجه می‌شود که از خانهٔ \noun{سارا} رد شده است. وقتی به سمت کبوترخانه می‌رود، دربارهٔ کبوترها فکر می‌کند: مخصوصاً اصرار مریم باکره بر این که توسط یک کبوتر باردار شده است (آنگونه که در زندگی عیسی نوشته \noun{لئو تاکسیل}\LTRfootnote{\lr{Léo Taxil's La Vie de Jesus}} آمده است). او به \noun{پاتریس اگان}\LTRfootnote{\lr{Patrice Egan}} پسر \noun{کوین اگان}\LTRfootnote{\lr{Kevin Egan}} فکر می‌کند که یک «غاز وحشی» (ملی‌گرای ایرلندی در تبعید) است که \noun{استیون} او را در پاریس می‌شناخته است. او خودش را به خاطر می‌آورد که در پاریس دانشجوی پزشکی بود و پول کمی داشت. او به یاد می‌آورد که یک بار آنقدر دیر به ادارهٔ پست رسید که نتوانست حوالهٔ پولی را که از مادرش دریافت کرده بود، نقد کند. بلندپروازی‌های \noun{استیون} برای زندگی‌اش در پاریس به طور ناگهانی با تلگرافی از جانب پدرش که از او خواسته بود به خانه و به بستر مرگ مادرش بیاید، از هم گسیخت. \noun{استیون} باز به عمهٔ \noun{باک}  فکر کرد که اصرار داشت او مادرش را با امتناع از دعا کردن در کنار بستر مرگش کشته است.

    \noun{استیون}، تصاویر و صداهای پاریس و گفتگوهای \noun{کوین اگان} درباره ملی‌گرایی، رسوم عجیب فرانسوی و جوانی‌اش را به خاطر می‌آورد. \noun{استیون} به کنارهٔ دریا می‌رود و برمی‌گردد و در افق، قلعهٔ \noun{مارتلو} را می‌پوید. او دوباره عهد می‌کند که امشب آنجا با \noun{باک}  و \noun{هینز}  نخوابد. او روی صخره‌ای می‌نشیند و متوجه لاشهٔ سگی می‌شود. سگ زنده‌ای در ساحل می‌دود و به سمت دو نفر برمی‌گردد. \noun{استیون} صحنهٔ ساحل را هنگام تهاجم اولین وایکینگ‌های دانمارکی به دوبلین، تصور می‌کند.

    سگی که پارس می‌کند به سمت \noun{استیون} می‌دود و \noun{استیون} به ترسش از سگ‌ها می‌اندیشد. با درنظرگرفتن مدعیان مختلف تاج و تخت در طول تاریخ، \noun{استیون} فکر می‌کند که آیا او هم یک مدعی است یا نه. او متوجه می‌شود که دو پیکرهٔ همراه سگ، یک مرد و یک زنِ صدف جمع‌کن هستند. او می‌بیند که سگ، لاشه را بو می‌کند و توسط صاحبش مورد عتاب قرار می‌گیرد. سگ می‌شاشد و سپس در شن‌ها چاله می‌کند. \noun{استیون} به یاد معمای صبحش دربارهٔ روباهی که مادربزرگش را دفن کرده بود می‌افتد.

    \noun{استیون} سعی می‌کند خوابی را که شب پیش دیده بود به خاطر آورد: مردی که هندوانه‌ای در دستش بود روی یک فرش قرمز به سمت \noun{استیون} می‌آمد. با نگاه به زن صدف جمع‌کن، \noun{استیون} به یاد یکی از ماجراهای جنسی قدیمی‌اش در مسیر \noun{فومبالی}\LTRfootnote{\lr{Fumbally}} می‌افتد. آن زوج از کنار \noun{استیون} عبور می‌کنند و به کلاهش نظر می‌اندازند. \noun{استیون} در ذهنش شعری می‌سازد و آن را روی تکه‌ کاغذی که از نوشتهٔ \noun{دیزی} پاره کرده است می‌نویسد. \noun{استیون} فکر می‌کند که «زن\LTRfootnote{\lr{She}}» شعرش چه کسی خواهد بود. او تشنهٔ محبت است. \noun{استیون} دراز می‌کشد به پوتین‌های عاریه‌ای‌اش و پاهای کوچکش که زمانی در کفش زنی جای می‌گرفته‌اند می‌اندیشد. او می‌شاشد. او باز به بدن مرد غرق شده فکر می‌کند. \noun{استیون} بلند می‌شود که برود، انگشت در دماغش می‌کند، سپس به اطراف نگاه می‌کند که مبادا کسی او را دیده باشد. او کشتی‌ای را می‌بیند که نزدیک می‌شود.

    \chapter[کالوپسو]{کالوپسو\protect\footnote{\lr{Calypso}-\rl{در اسطوره‌های یونان، یک پری دریایی است. دختر اطلس بود و در جزیره اوگوگیا زندگی می‌کرد. اولیس در راه بازگشت از تروا، به این جزیره وارد شد. کالوپسو به او دل بست و او را هفت سال نزد خود نگاه داشت. به اولیس پیشنهاد کرد همیشه با او بماند و جاودان شود. اما اولیس در هوای خانه بود. عاقبت زئوس، هرمس را فرستاد تا کالوپسو را راضی کند دست از اولیس بردارد.}}}\label{ep:4}
    \noun{لئوپلد بلوم} برای زنش، \noun{مالی}، صبحانه درست می‌کند و به گربه‌اش غذا می‌دهد. در حالی که با دست‌هایش روی زانوانش خم شده است، فکر می‌کند که از نظر گربه‌اش چه شکلی به نظر می‌رسد و وقتی گربه شیر می‌خورد، سبیل‌هایش به چه وضعی در می‌آیند. \noun{بلوم} فکر می‌کند که از قصاب برای صبحانهٔ خودش چه بگیرد. او به طبقهٔ بالا می‌رود تا از \noun{مالی} بپرسد که چیزی از بیرون می‌خواهد یا نه. \noun{مالی} زیرلب می‌گوید نه و تخت، زیر او جیرجیر می‌کند. \noun{بلوم} به تخت فکر می‌کند که \noun{مالی} آن را با خودش از \noun{جبل‌الطارق} آورده بود، جایی که پیش پدرش، سرگرد \noun{توییدی}، بزرگ شده بود.

    \noun{بلوم} تکه کاغذی در کلاهش و سیب‌زمینی‌اش\footnote{بلوم از روی خرافات، همراه خود یک سیب‌زمینی دارد که معتقد است خوش‌شانسی می‌آورد.} را بررسی می‌کند و یادداشتی می‌نویسد تا قبل از اینکه خانه را برای طول روز ترک کند، کلیدهایش را از طبقهٔ بالا بردارد. \noun{بلوم} بیرون می‌رود و پیش‌بینی می‌کند که با لباس‌های سیاهی که امروز برای خاکسپاری \noun{پدی دیگنام} خواهد پوشید، گرمش می‌شود. او در خیالش تجسم می‌کند که دارد مسیری را در وسط دنیا در برابر مسیر آفتاب طی می‌کند تا در همان سن باقی بماند و چشم‌اندازهای شرقی را متصور می‌شود. ولی او با خود دلیل می‌آورد که نه، تصاویر ذهنی او، ساختگی هستند و دقیق نمی‌باشند. \noun{بلوم} از جلوی میخانهٔ \noun{لری اُرورک}\LTRfootnote{\lr{Larry O'Rourke}} می‌گذرد و فکر می‌کند که آیا خوب است توقف کند و خاکسپاری \noun{دیگنام} را یادآوری کند یا نه، ولی به جای آن برای \noun{اُرورک} روز خوبی آرزو می‌کند. \noun{بلوم}، با فرض وجودِ تعداد زیادی میخانه در دوبلین، سعی می‌کند بفهمد که میخانه‌داران خرده‌پایی مثل \noun{اُرورک} چگونه پول درمی‌آورند. \noun{بلوم} از جلوی مدرسه‌ای رد می‌شود و به دانش‌آموزانی گوش فرا می‌دهد که الفبا و نام‌های ایرلندی مکان‌ها را از بر می‌خواند. \noun{بلوم} به نام ایرلندی مکان خود فکر می‌کند، «اسلیو بلوم\LTRfootnote{\lr{Slieve Bloom}}».

    \noun{بلوم} به \noun{لوگاچ}\LTRfootnote{\lr{Dlugacz}}، دکان قصابی می‌رسد. می‌بیند که یک قلوه مانده و آرزو می‌کند زنی که جلوی او است، آن را نخرد. \noun{بلوم} برگه‌ای روزنامهٔ بسته‌بندی برمی‌دارد و تبلیغات آن را می‌خواند. زن پول سفارش خود را پرداخت می‌کند و \noun{بلوم} به آن قلوه اشاره می‌کند با این امید که سفارشش زود انجام شود و بتواند آن زن را تا خانه‌اش تعقیب کند و بالا و پایین شدنِ کفل‌هایش را تماشا کند. از آنجا که برای رسیدن به آن زن دیر شده بود، به خواندن برگه روزنامه‌اش در راه خانه ادامه می‌دهد. در آن روزنامه تبلیغ کاشت میوه برای احتکار در فلسطین آمده بود و \noun{بلوم} به میوه‌های مدیترانه و خاورمیانه فکر می‌کند. \noun{بلوم} از کنار مردی که می‌شناسد می‌گذرد و آن مرد او را نمی‌بیند.

    وقتی ابری جلوی خورشید را گرفت، تفکرات \noun{بلوم} با تصویر تهی‌تری از خاور میانه و تراژدی نژاد یهود، تیره شد. \noun{بلوم} با خودش عهد می‌کند که با شروع دوبارهٔ ورزش صبحگاهی‌اش، حالش را بهتر کند، سپس توجه خود را به قسمت اجاره داده نشده‌ای از املاک خیابان و نهایتاً به \noun{مالی} جلب می‌کند. خورشید دوباره بیرون می‌آید و دختر بلوندی از جلوی \noun{بلوم} می‌دود.

    \noun{بلوم} دو نامه و یک کارت در سرسرا می‌یابد. \noun{بلوم} حس می‌کند که نامهٔ \noun{مالی} از طرف \noun{بلیزس بویلان‬}، شریک و احتمالاً عاشق \noun{مالی} است. وقتی وارد اتاق خواب می‌شود، نامه و کارتی از طرف دخترشان \noun{میلی} که در \noun{مالینگار} است را به \noun{مالی} می‌دهد. \noun{مالی} نامهٔ \noun{بویلان} را زیر بالشش می‌گذارد و کارت \noun{میلی} را می‌خواند. \noun{بلوم} به طبقه پایین می‌رود تا چای و قلوه را آماده کند. او نامهٔ مربوط به خودش را که از طرف \noun{میلی} است، با سرعت برمی‌دارد.

    \noun{بلوم}، صبحانهٔ \noun{مالی} را در تخت برایش می‌برد. \noun{بلوم} از او راجع به نامه‌اش می‌پرسد و او توضیح می‌دهد که \noun{بویلان} قرار است این بعدازظهر برای برنامه‌ریزی کنسرتی بیاید. \noun{مالی} «دست در دست خواهیم بود\LTRfootnote{\lr{Là ci darem}}» و «آواز دلنشین قدیمی عشق\LTRfootnote{\lr{Love's Old Sweet Song}}» را خواهد خواند. \noun{مالی} از \noun{بلوم} می‌خواهد که کتابی برایش بیاورد. وقتی \noun{بلوم} دارد دنبال کتاب می‌گردد، خطوطی از «دست در دست خواهیم بود» را در سرش تکرار می‌کند، با این فکر که آیا \noun{مالی} آنها را درست تلفظ خواهد کرد یا نه. \noun{مالی} کتاب را می‌گیرد، یک داستان مهیج به نام: «یاقوت: غرورِ حلقه»، و کلمه‌ای را که می‌خواست از \noun{بلوم} بپرسد می‌یابد - «تناسخ». \noun{بلوم} از لحاظ ریشه‌شناسی توضیح می‌دهد، ولی \noun{مالی} می‌خواهد که معنی ساده و سرراست آن را بداند. \noun{بلوم}، کلمهٔ حلول را توضیح می‌دهد. با مشاهده نقاشی‌ای از یک حوری بالای تختشان، او به زنش مثالی از حوری‌هایی می‌زند که به شکل دیگری مانند یک درخت بازگشته‌اند. \noun{مالی} کتاب دیگری از \noun{پاول دو کوک}\LTRfootnote{\lr{Paul de Kock}} می‌خواهد.

    \noun{مالی} بوی سوختن قلوهٔ \noun{بلوم} را حس می‌کند و او به طبقه پایین می‌دود تا از سوختن آن جلوگیری کند. \noun{بلوم} برای خوردن می‌نشیند و نامهٔ \noun{میلی} را دوباره می‌خواند. او از \noun{بلوم} برای کادوی تولدش تشکر کرده و به یک دوست پسر به نام \noun{بانون} اشاره کرده است. \noun{بلوم} به کودکی \noun{میلی} و پسرش \noun{رودی} که چند روز بعد از تولدش مرد، فکر می‌کند. به زن شدنِ \noun{میلی} و آگاه شدن از جذابیت‌هایش فکر می‌کند. از آنجا که \noun{میلی} در نامه‌اش به \noun{بویلان} اشاره کرده بود، \noun{بلوم} به اعتماد به نفس \noun{بلیزس بویلان‬} فکر می‌کند و احساس درماندگی و تأثر می‌کند. \noun{بلوم} به ملاقات \noun{بویلان} با \noun{میلی} فکر می‌کند.

    \noun{بلوم} یک نسخه از مجله \noun{تیت‌بیتس}\LTRfootnote{\lr{Titbits}} برمی‌دارد و به حیاط بیرونی برای قضای حاجت می‌رود. \noun{بلوم} به نقشه‌هایی که برای باغچه‌اش دارد فکر می‌کند. در دستشویی، \noun{بلوم} داستان «شاهکارِ ماچام\LTRfootnote{\lr{Matcham's Masterstroke}}» نوشته \noun{فیلیپ بوفوی}\LTRfootnote{\lr{Philip Beaufoy}} را می‌خواند. با رضایت از کارکردن منظم شکمش، داستان را به اتمام می‌رساند و به این فکر می‌کند که خودش می‌تواند داستانی بنویسد و در ازای آن پول دریافت کند. او می‌توانست درباره یک ضرب‌المثل یا درباره وِروِر کردن‌های \noun{مالی} بنویسد. \noun{بلوم} خودش را با تکه‌ای از داستان پاک می‌کند. به خودش یادآوری می‌کند که زمان خاکسپاری را در کاغذ بررسی کند. با شنیدن صدای ناقوس کلیسا، او با افسوس به \noun{دیگنام} فکر می‌کند.

    \chapter[لوتوفاگ‌ها]{لوتوفاگ‌ها\protect\footnote{\lr{Lotus-eaters} یا \lr{Lotophagi} یا \lr{Lotophaguses}-\rl{به معنی خورندگان نیلوفر آبی، در اسطوره‌های یونان، نام قبیله‌ای است که در ساحل لیبی زندگی می‌کردند. اولیس با همراهانش به جزیره لوتوفاگ‌ها وارد شد. آن‌ها از میوه نیلوفر آبی خوردند و حافظه خود را از دست دادند.}}}\label{ep:5}
    \noun{بلوم} از مسیری غیرمستقیم به سمت ادارهٔ پست مرکز شهر می‌رود، و به آدم‌هایی که از کنارشان می‌گذرد و به مراسم خاکسپاری که در ساعت 11:00 در آن شرکت خواهد داشت فکر می‌کند. در حالی که برچسب بسته‌های درون ویترین شرکت «بلفاست و چای شرقی\LTRfootnote{\lr{Belfast and Oriental Tea Company}}» را می‌خواند، کارت پستالِ با نام مستعارِ خود را بیرون می‌آورد، \noun{هنری فلاور}. تحت تأثیر برچسب‌های چای، \noun{بلوم} فضای مست‌کنندهٔ مشرق را تصور می‌کند. او یواشکی وارد ادارهٔ پست می‌شود و یک نامهٔ تایپ شده که به اسم مستعارش فرستاده شده را برمی‌دارد.

    بیرون ادارهٔ پست، \noun{بلوم} نامه‌اش را باز می‌کند، ولی قبل از این که بتواند آن را بخواند با \noun{مک‌کوی}\LTRfootnote{\lr{McCoy}} برخورد می‌کند. \noun{بلوم} با \noun{مک‌کوی} کمی حرف می‌زند در حالی که سعی می‌کند بفهمد چه چیزی به نامه‌ای که اکنون درون جیبش است سنجاق شده است. در حالی که \noun{بلوم} یک زن سکسی و از طبقه بالای جامعه را در حال ردشدن از خیابان می‌بیند، \noun{مک‌کوی} درباره مرگ \noun{پدی دیگنام} که قضیه‌اش را از \noun{بانتام لاینز}\LTRfootnote{\lr{Bantam Lyons}} شنیده بود، حرافی می‌کند. \noun{بلوم} منتظر است که وقتی آن زن سوار تاکسی می‌شود، پایش را دید بزند، ولی تراموایی جلوی دیدش را می‌گیرد. \noun{بلوم} که همچنان در حال گپ زدن با \noun{مک‌کوی} است، روزنامه‌اش را باز می‌کند و یک آگهی تبلیغاتی را می‌بیند: «یک خانه چیست بدون / گوشت کنسرو پلامتری؟ / ناتمام / با آن منزلگاه سعادت». \noun{مک‌کوی} و \noun{بلوم} راجع به تور کنسرت \noun{مالی} صحبت می‌کنند (زن \noun{مک‌کوی} یک خوانندهٔ بلندپرواز است). \noun{بلوم} به نامهٔ صبح \noun{بویلان} فکر می‌کند و از صحبت دربارهٔ موضوع مدیریت \noun{بویلان} بر تور \noun{مالی} طفره می‌رود. موقع جداشدن از \noun{بلوم}، \noun{مک‌کوی} از او می‌خواهد که اسمش را در دفتر مراسم خاکسپاری \noun{دیگنام} ثبت کند. وقتی \noun{مک‌کوی} می‌رود، \noun{بلوم} به خوانندگی درجه دو و بی‌کیفیت زن \noun{مک‌کوی} فکر می‌کند.

    \noun{بلوم} یک آگهی از نمایش \noun{لیه}\LTRfootnote{\lr{Leah}} می‌بیند. \noun{بلوم} خط داستان را به یاد می‌آورد، که درباره \noun{آبراهامِ}\LTRfootnote{\lr{Abraham}} کور و درحال مرگ است که صدای پسر گمشده‌اش، \noun{ناتان}\LTRfootnote{\lr{Nathan}}، را می‌شنود. این مسأله \noun{بلوم} را به یاد مرگ پدر خودش می‌اندازد. \noun{بلوم} بالاخره نامه‌اش را درمی‌آورد - یک گل درون آن است. نامه از دوست مکاتبه‌ایِ شهوانی‌اش، \noun{مارتا کلیفورد} است. در آن نامه، آن زن خواسته است تا طرف مکاتبهٔ خود را شخصاً ببیند و به او به خاطر به کاربردن کلمهٔ خاصی در نامهٔ قبلی، صفت «کثافت» داده است و نهایتاً از او پرسیده است که زنش چه عطری می‌زند. \noun{بلوم} نامه را در جیبش می‌گذارد. او هیچ وقت قبول نخواهد کرد که آن زن را ببیند ولی برای جمله‌بندی نامه بعدی‌اش دقت بیشتری خواهد کرد. \noun{بلوم} سنجاق را از گل درون پاکت در می‌آورد و به سنجاق‌های زیاد لباس زنان فکر می‌کند. شعری به خاطرش می‌رسد:«آخ آخ، ماری سنجاق تنبونشو گم کرده...». او به اسامی \noun{مارتا} و \noun{ماری} و به نقاشی‌ای از \noun{مارتا} و \noun{مریمِ} کتاب مقدس فکر می‌کند.

    زیر طاق پل راه‌آهن، \noun{بلوم} نامه \noun{مارتا} را تکه‌پاره می‌کند. \noun{بلوم} از درِ پشتی یک کلیسا وارد می‌شود، اعلان مذهبی را می‌خواند و به تاکتیک‌های جذب بومی‌ها به مذهب می‌اندیشد. درون کلیسا، مراسمی در حال برگزاری است. \noun{بلوم} به این فکر می‌کند که کلیساها امکانِ نشستن کنار زنان جذاب را فراهم می‌آورند. او به قدرت تخدیر و تحمیقِ زبان لاتین می‌اندیشد. پشت نیمکتی می‌نشیند و به احساسات اجتماعی‌ای فکر می‌کند که باید از برگزاری مراشم عشاء ربانی نشأت بگیرد.

    او به این فکر می‌کند که \noun{مارتا} لحظه‌ای با حسی آمیخته از خشم و احترام به طرز بیان و انتخاب کلمات او نگریسته است و لحظه‌ای بعد، از او (یک مرد متأهل) درخواست ملاقات کرده است. این دوگانگی، \noun{بلوم} را به یاد \noun{کریِ}\LTRfootnote{\lr{Carey}} خائن می‌اندازد که یک زندگی مذهبی آبرومند داشته ولی در عین حال درگیر فرقه «شکست ناپذیران\LTRfootnote{\lr{Invincibles}}» شد که جنایت \noun{فینکس پارک}\LTRfootnote{\lr{Phoenix Park}} را مرتکب شدند. \noun{بلوم}، کشیش را در حال تطهیرِ جام شراب می‌بیند و در شگفت است که چرا آنها از \noun{گینس}\footnote{\lr{Guinness}-نام آبجویی ایرلندی با رنگی تیره و طعمی مانند قهوه.} یا مشروب دیگری استفاده نمی‌کنند. با دیدن گروه همسرایان در سمت چپ، \noun{بلوم} به فکر اجرای \noun{مالی} از «مادر غمگین\LTRfootnote{\lr{Stabat Mater}}» می‌افتد. وقتی کشیش، مراسم را به اتمام می‌رساند، \noun{بلوم} تأثیر نهادهای اعتراف‌گیری مذهبی و ایدهٔ اصلاح را تحسین می‌کند. مراسم تمام شده است، \noun{بلوم} بلند می‌شود تا قبل از اینکه اعانه را جمع کنند برود. \noun{بلوم} زمانش را بررسی می‌کند و به سمت \noun{سوئنی}\LTRfootnote{\lr{Sweny}} می‌رود تا لوسیونِ \noun{مالی} را سفارش دهد هر چند که نسخه را (همراه با کلیدش) در خانه و در شلوارهای هرروزه‌اش جا گذاشته است.

    نزد داروساز، \noun{بلوم} به کیمیاگری و داروهای مسکن می‌اندیشد. وقتی که داروساز دنبال دستورالعمل لوسیون می‌گردد، \noun{بلوم} به پوست دوست‌داشتنی \noun{مالی} فکر می‌کند و می‌اندیشد که آیا وقت برای حمام دارد یا نه. \noun{بلوم} یک صابون لیمو از داروساز می‌گیرد و تصمیم می‌گیرد بعداً برای گرفتن لوسیون و پرداخت پول هر دو جنس برگردد. وقتی مغازه را ترک می‌کند، \noun{بلوم} با \noun{بانتام لاینز} برخورد می‌کند. \noun{لاینز} می‌خواهد روزنامهٔ \noun{بلوم} را برای یک مسابقه اسب‌دوانی ببیند. \noun{بلوم} به \noun{لاینز} می‌گوید که می‌تواند روزنامه را برای خودش نگه دارد چرا که می‌خواهد آن را دور بیندازد. \noun{لاینز} که اشتباهاً فکر می‌کند این کار \noun{بلوم} یک اشارهٔ مخفی به اسب مسابقه‌ای است\footnote{بعداً می‌فهیم که اسم یکی از اسب‌های مسابقه \noun{ثرواوی} یا \lr{Throwaway} است که شبیه فعل \lr{Throw Away} به معنی دور انداختن است.}، روزنامه را به \noun{بلوم} برمی‌گرداند و با عجله می‌رود. \noun{بلوم} با تنفر دربارهٔ تب شرط‌بندی می‌اندیشد و به سمت حمام عمومی می‌رود. او یک آگهی تبلیغاتی بیهوده دربارهٔ ورزش دانشگاهی را نقد می‌کند. به \noun{هورنبلوئر}\LTRfootnote{\lr{Hornblower}} دربان سلام می‌کند و از قبل به لحظه‌ای فکر می‌کند که بدنش لخت خواهد شد و در وان لم خواهد داد، آلت تناسلی‌اش شل و ول و مانند گلی روی آب شناور خواهد شد.

    \chapter[هادس]{هادس\protect\footnote{\lr{Hades}-\rl{ در اساطیر یونانی، فرمانروای مردگان و دنیای زیرزمین، فرزند کرونوس و رئا است. او در قرعه‌کشی با برادرانش، بدترین سهم را برنده شد و آن جهان زیرین یا دنیای مردگان بود درصورتی که برادران او زئوس و پوزئیدون به ترتیب آسمان و دریا نصیبشان شد. از آنجایی که رعایای هادس را مردگان تشکیل می‌دادند، او به کسانی که موجب افزایش جمعیت سرزمینش می‌شدند بسیار علاقه داشت. مانند ارینی‌ها \lr{Erinnyes} یا خشم و ناامیدی، که کارشان تعقیب گناهکاران و سوق دادن آن‌ها به سمت خودکشی بود.}}}\label{ep:6}
    \noun{بلوم} بعد از \noun{مارتین کانینگهام}، \noun{جک پاور} و \noun{سایمون ددالوس‬} سوار کالسکه می‌شود - آنها دارند به مراسم خاکسپاری \noun{دیگنام} می‌روند. وقتی کالسکه شروع به حرکت می‌کند، \noun{بلوم}، \noun{استیون} را در خیابان نشان می‌دهد. \noun{سایمون}  مرددانه می‌پرسد که آیا \noun{مالیگان} هم همراه اوست یا نه. \noun{بلوم} فکر می‌کند که \noun{سایمون}  خیلی عصبانی است ولی با خود استدلال می‌کند که حق با \noun{سایمون}  است که مراقب \noun{استیون} باشد چنان که اگر \noun{رودی} هم زنده بود، \noun{بلوم} همین کار را می‌کرد.

    \noun{کانینگهام} شروع به توصیف شبی که در میخانه گذرانده است می‌کند و از \noun{ددالوس} می‌پرسد که آیا سخنرانی \noun{دن داوسون}\LTRfootnote{\lr{Dan Dawson}} را در روزنامه صبح خوانده است یا نه. \noun{بلوم} جابجا می‌شود تا روزنامه را به \noun{ددالوس} بدهد ولی \noun{ددالوس} اشاره می‌کند که خواندن آن در این زمان مناسب نیست. \noun{بلوم} اعلامیه‌های فوت را می‌قاپد و بررسی می‌کند که آیا هنوز نامهٔ \noun{مارتا} را به همراه دارد یا نه. تفکرات \noun{بلوم} خیلی زود به سمت \noun{بویلان} و ملاقات قریب‌الوقوعش با \noun{مالی} در بعدازظهر، می‌رود. در این لحظه، کالسکه از کنار \noun{بویلان} در خیابان می‌گذرد و باقی مردان از درون کالسکه به او سلام می‌دهند. \noun{بلوم} از این همزمانی آشفته می‌شود. او نمی‌تواند درک کند که \noun{مالی} و بقیه چه چیزی در \noun{بویلان} دیده‌اند. \noun{پاور} از \noun{بلوم} دربارهٔ کنسرت \noun{مالی} می‌پرسد و از او با لفظ \noun{مادام} یاد می‌کند که باعث ناراحتی \noun{بلوم} می‌شود.

    کالسکه از کنار \noun{رویبن جِی داد}\LTRfootnote{\lr{Reuben J. Dodd}ِ} نزول‌خوار می‌گذرد و مردها او را نفرین می‌کنند. \noun{کانینگهام} خاطرنشان می‌کند که همهٔ آنها به \noun{داد} بدهکارند - به غیر از \noun{بلوم} که از چهره‌اش می‌توان فهمید. \noun{بلوم} شروع می‌کند به تعریف داستان طنزی دربارهٔ اینکه چطور پسرِ \noun{داد} غرق شده است، ولی \noun{کانینگهام} گستاخانه رشته سخن را در دست می‌گیرد. مردها زود خنده‌هایشان را جمع و جور می‌کنند و با اندوه به یاد \noun{دیگنام} می‌افتند. \noun{بلوم} خاطرنشان می‌کند که او به بهترین شکل ممکن، سریع و بدون درد مرد، ولی باقی مردان در سکوت مخالفت می‌کنند - کاتولیک‌ها از مرگ ناگهانی می‌ترسند چون که شخص فرصتی برای توبه ندارد. \noun{پاور} می‌گوید که بدترین مرگ خودکشی است و \noun{ددالوس} موافقت می‌کند. \noun{کانینگهام} با علم به این که پدر \noun{بلوم} خودکشی کرده است، با لحن مداراجویانه‌ای درباره موضوع حرف می‌زند. \noun{بلوم} در باطن، حس همدردی \noun{کانینگهام} را تحسین می‌کند.

    کالسکه توقف می‌کند تا گله‌ای احشام رد شوند. \noun{بلوم} در عجب است که چرا هیچ خط ویژه‌ای مثل خطوط تراموا برای چهارپایان وجود ندارد و \noun{کانینگهام} موافقت می‌کند. \noun{بلوم} همچنین پیشنهاد ایجاد ترامواهای ویژه‌ای برای مراسم خاکسپاری را می‌دهد ولی بقیه با اکراه موافقت می‌کنند. \noun{کانینگهام} استدلال می‌کند که یک تراموا از تصادف نعش‌کش‌ها جلوگیری خواهد کرد مثل حادثه‌ای که اخیراً به پرتاب شدن تابوت وسط جاده منجر شد. \noun{بلوم} در خیال خود \noun{دیگنام} را می‌بیند که از تابوتش به بیرون پرت شده است. کالسکه از کنار یک کانال آب می‌گذرد که به \noun{مالینگار}، جایی که \noun{میلی} زندگی می‌کند منتهی می‌شود و \noun{بلوم} به فکر ملاقات با او می‌افتد. در همین حین، \noun{پاور} به خانه‌ای اشاره می‌کند که برادرکشی \noun{چایلدز}\LTRfootnote{\lr{Childs}}، یک جنایت معروف، در آن رخ داده است.

    کالسکه به مقصد می‌رسد و مردان پیاده می‌شوند. \noun{کانینگهام} که پشت سر قرار گرفته، ماجرای خودکشی پدر \noun{بلوم} را برای \noun{پاور} شرح می‌دهد. \noun{بلوم} از \noun{تام کرنان}\LTRfootnote{\lr{Tom Kernan}} می‌پرسد که آیا \noun{دیگنام} بیمه بوده یا نه. \noun{ند لمبرت} می‌گوید که \noun{کانینگهام} دارد برای بچه‌های \noun{دیگنام} اعانه جمع می‌کند. \noun{بلوم} به یکی از پسران \noun{دیگنام} با ترحم نگاه می‌کند. آنها وارد کلیسا می‌شوند و زانو می‌زنند - \noun{بلوم} آخر از همه. \noun{بلوم} به مراسم عجیب و نامأنوس می‌نگرد و به تکراری بودن شغل کشیش‌ها فکر می‌کند. مراسم پایان می‌یابد و تابوت به بیرون حمل می‌شود.

    وقتی دستهٔ عزاداران از کنار قبر \noun{مای ددالوس}\LTRfootnote{\lr{May Dedalus}} می‌گذرد، \noun{ددالوس} شروع به گریه می‌کند. \noun{بلوم} به واقعیت‌های مرگ می‌اندیشد - علی‌الخصوص، از کارافتادن اندام‌های بدن. \noun{کورنی کله‌هر}، مسئول کفن و دفن، به آنها می‌پیوندد. در جلوی دسته، \noun{جان هنری منتون} می‌پرسد که \noun{بلوم} کیست. \noun{لمبرت} توضیح می‌دهد که او شوهر \noun{مالی} است. \noun{منتون} مشتاقانه به یاد این می‌افتد که زمانی با \noun{مالی} رقصیده است و غضبناک با خود می‌اندیشد که چرا \noun{مالی} با \noun{بلوم} ازدواج کرده است.

    سرایدار گورستان، \noun{جان اُکونل}\LTRfootnote{\lr{John O'Connell}}، به مردان نزدیک می‌شود و لطیفهٔ خوبی می‌گوید. \noun{بلوم} می‌اندیشد که زنِ اُوکونل بودن، چه حسی می‌تواند داشته باشد - آیا قبرستان دیوانه‌کننده است؟ او پاکیزگی گورستان \noun{اُکونل} را تحسین می‌کند، ولی با خود فکر می‌کند که دفن کردن مرده‌ها به صورت عمودی مقرون به صرفه‌تر خواهد بود. او به قدرت حاصلخیزی اجساد مرده فکر می‌کند و به سیستمی می‌اندیشد که در آن آدم‌ها اجساد خود را برای حاصلخیز کردن باغ‌ها اهدا می‌کنند. با فکر به لطیفه‌های \noun{اُکونل}، \noun{بلوم} قبرکن‌های بذله‌گوی هملت را به خاطر می‌آورد. با این حال \noun{بلوم} با خود می‌اندیشد که نیابد در طول دو سال زمان عزاداری، دربارهٔ مرده‌ها لطیفه گفت. در پس‌زمینه، \noun{اُکونل} و \noun{کله‌هر} درباره خاکسپاری فردا مشورت می‌کنند.

    مردان دور قبر جمع می‌شوند و \noun{بلوم} از خود می‌پرسد که مرد بارانی‌پوش کیست - او سیزدهمین عضو بدیُمن گروه است و در کلیسا برای عبادت حاضر نبوده است. \noun{بلوم} به مراسم خاکسپاری خودش فکر می‌کند که مادر و پسرش هم در آن حاضرند. او به وحشت زنده به گور شدن فکر می‌کند و این که وجود تلفن در تابوت می‌تواند از این کار جلوگیری کند.

    فرد گزارشگر، \noun{هاینز}، اسم کامل \noun{بلوم} را از او می‌پرسد. \noun{بلوم} از او می‌خواهد که از \noun{مک‌کوی} هم نام برده شود، همانطور که در اپیزود \ref{ep:5} \noun{مک‌کوی} از \noun{بلوم} خواسته بود. او از \noun{بلوم} اسم مرد ناشناس بارانی‌پوش را می‌پرسد، ولی \noun{بلوم} نمی‌داند. \noun{بلوم} به کار قبرکن‌ها می‌نگرد که دارد تمام می‌شود. \noun{بلوم} در گورستان قدم می‌زند و به این فکر می‌کند که پولی که بابت قبرهای پر زرق و برق داده شده است را می‌توان به مؤسسات خیریه داد تا برای زنده‌ها هزینه شود و به این می‌اندیشد که اگر سنگ‌قبرها توضیح می‌دادند که شخص مدفون چه کسی بوده است، جالب توجه‌تر می‌شدند. او به زیارت قریب‌الوقوعِ قبر پدرش اندیشید. او یک موش می‌بیند و موشی را تجسم می‌کند که جسدی را می‌خورَد. \noun{بلوم} از ترک کردن گورستان خوشحال است چرا که داشت به مرده‌گرایی\footnote{علاقه به جسدهای مرده به طور عام و علاقه به رابطه جنسی با مرده‌ها به طور خاص.}، ارواح، جهنم و این که چطور بازدید از گورستان باعث می‌شود احساس نزدیکی به مرگ داشته باشیم، فکر می‌کرده است. سر راهش از کنار \noun{منتون} می‌گذرد و به او می‌گوید که در کلاهش فرورفتگی ایجاد شده. \noun{منتون} توجهی به او نمی‌کند.

    \chapter[آیولوس]{آیولوس\protect\footnote{\lr{Aeolus}-\rl{ در اسطوره‌های یونان، پادشاه جزیره شناور آیولیا و خدای زمینی بادها است. زئوس قدرت مهار بادها را به او داده بود و خدایی زمینی محسوب می‌شد. با مهار باد به اولیس در رسیدن به تروا کمک کرد.}}}\label{ep:7}
    اپیزود \ref{ep:7} در دفتر روزنامه \noun{فریمن}\LTRfootnote{\lr{Freeman}} می‌گذرد. عناوینِ روزنامه‌وار، این اپیزود را به عبارات کوچکتری می‌شکنند. بدون این عناوین، این اپیزود مانند اپیزودهای قبلی بازخوانی می‌شود.

    در مرکز شهر دوبلین، ترامواها، درشکه‌های پست، و بشکه‌های آبجو همزمان به سوی مقصدهایشان روانند. \noun{بلوم} در دفتر کار \noun{فریمن} است و یک کپی از آگهی خود دربارهٔ \noun{کیز}\LTRfootnote{\lr{Keyes}} برمی‌دارد. \noun{بلوم} از اتاق‌های چاپ به سمت دفاتر تلگراف می‌رود که تحت مالکیت صاحبِ \noun{فریمن} می‌باشند. او به سمت سرکارگر، \noun{نانتی}، عضو شورای شهر، می‌رود که اصالتاً ایتالیایی و شهروند ایرلند است. \noun{نانتی} دارد با \noun{هاینز} دربارهٔ گزارشش از مراسم تدفین \noun{دیگنام} صحبت می‌کند. \noun{هاینز}، سه شیلینگ به \noun{بلوم} بدهکار است و \noun{بلوم} سعی می‌کند با ظرافت و نزاکت این مسأله را به یاد او بیاورد، ولی \noun{هاینز} متوجه قضیه نمی‌شود.

    در میان سر و صدای دستگاه‌های چاپ، \noun{بلوم} طرح جدید آگهی \noun{کیز} را شرح می‌دهد: دو کلیدِ متقاطع، برای یادآوری پارلمان مستقل \noun{آیل آو من}\LTRfootnote{\lr{Isle of Man}} و درنتیجه رویای حکومت ملی ایرلند. \noun{نانتی} به \noun{بلوم} می‌گوید که یک کپی از طرح و قول سه ماه آگهی را از \noun{کیز} بگیرد. \noun{بلوم} برای لحظه‌ای به صدای کاغدها که درون دستگاه چاپ می‌روند گوش می‌دهد، سپس به سمت دفتر کارمندان می‌رود. \noun{بلوم} می‌بیند که مردان، به صورت برعکس حروفچینی می‌کنند و به پدرش فکر می‌کند که عبری را از راست به چپ می‌خواند. \noun{بلوم} وارد دفتر \noun{ایونینگ تلگراف}\LTRfootnote{\lr{Evening Telegraph}} می‌شود، که در آنجا پروفسور \noun{مک‌هیو}\LTRfootnote{\lr{MacHugh}} و \noun{سایمون ددالوس‬} دارند به \noun{ند لمبرت} گوش می‌کنند که در حال مسخره کردن سخنرانی میهن‌دوستانهٔ پرطمطراق \noun{دن داوسون} است که در روزنامه صبح مجدداً به طبع رسیده است. \noun{جی.جی اُمالوی} وارد می‌شود و دستگیرهٔ در به \noun{بلوم} می‌خورد. \noun{بلوم} به یاد گذشتهٔ \noun{اُمالوی} می‌افتد که وکیلی توانمند بود - \noun{اُمالوی} هم‌اکنون درگیر مشکلات مالی است.

    \noun{لمبرت} به تمسخر سخنرانی \noun{داوسون} ادامه می‌دهد - \noun{بلوم} با انتقاد موافق است ولی به خودش یادآوری می‌کند که چنین سخنرانی‌هایی فی‌نفسه مورد استقبال قرار می‌گیرند. \noun{کرافورد}\LTRfootnote{\lr{Crawford}} وارد می‌شود و به \noun{مک‌هیو} با نفرتی ساختگی سلام می‌کند. \noun{ددالوس} و \noun{لمبرت} برای نوشیدن می‎‌روند. \noun{بلوم} از تلفن \noun{کرافورد} برای تماس با \noun{کیز} استفاده می‌کند. \noun{لنه‌هان} با نسخهٔ ورزشی وارد می‌شود و ادعا می‌کند که \noun{سپتر} مسابقه اسبدوانی امروز را می‌برد. ما صدای \noun{بلوم} را پشت تلفن می‌شنویم - به نظر می‌رسد که او \noun{کیز} را در دفترش پیدا نکرده است. وقتی \noun{بلوم} دوباره به اتاق برمی‌گردد، به \noun{لنه‌هان} می‌خورد. \noun{بلوم} به \noun{کرافورد} می‌گوید که بیرون می‌رود تا آگهی \noun{کیز} را سروسامان دهد - \noun{کرافورد} هم به همین اندازه نگران است. لحظه‌ای بعد، \noun{مک‌هیو} از پنجره می‌بیند که پسران روزنامه‌فروش دنبال \noun{بلوم} راه افتاده‌اند و راه رفتن افتان و خیزان او را مسخره می‌کنند. \noun{لنه‌هان} هم همین کار را می‌کند.

    \noun{اُمالوی} سیگاری به \noun{مک‌هیو} تعارف می‌کند. \noun{لنه‌هان} سیگار آنها را روشن می‌کند و منتظر می‌شود تا سیگاری به او تعارف کنند. \noun{کرافورد} با \noun{مک‌هیو}، استاد لاتین، دربارهٔ امپراطوری روم شوخی می‌کند. \noun{لنه‌هان} می‌کوشد معمایی بگوید ولی کسی گوش نمی‌دهد.

    \noun{اُمدن بورک}\LTRfootnote{\lr{O'Madden Burke}} وارد می‌شود و \noun{استیون ددالوس} پشت سر اوست. \noun{استیون} نامه \noun{دیزی} را به \noun{کرافورد} می‌دهد. \noun{کرافورد}، \noun{دیزی} را می‌شناسد و دربارهٔ زن بداخلاق سابق \noun{دیزی} صحبت می‌کند که به \noun{استیون} کمک می‌کند تا دیدگاه \noun{دیزی} را که معتقد است زن‌ها مسئول گناه جهان هستند، درک کند. \noun{کرافورد} نامهٔ \noun{دیزی} را می‌قاپد و با چاپ آن موافقت می‌کند. \noun{مک‌هیو} دارد بحث می‌کند که یونانی‌ها و ایرلندی‌ها شبیه هم هستند چرا که تحت سلطهٔ فرهنگ‌های دیگرند (به ترتیب فرهنگ‌های رومی و بریتانیایی) و در عین حال معنویتی دارند که آن فرهنگ‌ها فاقد آنند. \noun{لنه‌هان} بالاخره معمای خود را می‌گوید. \noun{کرافورد} می‌گوید که استعدادهای زیادی در اتاق گرد هم آمده‌اند (ادبیات، حقوق، و غیره). \noun{مک‌هیو} اظهار می‌دارد که \noun{بلوم} نشانگر هنر تبلیغات است و \noun{اُمدن بورک} اضافه می‌کند که خانم \noun{بلوم} استعداد آواز دارد. \noun{لنه‌هان} نظری گستاخانه دربارهٔ \noun{مالی} بیان می‌دارد.

    \noun{کرافورد} از \noun{استیون} می‌خواهد که مقاله‌ای تند و تیز بنویسد. \noun{کرافورد} استعداد خارق‌العاده \noun{اایگناتیوس گالاهر}\LTRfootnote{\lr{Ignatius Gallaher}} را به خاطر می‌آورد که جنایات \noun{فینکس پارک} در سال 1882 را گزارش کرده بود (وزیر امور خارجه بریتانیا و معاونش کشته شده بودند). این تجدید خاطره، ماجراهای زیادی را دربارهٔ قتل و جنایت و گروه \noun{شکست‌ناپذیران}، که مسئولیت آنها را به عهده گرفته بود، به یاد او آورد. برخی از آنها اعدام شدند ولی بقیه زنده ماندند مانند «بز پوست کن»، شخصیتی که بعداً در اولیس ظاهر می‌شود. در همین حین، \noun{مک‌هیو} به تلفن جواب می‌دهد. تلفن از جانب \noun{بلوم} است، ولی \noun{کرافورد} آنقدر درگیر گفتگو است که نمی‌تواند با او صحبت کند.

    \noun{اُمالوی} به \noun{استیون} می‌گوید که او و پروفسور \noun{مگنیس}\LTRfootnote{\lr{Magennis}} داشتند دربارهٔ \noun{استیون} صحبت می‌کردند. آنها کنجکاوند نظر \noun{استیون} را دربارهٔ \noun{اِی.ای}، شاعر مرموز بدانند. \noun{استیون} در برابر این میل که بداند \noun{مگنیس} دربارهٔ او چه گفته است مقاومت می‌کند. \noun{مک‌هیو} صحبت آنها را قطع می‌کند تا ناب‌ترین مثال فصاحت و سخنوری را شرح دهد - سخنرانی \noun{جان اِف تیلور}\LTRfootnote{\lr{John F. Taylor}} در انجمن تاریخِ \noun{ترینیتی کالج}\LTRfootnote{\lr{Trinity College}}  دربارهٔ احیای زبان ایرلندی. \noun{مک‌هیو}، آن سخنرانی را بازسازی می‌کند که در آن بریتانیایی‌ها که از لحاظ سلطهٔ فرهنگی، تهدیدی برای ایرلندی‌ها به حساب می‌آیند با مصری‌ها مقایسه شده‌اند که تهدیدی برای نابودی فرهنگیِ کامل یهودی‌ها به شمار می‌روند.

    \noun{استیون} پیشنهاد می‌دهد که بحث را خاتمه دهند تا به میخانه بروند و \noun{لنه‌هان} پیش می‌افتد. \noun{اُمالوی}، \noun{کرافورد} را به عقب می‌کشد تا از او تقاضای قرض کند. \noun{استیون} با پروفسور \noun{مک‌هیو} بیرون می‌رود و به او تمثیلی رمزی از دو باکرهٔ پیر می‌گوید که به بالای ستون \noun{نلسون}\LTRfootnote{\lr{Nelson}} می‌روند تا مناظر دوبلین را ببینند و آلو بخورند.

    در حالی که \noun{استیون} داستانش را می‌گوید، \noun{کرافورد} بالاخره بیرون می‌آید و \noun{بلوم} که دارد وارد می‌شود سعی می‌کند تا در پله‌های جلویی او را مخاطب قرار دهد. \noun{بلوم} موافقت دوماههٔ تجدید آگهی \noun{کیز} را به جای سه ماه می‌خواهد. \noun{کرافورد} با بی‌توجهی، این پیشنهاد را رد می‌کند و به صحبتش با \noun{اُمالوی} ادامه می‌دهد. او نمی‌تواند به \noun{اُمالوی} هیچ پولی قرض بدهد.

    در جلو، داستانِ \noun{استیون} ادامه می‌یابد: زن‌ها دچار سرگیجه در بالای ستون، آلو می‌خورند و هسته‌ها را به کناره‌ها تف می‌کنند. \noun{استیون} می‌خندد - داستان به وضوح تمام شده است، ولی شنوندگان گیج شده‌اند. \noun{استیون} داستانش را «دیدِ فسقا\footnote{در عبری، نام کوهی است و به طور کلی به ارتفاع بلند و قله کوه‌ها گفته می‌شود.} از فلسطین» یا «تمثیل آلوها» می‌نامد. \noun{مک‌هیو} عامدانه و عالمانه می‌خندد. در همین حیل، ترامواها و دیگر وسائل نقلیه در کل شهر به چرخش و گردش ادامه می‌دهند.

    \chapter[لاستریگون‌ها]{لاستریگون‌ها\protect\footnote{\lr{Laestrygonians} یا \lr{Laestrygones} یا \lr{Laistrygones}-\rl{ در اساطیر یونان، قبیله‌ای از غول‌های آدمخوار. اولیس در راه بازگشت به ایتاکا با آنها مواجه شد. غول‌ها، بسیاری از مردان اولیس را خوردند و یازده کشتی از دوازه کشتی او را با پرتاب سنگ از بالای تپه نابود کردند.}}}\label{ep:8}
    \noun{بلوم} از کنار یک مغازهٔ آب‌نبات فروشی می‌گذرد. مردی به \noun{بلوم} یک کاغذپاره آگهی می‌دهد، که آمدن یک اونجلیست\LTRfootnote{\lr{Evangelist}} آمریکایی را اعلان کرده است. \noun{بلوم} ابتدا فکر می‌کند که اسم خودش روی کاغذپاره است ولی بعد می‌فهمد که نوشته شده: «خونِ برّه»\footnote{\lr{``Blood of the Lamb''}: در زبان انگلیسی املای \lr{Bloom} و \lr{Blood} فقط در یک حرف اختلاف دارد.}.

    \noun{بلوم} از کنار \noun{دیلی ددالوس} می‌گذرد. \noun{بلوم} برای \noun{ددالوس}‌های بی‌مادر ابراز تأسف و همدردی می‌کند. \noun{دیلی} لاغر به نظر می‌رسد و \noun{بلوم} به بی‌عاطفگی کلیسای کاتولیک فکر می‌کند که والدین را مجبور می‌کند فرزندانی بیشتر از آنچه که می‌توانند سیر کنند به دنیا بیاورند. \noun{بلوم} از روی پل \noun{اُکونل} عبور می‌کند و کاغذپاره را پرت می‌کند. او دو کیک \noun{بنبری}\LTRfootnote{\lr{Banbury}} می‌خرد تا به مرغان دریایی بدهد. او متوجه یک آگهی روی یک قایق پارویی در لنگرگاه می‌شود. او به دیگر مکان‌های تأثیرگذار برای آگهی‌ها فکر می‌کند مثلاً قراردادن آگهیِ یک دکتر دربارهٔ امراض مقاربتی، در دستشویی. \noun{بلوم} ناگهان به این فکر می‌کند که آیا \noun{بویلان} مبتلا به امراض مقاربتی است یا نه.

    \noun{بلوم} به یک مفهوم ستاره‌شناسی فکر می‌کند که هیچ وقت آن را درست نفهمیده بود - «پارالاکس\LTRfootnote{\lr{Parallax}}» (اختلاف منظر یا شکست نور). \noun{بلوم} به یاد بحث امروز صبح دربارهٔ «تناسخ» می‌افتد. ستونی از مردان که لباس‌های تبلیغاتی \noun{ویزدوم هلی}\LTRfootnote{\lr{Wisdom Hely}} را پوشیده‌اند می‌گذرند. وقتی \noun{بلوم} نزد \noun{هلی} کار می‌کرد، کارفرمایانش ایدهٔ تبلیغاتی او را مبنی بر این که زنانی را داخل درشکه‌های شفافی قرار دهند که از بیرون معلوم باشد دارند با لوازم‌التحریر \noun{هلی} چیز می‌نویسند، رد کرده بودند. \noun{بلوم} سعی می‌کند به خاطر بیاورد که او و مالی در آن زمان کجا زندگی می‌کردند.

    \noun{بلوم} با \noun{جوسی برین} برخورد می‌کند، که زمانی با او عشق بازی کرده بود. او الان با \noun{دنیس برین} ازدواج کرده است که از لحاظ ذهنی نامتعادل است. آقای \noun{برین} صبح امروز یک کارت پستال ناشناس دریافت کرده بود که به صورت رمزی روی آن نوشته شده بود : \footnote{ یکی از رمزهای کتاب اولیس است که معنی آن مشخص نیست ولی احتمالاً به این معنی است که آقای \noun{برین} موقع انزال، به جای منی از خود ادرار دفع می‌کند و تلویحاً به ناتوانی جنسی او اشاره دارد.}\lr{``u.p.:up.''}. امروز او در تلاش است تا علیه این شوخی اقدامات قانونی انجام دهد. \noun{بلوم} راجع به یک دوست مشترک، \noun{مینا پیورفوی}\LTRfootnote{\lr{Mina Purefoy}}، پرس و جو می‌کند که سه روز است در زایشگاه بستری می‌باشد. وقتی \noun{بلوم} و خانم \noun{برین} حرف می‌زنند، یک دوبلینیِ ابلهِ دیگر تلوتلوخوران می‌گذرد - \noun{کشل بویل اُکونور فیتس‌موریس تیسدال فارل}\LTRfootnote{\lr{Cashel Boyle O'Connor Fitzmaurice Tisdall Farrel}}.

    \noun{بلوم} به راه خود ادامه می‌دهد و از دفتر \noun{آیریش تایمز}\LTRfootnote{\lr{Irish Times}} می‌گذرد - او آگهی روزنامه‌ای را به یاد می‌آورد که برای استخدام یک بانوی تایپیست داده بود و \noun{مارتا} را جذب خود کرده بود. تقاضای کار از طرف کس دیگری هم وجود داشت - \noun{لیزی توییگ}\LTRfootnote{\lr{Lizzie Twigg}} - ولی او (\noun{لیزی})، \noun{اِی.ای} را به عنوان معرّفِ خود معرفی کرده بود و لذا در نظر \noun{بلوم} بیش از حد ادیب‌مآبانه و احتمالاً زشت جلوه کرده بود. افکارش به سمت \noun{مینا پیورفوی} و و آبستنی‌های همیشگی‌اش منحرف شد.

    با گذشتن از کنار گروهی از نیروهای پلیس، \noun{بلوم} به یاد گروهی از پلیس‌ها افتاد که دانشجویان پزشکی را که شعارهای ضدانگلیسی می‌دادند، تعقیب می‌کردند. \noun{بلوم} پیش خود فکر می‌کند که آن دانشجویان پزشکی حالا احتمالاً عضوی از همان نهادهایی شده‌اند که زمانی منتقد آن بوده‌اند. او به کسان دیگری که تغییر مسلک داده بودند فکر می‌کند - \noun{کری} از گروه \noun{شکست‌ناپذیران} و خدمتکاران خانه که کارفرمایان خود را لو می‌دهند.

    ابری روی خورشید را می‌پوشاند و \noun{بلوم} با دلتنگی می‌اندیشد که چرخه‌های زندگی - مرگ \noun{دیگنام}، زایمان خانم \noun{پیورفوی} - بی‌معنی‌اند. \noun{اِی.ای} و یک زن جوان شلخته‌پوش که احتمالاً خود \noun{لیزی توییگ} باشد از کنار \noun{بلوم} می‌گذرند.

    با عبور از کنار یک عینک‌سازی، \noun{بلوم} دوباره به پارالاکس (اختلاف منظر یا شکست نور) و کسوف و خسوف فکر می‌کند. او به عنوان آزمایش انگشت کوچکش را جلوی خورشید می‌گیرد. او به یاد شبی می‌افتد که او و \noun{مالی} با \noun{بویلان} زیر مهتاب راه می‌رفتند - او فکر می‌کند که آیا \noun{مالی} و \noun{بویلان} همدیگر را دست‌مالی می‌کردند یا نه. \noun{بلوم} از کنار \noun{باب دوران}\LTRfootnote{\lr{Bob Doran}} می‌گذرد که به وضوح در سرخوشی میگساری سالیانه خود است. \noun{بلوم} به این می‌اندیشد که مردان چطور برای تعاملات اجتماعی به الکل تکیه می‌کنند.

    \noun{بلوم} از فرط گشنگی وارد رستوران \noun{برتون}\LTRfootnote{\lr{Burton}} می‌شود، ولی فوراً از منظره مردان زیادی که وحشیانه غذا می‌خورند منزجر می‌شود. آنجا را ترک می‌کند و به سمت \noun{دیوی برن}\LTRfootnote{\lr{Davy Byrne}} می‎‌رود تا غذای سرپایی سبکی بخورد.

    \noun{بلوم} وارد \noun{دیوی برن} می‌شود و \noun{نوسی فلین}\LTRfootnote{\lr{Nosey Flynn}} از گوشه‌ای به او خوشامد می‌گوید. \noun{فلین} راجع به \noun{مالی} و تور آوازخوانی آینده‌اش می‌پرسد. \noun{فلین} به \noun{بویلان} اشاره می‌کند و \noun{بلوم} با ناراحتی به یاد ملاقات قریب‌الوقوع \noun{بویلان} با \noun{مالی} می‌افتد. \noun{فلین} دربارهٔ مسابقهٔ اسبدوانی \noun{گلدکاپ}\LTRfootnote{\lr{Gold Cup}} بحث می‌کند. \noun{بلوم} غدایش را می‌خورد و در دلش \noun{فلین} را نکوهش می‌کند.

    \noun{بلوم} روی نوشگاه به قوطی‌های غذا نگاه می‌کند. او دربارهٔ غذا اندیشه می‌کند: انواع عجیب و غریب، توت‌های سمی، غذاهای تقویت کننده قوای جنسی، و غذاهای مورد علاقه شخصی. \noun{بلوم} متوجه دو مگس می‌شود که روی قاب پنجره چسبیده‌اند. او با اشتیاق، لحظه‌ای خودمانی و صمیمی با \noun{مالی} را روی تپهٔ \noun{هاوث}\LTRfootnote{\lr{Howth}} به یاد می‌آورد: وقتی \noun{بلوم} روی او (\noun{مالی}) خوابید، \noun{مالی} تکه‌ای کیک از دهانش را در دهان \noun{بلوم} گذاشت و با هم عشقبازی کردند. با نگاه دوباره به مگس‌ها، \noun{بلوم} غمگینانه به ناهمخوانی و اختلاف بین خودش در آن زمان و در حال حاضر فکر کرد.

    خیره به نوشگاه چوبیِ باصفا، \noun{بلوم} دربارهٔ زیبایی می‌اندیشد. او زیبایی را هم‌ارز الهه‌های لمس‌ناپذیر، مثل تندیس‌های موزه ملی قرار می‌دهد. او فکر می‌کند که آیا چیزی زیر جامهٔ تندیس‌ها است یا نه و با خود عهد می‌کند تا بعداً در روز جاری، نگاهی دزدکی بیندازد. \noun{بلوم} شرابش را تمام می‌کند و به سمت حیاط پهلویی می‌رود.

    \noun{دیوی برن} دربارهٔ \noun{بلوم} کنجکاو است. \noun{فلین} شروع به شایعه‌پراکنی می‌کند: او دربارهٔ کار \noun{بلوم}، مشارکت او در کارهای فراماسونری، اینکه چقدر به ندرت مست می‌کند و امتناع او از امضای اسمش زیر هرگونه قراردادی سخن‌پراکنی می‌کند. \noun{پدی لئونارد}\LTRfootnote{\lr{Paddy Leonard}}، \noun{بانتام لاینز} و \noun{تام راچفورد}\LTRfootnote{\lr{Tom Rochford}} وارد می‌شوند و سفارش مشروب می‌دهند. آنها دربارهٔ شرط‌بندی \noun{لاینز} در مسابقات \noun{گلدکاپ} بحث می‌کنند. \noun{بلوم} دوباره وارد میکده شده و خارج می‌شود. \noun{لاینز} با پچ‌پچ می‌گوید که \noun{بلوم} به او محرمانه اطلاع داده است.

    در خیابان، \noun{بلوم} به خاطر می‌آورد که به سمت کتابخانه ملی برود تا به آگهی \noun{کیز} رسیدگی کند. \noun{بلوم} به یک مرد کور در عبور از چهارراه کمک می‌کند. \noun{بلوم} می‌اندیشد که چگونه دیگر حواس افراد نابینا مثل حس لامسه تقویت شده است. او به این فکر می‌کند که کور بودن چه حسی می‌تواند داشته باشد.

    \noun{بلوم} ناگهان \noun{بویلان} را در خیابان می‌بیند. او که دستپاچه شده، به سرعت وارد مدخل موزهٔ ملی می‌شود.

    \chapter[سیلا و کاریبد]{سیلا و کاریبد\protect\footnote{\lr{Scylla and Charybdis}-\rl{یا اسکیلا و کاریبدس یا سیلا و شاریبدیس، دو هیولای دریایی از اساطیر یونان هستند که توسط هومر مورد اشاره قرار گرفته اند. بعد از سنت‌های یونانی، محل آنها را در دو طرف تنگه مسینا و مقابل یکدیگر، در نظر گرفته‌اند. این محل بین سیسیل و سرزمین اصلی گراسیا مگنا یا همان یونان بزرگ (در جنوب ایتالیا) واقع است. گفته می‌شود که سیلا و کاریبد، در واقع در کنار یکدیگر، تهدیدی جدی و غیر قابل اجتناب در مسیر عبور ملوانان به حساب می‌آمدند؛ بدین ترتیب، کاریبد و سیلا هر دو در خود معنای جلوگیری کننده از عبور را دارند.}}}\label{ep:9}

   در دفتر مدیر کتابخانهٔ ملی، کمی بعد از ساعت یک بعدازظهر، \noun{استیون} «تئوری هملت» خود را برای  \noun{جان اگلینتونِ} منتقد و مقاله‌نویس، \noun{اِی.ایِ} شاعر و \noun{لیسترِ} کتابدار و عضو فرقهٔ \noun{کویکر}، شرح می‌دهد. \noun{استیون} ادعا می‌کند که شکسپیر، خود را با پدر هملت مرتبط کرده است نه با خودِ هملت. در آغاز این اپیزود (اپیزود \ref{ep:9})، \noun{استیون} از تکرارِ برداشت‌های غیراصیلِ مردانِ مسن‌تر، از شکسپیر، بی‌حوصله و بدخلق شده بود. \noun{جان اگلینتون}  ریشخندکنان از \noun{استیون} دربارهٔ فضائل ادبی‌اش یا فقدان آنها می‌پرسد تا او را سر جای خود بنشاند. از گوشه‌ای، \noun{اِی.ای} تئوری هملت \noun{استیون} را مورد تحقیر و اهانت قرار می‌دهد با این ادعا که نقد زندگی‌نامه‌ای بی‌فایده است چرا که باید فقط روی ژرفای بیان شده توسط هنر تمرکز داشت. \noun{استیون} به تمسخر جوانی‌اش توسط \noun{اگلینتون} پاسخ می‌دهد و خاطرنشان می‌کند که ارسطو زمانی شاگرد افلاطون بوده است. \noun{استیون} دانش خود از کار و آثار فیلسوفان را به رخ می‌کشد.

    آقای \noun{بِستِ} کتابدار، وارد می‌شود - او داشت «سرودهای عاشقانهٔ کونات\LTRfootnote{\lr{Lovesongs of Connacht}}» اثر \noun{داگلاس هاید}\LTRfootnote{\lr{Douglas Hyde}} را به \noun{هینز}  نشان می‌داد. \noun{اِی.ای} بیان می‌دارد که اشعار روستاییِ \noun{هاید} را ترجیح می‌دهد. \noun{استیون} با توصیف صحنه‌ای از لندنِ شکسپیر، به شرح تئوری‌اش ادامه می‌دهد: شکسپیر در کنار رودخانه قدم می‌زند تا به اجرای خودش از هملت برود که در آن، او نه نقش هملت را، بلکه نقش روح پدر هملت را بازی می‌کند. \noun{استیون} ادعا می‌کند که بنابراین هملت متناظر پسرِ مردهٔ شکسپیر، \noun{همنت}\LTRfootnote{\lr{Hamnet}} است و \noun{گرترود}\LTRfootnote{\lr{Gertrude}} بی‌وفا نشانگر زنِ زناکار شکسپیر، \noun{آن هاثوِی}\LTRfootnote{\lr{Ann Hathaway}} می‌باشد. \noun{اِی.ای} تکرار می‌کند که یک منتقد باید به نفسِ کار توجه کند نه به جزئیات زندگی شخصی شاعر، مثل عادات مشروب‌خوری‌اش یا قرض‌هایش. \noun{استیون} به خاطر می‌آورد که خودش مبلغی پول به \noun{اِی.ای} بدهکار است.

    \noun{اگلینتون} ادعا می‌کند که \noun{آن هاثوی} از لحاظ تاریخی مهم نیست، و از شرح‌حال‌نویسانی نقل می‌کند که ازدواج زودهنگام شکسپیر با \noun{آن هاثوی} را خطا جلوه می‌دهند - خطایی که او با رفتن به لندن جبرانش کرد. \noun{استیون} از درِ مخالفت می‌گوید که نوابغ مرتکب خطا نمی‌شوند. \noun{لیستر} دوباره وارد اتاق می‌شود. \noun{استیون} از طریق طرح و نقشهٔ نمایشنامه‌های اولیه نشان می‌دهد که \noun{آنِ} مسن‌تر، شکسپیر جوان را در \noun{استراتفورد}\LTRfootnote{\lr{Stratford}} اغوا کرده است.

    \noun{اِی.ای} بلند می‌شود که برود - او جای دیگری دعوت است. \noun{اگلینتون} می‌پرسد که آیا او امشب نزد \noun{مور}\LTRfootnote{\lr{Moore}} (داستان‌نویس ایرلندی) می‌آید یا نه - \noun{باک}  و \noun{هینز}  آنجا خواهند بود. \noun{لیستر} اشاره می‌کند که \noun{اِی.ای} در حال تدوین کتابی از شاعران جوان ایرلندی است. شخصی می‌گوید که \noun{مور}، کسی است که باید حماسهٔ ایرلند را بنویسد. \noun{استیون} از اینکه نه اسمش در مجموعه شعر است نه خودش در حلقهٔ آنان، اوقاتش تلخ است. او عهد می‌کند سرزنش و تحقیر آنان را فراموش نکند. \noun{استیون} از \noun{اِی.ای} برای چاپ نسخه‌ای از نامهٔ \noun{دیزی} تشکر می‌کند.

    \noun{اگلینتون} به بحث برمی‌گردد: او معتقد است که خودِ شکسپیر، هملت است چرا که کاراکتر شخصی آنان مثل هم است. \noun{استیون} بحث می‌کند که شکسپیر چنان نابغه‌ای بوده که می‎‌توانسته به بسیاری از کاراکترها جان بدهد. \noun{استیون} باز هم با تأکید روی زناکاریِ \noun{آن هاثوی} خاطرنشان می‌کند که نمایشنامه‌های دورهٔ میانی شکسپیر، تراژدی‌های سیاهی هستند. نمایشنامه‌های بعدی و سرزنده‌تراو (از طریق کاراکترهای زن جوان)، آمدن دختری که نوهٔ شکسپیر است و با مادربزرگش آشتی می‌کند را تصدیق می‌کنند.

    \noun{استیون} نکتهٔ دیگری را گوشزد می‌کند: روح پدر هملت به طور غیر قابل توضیحی از نحوهٔ قتل خود و خیانت زنش خبر دارد. شکسپیر بدین دلیل به او چنین فرادانشی داده که این کاراکتر، بخشی از خودِ شکسپیر است. \noun{باک}  که دم در ایستاده بود، با تمسخر \noun{استیون} را تشویق می‌کند. \noun{باک}  به \noun{استیون} نزدیک می‌شود و تلگرافی رمزی را بازگو می‌کند که \noun{استیون} به جای رفتن به میخانهٔ \noun{کشتی} برای او فرستاده بود. \noun{باک}  با خنده، \noun{استیون} را به خاطر قال گذاشتن خودش و \noun{هینز}  سرزنش می‌کند.

    یکی از نگهبانان کتابخانه به سمتِ در می‌آید و از \noun{لیستر} می‌خواهد که به یک مشتری (\noun{بلوم}) برای یافتن «مردمان کیلکنی\LTRfootnote{\lr{Kilkenny People}}» کمک کند. \noun{باک} ، \noun{بلوم} را که در سرسرا ایستاده است می‌شناسد و می‌گوید که به تازگی \noun{بلوم} را در موزهٔ ملی دیده که ماتحتِ تندیسِ الهه‌ای را دید می‌زده. با اشاره ضمنی به همجنس‌باز بودن \noun{بلوم}، \noun{باک}  سربه‌سر \noun{استیون} می‌گذارد و به او هشدار می‌دهد از \noun{بلوم} برحذر باشد.

    \noun{استیون} ادامه می‌دهد: درحالیکه شکسپیر در لندن زندگی مجلل و شرکای جنسی زیادی داشته، \noun{آن} در \noun{استراتفورد} به او خیانت کرد - این فرضیه بیان می‌دارد که چرا هیچ اشارهٔ دیگری به او در نمایشنامه‌ها نشده است. وصیتنامهٔ شکسپیر برای او فقط «تختخواب درجه دو»اش را باقی گذاشت.

    \noun{اگلینتون} اظهار می‌کند که پدر شکسپیر متناظر روح پدر هملت است. \noun{استیون} قویاً این نظریه را رد می‌کند و اصرار می‌کند که روح پدر هملت، پدر شکسپیر نیست بلکه خودِ شکسپیر است که در زمان نگارش نمایشنامه، پیر و سفیدمو شده بود. \noun{استیون} گریزی می‌زند و می‌گوید که پدرها مهم نیستند. پدر بودن قابل اثبات نیست و لذا موضوعیت ندارد - پدرها فقط از طریق یک عمل جنسیِ مختصر با فرزندان خود پیوند دارند.

    \noun{استیون} در ادامه می‌گوید که خیانتِ \noun{آن} به شکسپیر با برادران شکسپیر، \noun{ادموند}\LTRfootnote{\lr{Edmund}} و \noun{ریچارد}\LTRfootnote{\lr{Richard}} بوده که نامشان در نمایشنامه‌های شکسپیر به صورت برادران زناکار یا غاصب آمده است. \noun{اگلینتون} از \noun{استیون} می‌پرسد که آیا خودش به تئوریِ خودش اعتقاد دارد و \noun{استیون} جواب منفی می‌دهد. \noun{اگلینتون} می‌پرسد چرا باید در قبال آن تقاضای پرداخت پول داشته باشی وقتی که خودت به آن اعتقاد نداری.

    \noun{باک}  به \noun{استیون} می‌گوید که زمان خوردن یک نوشیدنی است و می‌روند. \noun{باک}  سربه‌سر \noun{اگلینتون} که مجردی تنهاست می‌گذارد. \noun{باک}  با صدای بلند نمایشنامه‌ای را می‌خواند که او داشته موقع صحبت‌های \noun{استیون} تندتند می‌نوشته - آن نمایشنامه، هزلی است با عنوان «هرمردی زنِ خودش یا ماه عسل در کف دست». همچنان که آنها از در جلویی خارج می‌شوند، \noun{استیون} احساس می‌کند کسی پشت سرش است - او \noun{بلوم} است. \noun{استیون} از \noun{باک}  عقب می‌افتد و \noun{بلوم} از بین آنها از پله‌ها پایین می‌رود. \noun{باک}  دوباره با پچ‌پچ و شوخی‌کنان به همجنس‌بازی شهوت‌آمیز \noun{بلوم} اشاره می‌کند. \noun{استیون} از پله‌ها پایین می‌رود و احساس خستگی و بی‌رمقی می‌کند.

    \chapter[صخره‌های سرگردان]{صخره‌های سرگردان\protect\footnote{\lr{Wandering Rocks} یا \lr{Planctae}-\rl{در اساطیر یونان، گروهی از صخره‌ها بودند که دریای بین آنها بیرحمانه خروشان بود.}}}\label{ep:10}
    اپیزود \ref{ep:10} از نوزده منظر کوتاه از کاراکترهای بااهمیت و کم‌اهمیت، و ماجراهای آنها در بعدازظهر دوبلین تشکیل شده است. در هر زیربخش، پارگراف‌های کوتاه و منفصلی می‌آیند که اعمال هم‌زمانی را در نقطهٔ دیگری از شهر به تصویر می‌کشند. اینها در ادامه نیامده است.

    پدر \noun{جان کانمی}\LTRfootnote{\lr{John Conmee}} از خانهٔ کشیشی خود در دوبلین به سمت مدرسه‌ای در حومهٔ شهر می‌رود تا پسرِ \noun{پاتریک دیگنام} را به رایگان بپذیرد. \noun{کانمی} به سمت ایستگاه تراموا می‌رود و سر راهش از کنار ملوانی یک‌پا، سه پسربچهٔ مدرسه‌ای و افراد دیگری عبور می‌کند. \noun{کانمی} سوار تراموایی می‌شود و پوستری از \noun{اوجین استراتون}\LTRfootnote{\lr{Eugene Stratton}}، نوازنده‌ای سیاه‌پوست توجه او را به خود جلب می‌کند و دربارهٔ کار تبلیغ مذهبی فکر می‌کند. \noun{کانمی} در جادهٔ \noun{هاوث} پیاده می‌شود، کتاب نماز و ادعیه‌اش را درمی‌آورد و همچنان که راه می‌رود از روی آن می‌خواند. در جلویش، زوج جوانی گناهکارانه از پشت پرچین ظاهر می‌شوند. \noun{کانمی} برای آنها دعا می‌کند.

    \noun{کورنی کله‌هر}، درِ تابوتی را وارسی می‌کند و سپس با مأمور پلیسی پچ‌پچ می‌کند.

    ملوان یک‌پا با چوب زیربغل، لنگ‌لنگان از خیابان \noun{اکلس}\LTRfootnote{\lr{Eccles}} گذر می‌کند، ترانه‌ای میهن‌پرستانه می‌خواند و تقاضای صدقه می‌کند. او از کنار \noun{کیتی} و \noun{بودی ددالوس} عبور می‌کند. زنی (\noun{مالی}) سکه‌ای از پنجره برای ملوان می‌اندازد.

    \noun{کیتی} و \noun{بودی ددالوس} وارد آشپزخانه می‌شوند، جایی که خواهرشان \noun{مگی} دارد لباس‌ها را می‌شوید. خواهران \noun{ددالوس} راجع به بی‌پولی و بی‌غذایی خانواده بحث می‌کنند - خواهر \noun{ماری پاتریک}\LTRfootnote{\lr{Mary Patrick}} مقداری سوپ نخود فرنگی به آنها اعانه داده است. \noun{مگی} می‌گوید که \noun{دیلی} رفته تا پدرشان، \noun{سایمون ددالوس‬} را ببیند.

    تکه‌پاره‌ای که \noun{بلوم} در اپیزود \ref{ep:8} در رودخانه انداخته بود، همچنان در رودخانه شناور است و می‌رود.

    دختر فروشنده‌ای، سبدی غذا را برای \noun{بلیزس بویلان‬} آماده می‌کند. \noun{بویلان} آدرس تحویل کالا را می‌نویسد و لباس دخترک را برانداز می‌کند. او گل سرخی برای یقه‌اش برمی‌دارد و اجازه می‌خواهد که از تلفن دختر استفاده کند.

    \noun{استیون} با استاد آوازش، \noun{آلمیدانو آرتیفونی}\LTRfootnote{\lr{Almidano Artifoni}}، در خیابانِ بیرونِ \noun{ترینیتی کالج}، ملاقات می‌کند. \noun{آرتیفونی} تلاش دارد تا \noun{استیون} را برای دنبال کردن حرفهٔ موسیقی در دوبلین ترغیب کند. \noun{استیون} تحت تأثیر تعریف از خود قرار می‌گیرد. \noun{آرتیفونی} می‌دود تا به تراموا برسد.

    دوشیزه \noun{دون}\LTRfootnote{\lr{Dunne}}، منشی \noun{بلیزس بویلان‬}، کتابی را که داشت می‌خواند کنار می‌نهد. او دربارهٔ بیرون رفتن امشبش خیالبافی می‌کند. \noun{بویلان} صدایش می‌زند. دوشیزه \noun{دون} به \noun{بویلان} می‌گوید که \noun{لنه‌هان} رأس ساعت چهار در هتل \noun{اورموند} خواهد بود.

    \noun{ند لمبرت} با \noun{جی.جی اُمالوی} و عالیجناب \noun{هیو سی. لاو}\LTRfootnote{\lr{Hugh C. Love}} ملاقات می‌کند تا صومعهٔ مریم مقدس را (که الان انبار \noun{لمبرت} است) به عالیجناب نشان دهد. \noun{لمبرت} دربارهٔ تاریخچهٔ صومعه با \noun{لاو} که در حال نوشتن کتابی تاریخی است صحبت می‌کند. \noun{لمبرت} و \noun{اُمالوی} راجع به مشکلات مالیِ \noun{اُمالوی} حرف می‌زنند.

    \noun{تام راچفورد}، اختراعش را که مکانیزمی برای پیگیری مسابقات شرط‌بندی است به \noun{نوسی فلین}، \noun{مک‌کوی} و \noun{لنه‌هان} نشان می‌دهد. \noun{لنه‌هان} قول می‌دهد که امروز بعدازظهر با \noun{بویلان} راجع به اختراع \noun{راچفورد} صحبت کند. \noun{مک‌کوی} و \noun{لنه‌هان} با هم صحنه را ترک می‌کنند. \noun{لنه‌هان} وارد دفتر شرط‌بندی می‌شود تا قیمت \noun{سپتر}، انتخابش برای مسابقات \noun{گلدکاپ} را بررسی کند. \noun{لنه‌هان} دوباره ظاهر می‌شود و به \noun{مک‌کوی} می‌گوید که \noun{بانتام لاینز} داخل است و دارد روی یک اسبِ کم‌شانس (برای بردن) شرط می‌بندد (اسبی که در اپیزود \ref{ep:5}، \noun{لاینز} فکر می‌کرد \noun{بلوم} درباره‌اش به او اطلاع محرمانه داده است). مردان، \noun{بلوم} را که دارد در آن حوالی به قفسهٔ کتابفروشی نگاه می‌کند، می‌پایند. \noun{لنه‌هان} ادعا می‌کند که زمانی \noun{مالی} را با میل خودش دستمالی کرده است. \noun{مک‌کوی} از \noun{بلوم} دفاع می‌کند چون فکر می‌کند که جنبه‌ای هنرمندانه در او وجود دارد.

    \noun{بلوم} به کتاب‌های درون قفسهٔ کتابفروشی نگاه می‌کند و «لذات گناه» را برای \noun{مالی} انتخاب می‌کند.

    در اتاق حراجیِ \noun{دیلان}\LTRfootnote{\lr{Dillon}}، فراش زنگ را به صدا درمی‌آورد. \noun{دیلی ددالوس} بیرون منتظر پدرش است. \noun{سایمون}  ظاهر می‌شود و \noun{دیلی} از او تقاضای پول می‌کند. او یک شیلینگ را که از \noun{جک پاور} قرض گرفته بود به \noun{دیلی} می‌دهد. \noun{دیلی} فکر می‌کند که او احتمالاً پول بیشتری دارد ولی \noun{سایمون}  او را تنها می‌گذارد و می‌رود.

    دسته سوار نایب‌السلطنه، گردش درون‌شهری خود را آغاز کرده است.

    \noun{تام کرنان} از مکانی که \noun{رابرت امتِ}\LTRfootnote{\lr{Robert Emmet}} میهن‌پرست اعدام شده بود می‌گذرد و به \noun{بن دالرد}  فکر می‌کند که دارد آواز «پسر مو بریده\LTRfootnote{\lr{The Croppy Boy}}» را می‌خواند. \noun{کرنان} به دسته سوار نایب‌السطنه برخورد می‌کند ولی خیلی دیر دست تکان می‌دهد.

    \noun{استیون} به جواهرات درون ویترین مغازه‌ای نگاه می‌کند و سپس قفسهٔ کتابفروشی را از نظر می‌گذراند. خواهرش \noun{دیلی} به او نزدیک می‌شود و می‌پرسد که آیا کتاب آموزش مبتدی فرانسه‌ای که خریده است خوب است یا نه. \noun{استیون} به \noun{دیلی} می‌نگرد که چشمان و ذهن تیز او را دارد ولی در وضعیت سختی در خانه گرفتار شده است. \noun{استیون} بین تمایل برای نجات \noun{دیلی} و دیگران و تمایل به فرار از آنها گیر افتاده است.

    \noun{باب کاولی}\LTRfootnote{\lr{Bob Cowley}} به \noun{سایمون ددالوس‬} خوشامد می‌گوید و آنها راجع به بدهکاری \noun{کاولی} به \noun{رویبن جی. دادِ} نزول‌خوار بحث می‌کنند. \noun{بن دالرد}  با ابلاغیه‌ای دربارهٔ بدهی \noun{کاولی} از راه می‌رسد.

    \noun{مارتین کانینگهام}، همراه با \noun{جک پاور} و \noun{جان وایس نولان}\LTRfootnote{\lr{John Wyse Nolan}} برای بچه‌های \noun{دیگنام} اعانه جمع می‌کنند. \noun{نولان} با تمسخر از کمک سخاوتمندانهٔ پنج شیلینگی \noun{بلوم} یاد می‌کند. \noun{کانینگهام}، \noun{پاور} و \noun{نولان} با \noun{جان هنری}\LTRfootnote{\lr{John Henry}}، کارمند شهرداری و \noun{جان فانینگ}\LTRfootnote{\lr{John Fanning}}، معاون کلانتر برخورد می‌کنند. دسته سوار نائب‌السلطنه از کنار آنها می‌گذرد.

    \noun{باک مالیگان‬} و \noun{هینز}  در قهوه‌فروشی‌ای می‌نشینند، جایی که برادرِ \noun{پارنل}\LTRfootnote{\lr{Parnell}} در گوشه‌ای دارد شطرنج بازی می‌کند. \noun{هینز}  و \noun{مالیگان} راجع به \noun{استیون} صحبت می‌کنند - \noun{هینز}  فکر می‌کند که \noun{استیون} از لحاظ عقلی نامتعادل است. \noun{مالیگان} هم موافق است که \noun{استیون} هیچ وقت نخواهد توانست یک شاعر واقعی شود، چرا که از تصورات کلیسای کاتولیک از جهنم، آسیب دیده است.

    \noun{تیسدال فارل} به صورت کج و معوج پشت سر \noun{آلمیدانو آرتیفونی} حرکت می‌کند و با مرد کوری برخورد می‌کند که \noun{بلوم} به او در انتهای اپیزود \ref{ep:8} کمک کرده بود.

    پسر \noun{دیگنام}، \noun{پاتریک جونیور}، استیکِ خوک به دست، به سمت خانه می‌رود. او از کنار دیگر پسربچه‌های مدرسه‌ای عبور می‌کند و با خود فکر می‌کند که آیا آنها از مرگ پدرش خبر دارند یا نه. او به تابوت پدرش فکر می‌کند که حمل شد و به آخرین باری که پدرش را دیده بود که مست بود و داشت به میخانه می‌رفت.

    حرکت دسته سواره نائب‌السلطنه (شامل \noun{ویلیام هامبل}\LTRfootnote{\lr{William Humble}}، \noun{کنت دادلی}\LTRfootnote{\lr{Earl of Dudley}} و \noun{بانو دادلی}\LTRfootnote{\lr{Lady Dudley}} و بقیه) از لژ سواره در \noun{فینکس پارک} به سمت بازار \noun{میروس}\LTRfootnote{\lr{Mirus}} ادامه یافت. این دسته از بین بسیاری از افرادی که تا به حال در این اپیزود دیدیم عبور کرد. اکثر آنها متوجه آن شدند و برخی هم برایش دست تکان دادند.

    \chapter[سیرن‌ها]{سیرن‌ها\protect\footnote{\lr{Sirens}-\rl{یا سایرن، یا حوری دریایی اساطیر یونان، گاهی به صورت موجودی با بدن یک پرنده و سر یک زن، و در سایر موارد به شکل تنها یک زن تصویر شده‌است. سیرن‌ها دختران خدای دریا فورکیس بوده‌اند، هرچند در نسخهٔ دیگری از اساطیر، پدرشان خدای نهر، آکلوس دانسته شده است. آنها آوازی بسیار زیبا و فریبنده داشتند و دریانوردان را با آوای خود گمراه کرده و به کام صخره‌های مرگ‌آوری که بر روی آن آواز میخواندند، میکشیدند. اولیس، قهرمان افسانه‌ای یونان، توانست بدون هیچ خطری از جزیره آنان بگذرد، از آنرو که طبق نصیحت سیرسه ساحره، او از همراهانش خواست تا گوشهایشان را با موم پرکرده و او را محکم به دکل کشتی ببندند تا با اغوای آنان کشتی را به بیراهه نکشاند و بی هیچ خطری بتواند آواز آنان را بشنود.}}}\label{ep:11}
    اپیزود \ref{ep:11} با پیش‌درآمدی درهم‌آمیخته از عبارات شروع می‌شود - قطعاتی که نشانگر متن بعد از خود هستند. همچنین اپیزود \ref{ep:11} از تکنیکی مشابه اپیزود \ref{ep:10} استفاده می‌کند که در آن قطعه‌هایی از متن که رویدادهای در حال وقوع در مکان‌های دیگر را توصیف می‌کنند، در روایت جاری وقفه ایجاد می‌کنند.

    دختران پیشخدمتِ نوشگاه هتل \noun{اورموند}، \noun{لیدیا دوس} و \noun{مینا کندی،} از پنجره سرک می‌کشند تا دسته سوار نائب‌السلطنه را ببینند، سپس پچ‌پچ می‌کنند و با خنده سراغ چای خود می‌روند. در این بین، \noun{بلوم} از کنار ویترین مغازه‌های دور و بر می‌گذرد.

    \noun{سایمون ددالوس‬} وارد نوشگاه \noun{اورموند} می‌شود و بعد از او \noun{لنه‌هان} وارد می‌شود و دنبال \noun{بویلان} می‌گردند. دختران پیشخدمت برای آنها نوشیدنی می‌آورند و از کوک‌کنندهٔ کورِ پیانو که اوایل امروز، پیانوی \noun{اورموند} را کوک کرده بود صحبت می‌کنند. \noun{ددالوس}، پیانو را در تالار آزمایش می‌کند. \noun{بویلان} از راه می‌رسد و با دوشیزه \noun{کندی} لاس می‌زند در حالیکه او و \noun{لنه‌هان} منتظر نتایج تلگرافیِ مسابقات \noun{گلدکاپ} هستند.

    در این حین، \noun{بلوم} در حالیکه کاغذِ نامه می‌خرد تا به \noun{مارتا} نامه بنویسد، متوجه ماشین پر زرق و برق \noun{بویلان} در \noun{اسکس بریج}\LTRfootnote{\lr{Essex Bridge}} می‌شود. اندیشناک دربارهٔ ملاقات قریب‌الوقوع \noun{بویلان} با \noun{مالی} در ساعت چهار، \noun{بلوم} تصمیم می‌گیرد تا ماشین را تا هتل \noun{اورموند} تعقیب کند. بیرون هتل، \noun{بلوم} با \noun{ریچی گولدینگ} برخورد می‌کند و موافقت می‌کند که با او داخل شوند و ناهار بخورند - \noun{بلوم} نقشه کشیده تا \noun{بویلان} را زیر نظر بگیرد. آنها در سالن غذاخوری می‌نشینند.

    \noun{بویلان} و \noun{لنه‌هان} در حال ترک صحنه و سر راهشان از کنار \noun{باب کاولی} و \noun{بن دالرد} می‌گذرند. در سالن غذاخوری، \noun{پت}\LTRfootnote{\lr{Pat}} گارسون، سفارشات نوشیدنی \noun{گولدینگ} و \noun{بلوم} را می‌آورد. \noun{بلوم} صدای ماشین \noun{بویلان} را می‌شنود که می‌رود و تقریباً با دلواپسی سکسکه می‌کند. در تالار، \noun{ددالوس} و \noun{دالرد} خاطرات کنسرت‌های آواز گذشته را و فرصتی که \noun{دالرد} داشت تا لباس شب خود را از مغازهٔ دست‌دوم‌فروشی \noun{بلوم} برای اجرا قرض بگیرد، مرور می‌کنند. مردها تحسین‌آمیز از \noun{مالی} یاد می‌کنند. در سالن غذاخوری، همچنان که \noun{پت}، غذا را سرو می‌کند، \noun{بلوم} نیز دارد دربارهٔ \noun{مالی} فکر می‌کند.

    در بین این ماجراها، صدای ماشین \noun{بویلان} می‌آید و ادامه حرکت آن به سمت \noun{بلوم} را می‌شنویم.

    \noun{بن دالرد}  آواز «عشق و جنگ» را می‌خواند و \noun{بلوم} از سالن غذاخوری متوجه آن می‌شود. او به شبی فکر می‌کند که \noun{دالرد} از مغازهٔ \noun{مالی}، لباس شب قرض گرفته بود. در تالار، \noun{ددالوس} را ترغیب می‌کنند که آواز «ماپاری\LTRfootnote{\lr{M'appari}}» که آوازی تِنور از اپرای \noun{مارتا}\footnote{این «مارتا» اسم یک اپرا است و با اسم خانم \noun{مارتا} که دوست \noun{بلوم} است مشابهت دارد.} است را بخواند.

    \noun{گولدینگ} خاطرات اجراهای اپرا را مرور می‌کند. \noun{بلوم} از روی همدردی به درد مزمنِ کمرِ \noun{گولدینگ} و غیرهمدردانه به تمایل \noun{گولدینگ} به دروغ‌گویی فکر می‌کند. در تالار، \noun{ددالوس} شروع به خواندن «ماپاری» می‌کند. \noun{گولدینگ} صدای آواز \noun{ددالوس} را تشخیص می‌دهد. \noun{بلوم} به استعداد آواز \noun{ددالوس} فکر می‌کند که با مشروب‌خواری حرام شده است. \noun{بلوم} متوجه می‌شود که این آواز از اپرای \noun{مارتا} است - یک حسن تصادف، چرا که همان موقع می‌خواست به \noun{مارتا کلیفورد} نامه بنویسد. تحت تأثیر موسیقی، \noun{بلوم} به یاد اولین قرار مهم خود با \noun{مالی} می‌افتد. آواز با تشویق حضار تمام می‌شود. \noun{تام کرنان} وارد نوشگاه می‌شود.

    \noun{بلوم} به درگیری \noun{ددالوس} و \noun{گولدینگ} فکر می‌کند. با فکر دوباره به ترانهٔ مالیخولیایی «ماپاری»، \noun{بلوم} به مرگ و خاکسپاری \noun{دیگنام} در صبح امروز فکر می‌کند. \noun{بلوم} با خودش راجع به ریاضیاتِ موسیقی و اینکه چقدر \noun{میلی} در موسیقی بی‌ذوق است فکر می‌کند.

    \noun{بلوم} شروع به نوشتن نامه‌ای به \noun{مارتا} می‌کند. او صفحه را با روزنامه‌اش می‌پوشاند و به \noun{گولدینگ} می‌گوید که دارد جواب آگهی‌ای را می‌دهد. \noun{بلوم} خطوط لاس‌زنانه‌ای می‌نویسد و یک \noun{هاف‌کراون}\footnote{\lr{half-crown}-سکه‌ای معادل دو شیلینگ و شش پنس} درون آن می‌گذارد. \noun{بلوم} از این نامه‌نگاری احساس خستگی می‌کند.

    در این لحظه صدای «ضربه»‌ای تکراری می‌آید - این صدای ضربهٔ عصای کوک‌کنندهٔ کورِ پیانو است. او برگشته تا دیاپازونش (وسیله کوک) را بردارد.

    \noun{بلوم} به لاس زدن دوشیزه \noun{دوس} در نوشگاه نگاه می‌کند. \noun{کاولی}، مینوئت\footnote{یک نوع رقص عامیانهٔ فرانسوی، با وزن سه‌ضربی آرام است.} دون ژوان را می‌نوازد. \noun{بلوم} به حضور همه‌جانبهٔ موسیقی در جهان، صدای آوازخوانی زنان، و شهوت‌آمیزی موسیقی آکوستیک فکر می‌کند. او به این فکر می‌کند که \noun{بویلان} الان رسیده است تا \noun{مالی} را ببیند. فی‌الواقع، \noun{بویلان} هم‌اکنون دارد درب خانهٔ \noun{بلوم} را می‌زند.

    \noun{تام کرنان} درخواست «پسر مو بریده» را می‌کند (یک آواز میهنی دربارهٔ عضو جوانی از شورش 1798 که توسط مردی انگلیسی که خود را به شکل کشیش اعتراف‌گیرش درآورده بود فریب داده و اعدام شد). \noun{بلوم} آماده رفتن می‌شود - \noun{گولدینگ} مأیوس می‌شود. همه برای آواز ساکت شده‌اند. \noun{بلوم} به دوشیزه \noun{دوس} نگاه می‌کند و با خود فکر می‌کند آیا او متوجه این نگاه است یا نه. \noun{بلوم} قسمتی را می‌شنود که می‌گوید پسر مو بریده آخرین نفر از نسل خود بوده و به شجره خانوادگی مقطوع‌شدهٔ خود فکر می‌کند.

    \noun{بلوم} به نگاه به دوشیزه \noun{دوس} ادامه می‌دهد که دارد دستش را دور دستگیره مخزن آبجو که شبیه آلت مردانه است می‌چرخاند. \noun{بلوم} بالاخره تکانی به خود می‌دهد. او از \noun{گولدینگ} خداحافظی می‌کند، وسائل و متعلقاتش را وارسی می‌کند و قبل از اینکه تشویق‌ها در انتهای آواز به صدا درآیند، به تالار ورودی می‌گریزد.

    \noun{بلوم} به سمت ادارهٔ پست می‌رود و از شرابی که خورده احساس نفخ می‌کند. او از قرار ملاقات ساعت پنج با \noun{کانینگهام} دربارهٔ بیمهٔ \noun{دیگنام}، پشیمان است. \noun{بلوم} از روی شک و بدبینی فکر می‌کند که پسر مو بریده باید متوجه می‌شد که آن کشیش، یک سرباز انگلیسی در لباس مبدل است.

    در \noun{اورموند}، کسی به \noun{ددالوس} می‌گوید که \noun{بلوم} این جا بوده و همین الان رفته است - آنها دربارهٔ \noun{بلوم} و استعداد آواز \noun{مالی} حرف می‌زنند. کوک‌کنندهٔ کور پیانو نهایتاً موفق می‌شود دیاپازون خود را پیدا کند.

    \noun{بلوم}، \noun{بریدی کلی}\LTRfootnote{\lr{Bridie Kelly}}، فاحشه‌ای محلی را از نظر می‌گذراند که زمانی با او دیداری داشته است. او با نگاه کردن به تصویر \noun{رابرت امت}، قهرمان ملی و کلمات معروف آخرینش در ویترین مغازه‌ای، سعی می‌کند از آن فاحشه بر حذر باشد. \noun{بلوم} آن سخنرانی را با خود می‌خواند و همزمان در سر و صدای تراموایی که نزدیک می‌شود بادی از خود رها می‌سازد.

    \chapter[سیکلوپ]{سیکلوپ\protect\footnote{\lr{Cyclops}-\rl{یکی از موجودات افسانه‌ای در اساطیر یونانی است. سیکلوپ‌ها در اسطوره‌های یونان، غول‌هایی با یک چشم در وسط پیشانی هستند. آنها قدرتمند و سرسخت بودند. حرکات آنها همیشه همراه با خشونت و قدرت بود.}}}\label{ep:12}
    یک راوی اول‌شخص بی‌نام، وقایع بعدازظهرش را شرح می‌دهد. علاوه بر روایت اول‌شخص، این اپیزود شامل بیش از سی قطعه متن به شکل نثر است و نقیضه‌ای\footnote{نقیضه که به آن «پارودی» نیز گفته می‌شود و برخی «نظیره طنزآمیز» ترجمه کرده‌اند، تقلید طنزگونه از یک کار هنری دیگر است.} است - از ره مبالغه و اغراق - بر اسطوره‌شناسی ایرلندی، زبان رسمی حقوقی، روزنامه‌نگاری، و انجیل و غیره.

    راوی با \noun{جو هاینز} در خیابان ملاقات می‌کند و توافق می‌کند که در میخانهٔ \noun{بارتی کیرنان} لبی تر کنند تا \noun{هاینز} به \noun{شهروند‬} دربارهٔ همایش تب برفکی احشام توضیح دهد. متنی به سبک حماسه‌های قدیمی سِلتی، بازاری را که از آن می‌گذرند به صورت مدینهٔ فاضله‌ای توصیف می‌کند. با رسیدن به میخانه، آنها به \noun{شهروند‬} و سگش \noun{گری‌اون}\LTRfootnote{\lr{Garryowen}} خوشامد می‌گویند. شهروند بالاخره به صورت پهلوان پنبه‌ای به تصویر کشیده می‌شود.

    \noun{آلف برگان}\LTRfootnote{\lr{Alf Bergan}} در حال خنده به \noun{دنیس برین} که دارد بیرون با زنش راه می‌رود، وارد می‌شود. \noun{برگان} ماجرای کارت پستال \lr{``U.p: up''} \noun{برین} را تعریف می‌کند و به متصدی نوشگاه سفارش یک \noun{گینس} می‌دهد. نوشیدنی به طرزی دوست‌داشتنی توصیف شده است. \noun{شهروند‬} متوجه \noun{بلوم} می‌شود که بیرون دارد قدم می‌زند و با دشمنی و کینه‌توزی به این فکر می‌کند که او دارد چه کار می‌کند - او از \noun{بلوم} به فراماسونر یاد می‌کند.

    صحبت به \noun{پدی دیگنام} می‌کشد. جلسه احضار روحی که در آن روح \noun{دیگنام} ظاهر شده بود شرح داده می‌شود. \noun{باب دوران} (کاراکتری از کتاب دوبلینی‌ها) با صدای بلند به بیرحمی و قساوت خداوند که \noun{دیگنام} را از آنها گرفته بود طعنه می‌زند. راوی با بیزاری می‌گوید که \noun{دوران} در میگساری سالیانهٔ خود به سر می‌برد.

    \noun{بلوم} وارد می‌شود - او قرار است \noun{مارتین کانینگهام} را ببیند. \noun{هاینز} تلاش می‌کند مشروبی برای \noun{بلوم} بخرد، ولی \noun{بلوم} مؤدبانه تقاضایش را رد می‌کند. مسأله اعدام‌ها پیش می‌آید و \noun{بلوم} فضل‌فروشانه راجع به مجازات اعدام صحبت می‌کند. \noun{شهروند‬} رشته سخن را در دست می‌گیرد و اعدام میهن‌پرستان ایرلندی را یادآوری می‌کند. راوی به \noun{بلوم} می‌نگرد و تحقیرآمیز به \noun{مالی} فکر می‌کند - راوی از طریق \noun{پیسر بورک} که ارتباطاتی با خانواده \noun{بلوم} دارد، به اندازه کافی از \noun{بلوم}ها می‌داند. \noun{بلوم} سعی می‌کند مطلب ظریفی راجع به اعدام‌ها بگوید، ولی \noun{شهروند‬} با احساسات کوته‌نظرانهٔ ناسیونالیستی حرفش را قطع می‌کند. متنی به نثر روزنامه‌نگاری، صحنه عمومی اعدام شهیدی را توصیف می‌کند.

    \noun{هاینز} مشروب دیگری سفارش می‌دهد. راوی از اینکه \noun{بلوم} نه مشروب می‌خورد نه برای بقیه مشروب می‌خرد، اوقاتش تلخ است. \noun{بلوم} توضیح می‌دهد که ملاقات او با \noun{کانینگهام} به خاطر دیدن خانم \noun{دیگنام} است. \noun{بلوم} سپس راجع به پیچیدگی‌های مسائل بیمه‌ای سخن می‌گوید.

    مردها مختصراً راجع به \noun{نانتی} صحبت می‌کنند که قرار است شهردار شود، و \noun{شهروند‬}، \noun{نانتی} را متهم می‌کند که اصلیتی ایتالیایی دارد. صحبت به ورزش می‌کشد: \noun{هاینز} به نقش \noun{شهروند‬} به عنوان احیاگر ورزش‌های \noun{گیلیک}\LTRfootnote{\lr{Gaelic}} گریزی می‌زند. \noun{برگان} به مسابقهٔ بوکس اخیری اشاره می‌کند که \noun{بویلان} از آن متنفع شده است. \noun{بلوم} راجع به تنیس صحبت می‌کند در حالیکه بقیه دارند راجع به \noun{بویلان} حرف می‌زنند. متنی ورزشی به سبک روزنامه‌نگاری مسابقهٔ بوکسی بین یک ایرلندی و یک انگلیسی را شرح می‌دهد. \noun{برگان} توجه بقیه را به تور قریب‌الوقوع کنسرت \noun{بویلان} و \noun{مالی} جلب می‌کند. \noun{بلوم} سرد و غیرصمیمی شده است و راوی حدس می‌زند که \noun{بویلان} در حال عشقبازی با \noun{مالی} است.

    \noun{جی.جی اُمالوی} و \noun{ند لمبرت} وارد می‌شوند. صحبت به دیوانگی \noun{دنیس برین} کشیده می‌شود - \noun{بلوم} به رنج خانم \noun{برین} فکر می‌کند ولی هیچ کس احساس همدردی نمی‌کند. \noun{شهروند‬} که درگیر بحثی دربارهٔ مشکلات ایرلند است، در حالیکه نگاهش متوجه \noun{بلوم} است، شروع به اظهارنظرهای ضدسامی و بیگانه‌هراسانه می‌کند. \noun{بلوم} به او اعتنایی نمی‌کند.

    \noun{جان وایس نولان} و \noun{لنه‌هان} وارد می‌شوند. \noun{لنه‌هان} با راوی دربارهٔ مسابقات \noun{گلدکاپ} صحبت می‌کند. \noun{ثرواوی}، اسب دیگری برنده شده است - \noun{لنه‌هان}، \noun{بویلان} و «رفیقهٔ» \noun{بویلان} روی \noun{سپتر} باخته‌اند. \noun{شهروند‬} به صحبت دربارهٔ استثمار ایرلند ادامه می‌دهد - او آرزوی روزی را دارد که ایرلند بتواند پاسخ بدی‌هایی را که انگلیس به زور علیه آن انجام داده است بدهد.

    \noun{بلوم} مخالف است و می‌گوید که آزار و شکنجه، کینهٔ میهن‌پرستانه را کهنه می‌کند. \noun{نولان} و \noun{شهروند‬} از \noun{بلوم} درباره ملیت خودش می‌پرسند. \noun{بلوم} می‌گوید که ایرلندی به دنیا آمده است و اصالتی یهودی دارد. \noun{نولان} اعتقاد دارد که یهودیان خوب از خودشان دفاع نکرده‌اند. \noun{بلوم} پاسخ می‌دهد که عشق و زندگی گزینه‌های بهتری نسبت به زور و کینه هستند. \noun{بلوم} می‌رود تا \noun{کانینگهام} را بیابد. \noun{شهروند‬} حرف \noun{بلوم} دربارهٔ عشق را مسخره می‌کند.

   \noun{لنه‌هان} به بقیه می‌گوید که \noun{بلوم} احتمالاً رفته تا پولی را که در شرط‌بندی روی \noun{ثرواوی} برده است، نقد کند (برای این سؤتفاهم به پاراگراف آخر اپیزود \ref{ep:5} مراجعه نمایید). راوی به حیاط پهلویی می‌رود و به طرز اهانت‌آوری به خست و لئامت \noun{بلوم} فکر می‌کند. او به داخل باز می‌گردد و می‌بیند که همه دارند پشت سر \noun{بلوم} غیبت می‌کنند.

    \noun{کانینگهام}، \noun{پاور} و \noun{کرافتون}\LTRfootnote{\lr{Crofton}} از راه می‌رسند. متنی به سبک رنسانس، خوشامدگویی‌ها و احوالپرسی‌ها را شرح می‌دهد. \noun{کانینگهام} سراغ \noun{بلوم} را می‌گیرد و تازه از راه رسیده‌ها سریعاً وارد ماجرای غیبت پشت سر \noun{بلوم} می‌شوند. \noun{کانینگهام} اصالت مجارستانیِ \noun{بلوم} و نام خانوادگی اصلی او را که \noun{ویراگ}\LTRfootnote{\lr{Virag}} است فاش می‌کند. \noun{شهروند‬} به طرز کنایه‌آمیزی می‌گوید که \noun{بلوم} مسیحای جدید ایرلند است. او به مسخره می‌گوید که بچه‌های \noun{بلوم}، مال خودش نیستند سپس به زنانگی \noun{بلوم} اشاره می‌کند. \noun{کانینگهام} برای \noun{بلوم} تقاضای صدقه می‌کند و به سلامتی جمع می‌نوشد. متنی درباره مراسم برکت در ادامه داستان می‌آید.

    \noun{بلوم}، نفس‌زنان دوباره وارد میخانه می‌شود و می‌بیند که \noun{کانینگهام} آمده است. \noun{کانینگهام} که احساس می‌کند جوّ متخاصمی شکل گرفته است، \noun{بلوم}، \noun{پاور} و \noun{کرافتون} را به بیرون و به سمت ماشین‌هایشان همراهی می‌کند. \noun{شهروند‬} دنبال آنها می‌رود و و دربارهٔ یهودی بودن \noun{بلوم} با صدای بلند عربده می‌کشد. راوی از این معرکه‌گیری \noun{شهروند‬} ابراز انزجار می‌کند. \noun{بلوم} که \noun{پاور} جلویش را گرفته است، از یهودیانِ معروف نام می‌برد و در آخر، اسم عیسی مسیح را می‌آورد. \noun{شهروند‬} یک قوطی بیسکویت برداشته و به سمت ماشین پرتاب می‌کند. متنی طولانی، شرحی اغراق‌آمیز از ضربه و برخورد قوطی می‌دهد. متنی به سبک کتاب مقدس، \noun{بلوم} را به صورت \noun{الیاس}\LTRfootnote{\lr{Elijah}} به تصویر می‌کشد که با ارابه‌ای به آسمان عروج می‌کند.

    \chapter[ناوسیکائا]{ناوسیکائا\protect\footnote{\lr{Nausicaa}-\rl{در اسطوره‌های یونان، دختر آلکینوئوس است. زمانی که اولیس به جزیرهٔ سخریا رسید او عاشق اولیس شد و از پدر خواست اجازه دهد با وی ازدواج کند. اما اولیس که قصد بازگشت به سرزمین خود را داشت نپذیرفت.}}}\label{ep:13}
    یک راوی سوم‌شخصِ کلیشه‌ای و با احساسات زننده، غروب تابستان را در ساحل \noun{سندی‌مونت}\LTRfootnote{\lr{Sandymount}}، نزدیک \noun{ماری}، کلیسای \noun{ستارهٔ دریا}، شرح می‌دهد. \noun{بلوم} در ساحل، نزدیک سه دوستِ دختر - \noun{سیسی کافری}، \noun{ادی بوردمن} و \noun{گرتی مک‌داول‬} - و همراهان آنها: برادران دوقلوی نوپای \noun{سیسی} و برادر کودکِ \noun{ادی}، ایستاده است. \noun{سیسی} و \noun{ادی} مراقب بچه‌ها هستند و گاه و بیگاه سر به سر \noun{گرتی} که کمی دورتر نشسته است می‌گذارند. راوی از روی همدردی، \noun{گرتی} را زیبا توصیف می‌کند و محصولات تجاری‌ای را که برای سر و وضعش به کار می‌برد توصیف می‌کند. عشق \noun{گرتی} - پسری که جلوی خانه‌شان دوچرخه‌سواری می‌کرد - اخیراً رفته بود. \noun{گرتی} رویای ازدواج و یک زندگی خانوادگی با مردی آرام و قوی را در سر می‌پروراند. در این حین، \noun{ادی} و \noun{سیسی} با صدای بلند، دعوای بچه‌ها را آرام می‌کنند. \noun{گرتی} از وقاحتِ دور از شأنِ دوستانش علی‌الخصوص در حضور آقای متشخصی که آنجاست (\noun{بلوم}) حرص می‌خورد. در آن حوالی، در کلیسای \noun{ستارهٔ دریا}، مراسم ریاضتِ گروهی از مردان، با تضرع به درگاه مریم باکره آغاز می‌شود.

    کودکان نوپا توپشان را خیلی دور شوت می‌کنند. \noun{بلوم} آن را برداشته و برایشان پرتاب می‌کند - توپ آنقدر می‌چرخد تا زیر دامن \noun{گرتی} متوقف می‌شود. \noun{گرتی} سعی می‌کند تا توپ را به سمت \noun{سیسی} شوت کند ولی موفق نمی‌شود. \noun{گرتی}، سنگینی نگاه \noun{بلوم} را روی خودش حس می‌کند و متوجه صورت غمگین او می‌شود. او در رویایش تصور می‌کند که آن مرد، غریبه‌ای سوگوار است که به دلداری او نیاز دارد. \noun{گرتی} قوزک پا و موهایش را به \noun{بلوم} نشان می‌دهد با علم به این که دارد او را تحریک می‌کند.

   \noun{گرتی} با صدای بلند می‌گوید که خیلی دیر شده است، به این امید که \noun{سیسی} و \noun{ادی}، بچه‌ها را به خانه ببرند. \noun{سیسی} به \noun{بلوم} نزدیک می‌شود و از او ساعت می‌پرسد. ساعت \noun{بلوم} از کار افتاده است. \noun{گرتی} می‌بیند که \noun{بلوم} دوباره دستانش در جیب‌هایش می‌گذارد و احساس می‌کند که دارد پریود می‌شود. او مشتاقِ دانستن ماجرای \noun{بلوم} است - آیا ازدواج کرده؟ زنش مرده؟یا از روی وظیفه و اجبار با زنی دیوانه زندگی می‌کند؟

    وقتی آتش‌بازی بازار \noun{میروس} شروع می‌شود، \noun{سیسی} و بقیه آماده رفتن می‌شوند. آنها می‌دوند تا آتش‌بازی را تماشا کنند ولی \noun{گرتی} می‌ماند. \noun{گرتی} تکیه می‌دهد، زانوانش را در بغل می‌گیرد و عمداً پاهایش را نشان می‌دهد در حالیکه آتش‌بازیِ یک «آفتاب‌مهتاب\footnote{ وسیله آتش‌بازی که به شکل استوانه است.} دراز» را در دوردست آسمان تماشا می‌کند. در اوج اپیزود و احساسات \noun{گرتی} (و اوج ارگاسم \noun{بلوم}، که به زودی می‌فهمیم) آفتاب‌مهتاب در هوا منفجر شده و به فریادهای «اوه! اوه!» در زمین منجر می‌شود.

    وقتی که \noun{گرتی} بلند می‌شود و شروع به رفتن به سمت بقیه می‌کند \noun{بلوم} می‌فهمد که یک پای او لنگ است. او شوکه شده و احساس ترحم می‌کند، سپس با این فکر خود را تسکین می‌دهد که وقتی آن دختر داشت تحریکش می‌کرد، او از این مسأله خبر نداشت. \noun{بلوم} به جذابیت‌های جنسی معلولین و سپس به میل جنسی زیادشوندهٔ زنان در دوران پریودشان فکر می‌کند. با به خاطر آوردن دو دوستِ \noun{گرتی}، او به دوستی رقابت‌آمیز زنانه فکر می‌کند مثل دوستی \noun{مالی} با \noun{جوسی برین}. \noun{بلوم} به یاد می‌آورد که ساعتش در 4:30 از کار افتاده بود و با خود فکر می‌کند که آیا این وقتی بود که \noun{مالی} و \noun{بویلان} در حال سکس بودند یا نه.

    \noun{بلوم} لباسِ آغشته به منی خود را مرتب می‌کند و به راهکارهای اغوای زنان فکر می‌کند. \noun{بلوم} با خود می‌اندیشد که آیا \noun{گرتی} متوجه استمنا کردن او شده یا نه -حدس می‌زند که او متوجه شده، چرا که زنان خیلی تیز هستند. او لحظه‌ای به این فکر می‌کند که آیا \noun{گرتی} همان \noun{مارتا کلیفورد} است یا نه. \noun{بلوم} به این فکر می‌کند که دخترها چقدر زود مادر می‌شوند سپس به خانم \noun{پیورفوی} که در زایشگاه مجاور است می‌اندیشد. \noun{بلوم} به «مغناطیسی» فکر می‌کند که وقتی \noun{بویلان} و \noun{مالی} با هم بودند، باعث از کار افتادن ساعتش شد، احتمالاً همان مغناطیسی که مردان و زنان را جذب یکدیگر می‌کند.

    \noun{بلوم} عطر \noun{گرتی} را در هوا استشمام می‌کند - عطری ارزان برخلاف رایحه \noun{مرّ شیرینِ}\LTRfootnote{\lr{Opoponax}} \noun{مالی}. \noun{بلوم} داخل جلیقه خود را بو می‌کند و به این فکر می‌کند که بوی یک مرد چگونه است. بوی صابون لیمو به خاطرش می‌اورد که فراموش کرده لوسیونِ \noun{مالی} را بگیرد.

    «نجیب‌زاده‌ای» از کنار \noun{بلوم} می‌گذرد. \noun{بلوم} به آن مرد فکر می‌کند و از نظر می‌گذراند که داستانی تحت عنوان «مرد مرموز در ساحل» بنویسد. این فکر، او را به یاد مرد بارانی‌پوش در مراسم خاکسپاری \noun{دیگنام} می‌اندازد. با نگاه به فانوس دریایی \noun{هاوث}، \noun{بلوم} به علم نور و رنگ و سپس به روزی که او و \noun{مالی} در آنجا گذراندند فکر می‌کند. حالا \noun{بویلان} با او (\noun{مالی}) است. \noun{بلوم} احساس می‌کند رودست خورده است. او متوجه می‌شود که انگار مراسم تمام شده است. پستچی با چراغی، گشتِ ساعت نهِ خود را انجام می‌دهد. پسر روزنامه‌فروشی، نتایج مسابقات \noun{گلدکاپ} را فریاد می‌زند.

    \noun{بلوم} تصمیم می‌گیرد که فعلاً به خانه نرود. او دوباره به واقعه \noun{بارنی کیرنان} فکر می‌کند - شاید آن \noun{شهروند‬} نمی‌خواست آزاری برساند. \noun{بلوم} به ملاقات عصرش با خانم \noun{دیگنام} فکر می‌کند. \noun{بلوم} سعی می‌کند تا رویای دیشبش را به خاطر آورد. \noun{مالی} تنبان ترکی و کفش‌های راحتی قرمز پوشیده بود.

    \noun{بلوم} تکه‌ای کاغذ و سپس تکه‌چوبی برمی‌دارد. با این فکر که آیا \noun{گرتی} فردا برمی‌گردد یا نه، روی شن‌ها، شروع به نوشتن پیغامی برای او می‌کند - \lr{``I AM A''} - ولی چون جای کافی وجود ندارد دست می‌کشد. او حرف‌ها را پاک می‌کند و تکه چوب را پرت می‎‌کند که مستقیماً در شن‌ها فرود می‌آید. او تصمیم می‌گیرد چرت کوتاهی بزند و افکارش به واسطه خواب، درهم و برهم می‌شوند. همزمان با صدای ساعت دیواریِ صومعه مجاور، \noun{بلوم} به خواب می‌رود.

    \chapter[گلهٔ گاوِ خورشید]{گلهٔ گاوِ خورشید\protect\footnote{\lr{Oxen of the Sun} یا \lr{Cattle of Helios}-\rl{در اساطیر یونان، در جزیرهٔ تریناکیا می‌چریدند که معادل سیسیل امروزین است. هلیوس یا خدای خورشید، هفت گله گاو و هفت رمه گوسفند داشت.}}}\label{ep:14}
    تکنیک روایت اپیزود \ref{ep:14} به منظور نمایش روند تکامل زبان انگلیسی است. سبک نثر‌های دوره‌های مختلفِ زمانی همراه با سبک مشهورترین نویسنده‌های آن دوران، بازسازی شد و گاه به گاه هم به ترتیب زمانی از آنها نقیضه‌سازی (پارودی) شده است.

    نثر لاتینی و سپس سلسله کلماتی با صدای متشابه\LTRfootnote{\lr{Alliteration}} به نثر انگلوساکسون، ما را در زایشگاه خیابان \noun{هولس}\LTRfootnote{\lr{Holles}} قرار می‌دهد که توسط \noun{سِر اندرو هورن}\LTRfootnote{\lr{Sir Andrew Horne}} اداره می‌شود. \noun{بلوم} به دروازهٔ بیمارستان می‌رسد. او آمده تا خانم \noun{پیورفوی} را ببیند. پرستار \noun{کالان}\LTRfootnote{\lr{Callan}}، یکی از آشنایان \noun{بلوم}، دروازه را باز کرده و او را به داخل هدایت می‌کند. گفتگوی آنها دربارهٔ خانم \noun{پیورفوی} که به مدت سه روز بستری شده است، با نثر اخلاق‌گرایانهٔ قرون وسطی بیان می‌شود. ورود ناگهانی \noun{دیکسون}\LTRfootnote{\lr{Dixon}}، دانشجوی پزشکی از یک اتاق شلوغ در انتهای سرسرا، به سبک رومانسِ قرون وسطایی توصیف می‌شود. \noun{دیکسون} که زمانی \noun{بلوم} را برای نیش زنبور مداوا کرده بود، \noun{بلوم} را به داخل دعوت می‌کند، جایی که \noun{لنه‌هان}، \noun{کراترز}\LTRfootnote{\lr{Crotthers}}، \noun{استیون}، \noun{پانچ کاستلو}\LTRfootnote{\lr{Punch Costello}} و دانشجویان پزشکی: \noun{لینچ} و \noun{مدن}\LTRfootnote{\lr{Madden}} با سر و صدای زیاد دور بساط ساردین و آبجو نشسته‌اند. \noun{دیکسون} برای \noun{بلوم} آبجویی می‌ریزد، که \noun{بلوم} بی سروصدا آن را داخل لیوان شخص بغل دستی خالی می‌کند. راهبه‌ای دم در آمده و از آنها می‌خواهد که ساکت باشند.

    مردها دربارهٔ پرونده‌های پزشکی بحث می‌کنند که در آنها دکتر باید بین نجات مادر یا بچه یکی را انتخاب کند - \noun{استیون} راجع به جنبه‌های مذهبی این سوال حرف می‌زند در حالیکه بقیه دارند دربارهٔ جلوگیری از بارداری و سکس شوخی می‌کنند. \noun{بلوم} غمگین است و به خانم \noun{پیورفوی} و زایمان \noun{رودی} توسط \noun{مالی} فکر می‌کند. \noun{بلوم}، \noun{استیون} را از نظر می‌گذراند و تصور می‌کند که او چقدر دارد وقتش را با این مردها هدر می‌دهد.

    آبجو ریختن‌های پی در پی \noun{استیون} و گوشه و کنایه‌هایی که دربارهٔ بارداری مریم مقدس زده می‌شود به نثر دوره الیزابت توصیف می‌شود. \noun{پانچ کاستلو} با آوازی وقیحانه دربارهٔ زنی آبستن، حرف بقیه را قطع می‌کند. پرستار \noun{کویگلی}\LTRfootnote{\lr{Quigley}} دم در آمده و آنها را به سکوت دعوت می‌کند. سر به سر گذاشتن مردها دربارهٔ تقوی و پرهیزگاری دوران جوانی \noun{استیون}، به نثر اوایل قرن هفدهم شرح داده می‌شود. صدای رعد و برقی به گوش می‌رسد. \noun{بلوم} متوجه می‌شود که \noun{استیون} واقعاً از این آیتِ خشم خدا ترسیده است و با توصیف علمی پدیده رعد و برق سعی در آرام کردن \noun{استیون} دارد.

    ملاقات \noun{باک مالیگان‬} با \noun{الک بانون} در خیابان مجاور، به سبک خاطره‌نویسی قرن هفدهم توصیف می‌شود. \noun{الک} با \noun{باک}  دربارهٔ دختری (\noun{میلی بلوم}) صحبت می‌کند که در \noun{مالینگار} با او قرار و مدار می‌گذارد. سپس هر دو مرد با هم به بیمارستان خیابان \noun{هولس} می‌روند.

    شخصیت‌های بدرد نخورِ \noun{لنه‌هان} و \noun{کاستلو} به سبک نثر \noun{دانیل دوفوئه}\LTRfootnote{\lr{Daniel Defoe}} توصیف می‌شوند. موضوع نامهٔ \noun{دیزی} و سلامتی احشام پیش می‌آید. در پی آن، یک لطیفهٔ تمثیلی و کنایی دربارهٔ فتاوای پاپ، \noun{هنری هشتم} و روابط انگلیس با ایرلند می‌آید. از راه رسیدن \noun{باک}  به سبک مقاله‌نویسی \noun{اَدیسون}\LTRfootnote{\lr{Addison}} و \noun{استیل}\LTRfootnote{\lr{Steele}} توصیف می‌شود. \noun{باک}  راجع به شغل جدیدش به عنوان «باردار کننده» همه زن‌های بعد از این شوخی می‌کند. گفتگویی جانبی بین \noun{کراترز} و \noun{بانون} دربارهٔ \noun{میلی} و قصد \noun{بانون} برای خرید لوازم ضدبارداری از دوبلین به سبک \noun{لاورنس استرن}\LTRfootnote{\lr{Lawrence Sterne}} توصیف می‌شود. مردها در لفافه راجع به روش‌های مختلف ضدبارداری بحث می‌کنند.

    سپس نثر قرن هجدهم \noun{الیور گلداسمیت}\LTRfootnote{\lr{Oliver Goldsmith}} می‌آید. پرستار \noun{کالان}، \noun{دیکسون} را فرا می‌خواند: خانم \noun{پیورفوی} پسری زاییده است. مردها با هرزگی دربارهٔ پرستار \noun{کالان} صحبت می‌کنند. نثر سیاسی قرن هجدهم برای توصیف آسودگی خاطر \noun{بلوم} از شنیدن خبر زایمان خانم \noun{پیورفوی} و انزجارش از رفتار مردانِ جوان به کار رفته است. سبک طنزآمیز \noun{جونیوس}\LTRfootnote{\lr{Junius}}، نگاه ریاکارانه و حق به جانب \noun{بلوم} نسب به دانشجویان پزشکی را به پرسش می‌گیرد.

    از نثر \noun{ادوارد گیبون}\LTRfootnote{\lr{Edward Gibbon}} برای توصیف گفتگوی مردها دربارهٔ مسائل مختلفِ مرتبط با تولد استفاده شده است: بخش‌های سزارین، پدرانی که قبل از زایمان همسرانشان می‌میرند، پرونده‌های برادرکشی (شامل پرونده جنایت \noun{چایلدز}، که در اپیزود \ref{ep:6} از آن صحبت شد)، لقاح مصنوعی، یائسگی، تجاوزِ منجر به آبستنی، ماه‌گرفتگی‌های مادرزادی و دوقلوهای به هم چسبیده. از نثر گوتیک برای توصیف داستان روح و اجنه‌ای که \noun{باک}  تعریف می‌کند استفاده شده است.

    از سبک احساساتی \noun{چارلز لمب}\LTRfootnote{\lr{Charles Lamb}} برای توصیف یادآوری \noun{بلوم} از جوانی‌اش، و سپس احساس پدرانه‌ای که نسبت به مردان جوان در خود حس می‌کند، استفاده شده است. سبک مبهم و توهمی \noun{تامس دوکینسی}\LTRfootnote{\lr{Thomas DeQuincey}}، حالت بدبینانه‌ای که تفکرات \noun{بلوم} ناگهان به خود می‌گیرند را شرح می‌دهد. از سبک نثر \noun{سوج لندور}\LTRfootnote{\lr{Savage Landor}} برای توصیف اینکه چگونه \noun{لنه‌هان} و \noun{لینچ} از طریق به پیش کشیدن موضوع شاعریِ بی‌حاصلِ \noun{استیون} و مادرِ مرده‌اش، او را می‌رنجانند، استفاده شده است. صحبت به مسابقات \noun{گلدکاپ} و سپس به \noun{کیتی}، دوست‌دختر \noun{لینچ} کشیده می‌شود؛ ما می‌فهمیم که \noun{لینچ} و \noun{کیتی}، آن زوجی بودند که بعدازظهر امروز، پدر \noun{کانمی} آنها را دیده بود (در اپیزود \ref{ep:10}).

    در ادامه، سبک تاریخی و ناتورالیستی قرن نوزدهم را داریم. گفتگو به علل اسرارآمیز مرگ و میر کودکان می‌کشد. سبک احساساتی \noun{چارلز دیکنز}\LTRfootnote{\lr{Charles Dickens}} برای توصیف خانم \noun{پیورفوی}، مادرِ خوشحال، به کار می‌رود.

    از سبک نثر مذهبی \noun{کاردینال نیومن}\LTRfootnote{\lr{Cardinal Newman}} برای توصیف اینکه چگونه گناهان گذشته می‌توانند انسان را تسخیر کنند، استفاده شده است. سبک زیبایی‌شناسانهٔ \noun{والتر پاتر}\LTRfootnote{\lr{Walter Pater}} در ادامه می‌آید. \noun{بلوم} به کلمات و عبارات پرخاشگرانهٔ \noun{استیون} راجع به مادران و کودکان می‌اندیشد. \noun{بلوم} به یاد می‌آورد که \noun{استیون} را در کودکی دیده که نگاه‌های سرزنش‌آمیزی به مادرش می‌کرده است. از سبک \noun{جان راسکین}\LTRfootnote{\lr{John Ruskin}} برای توصیف پیشنهاد فوریِ \noun{استیون} برای رفتن به میخانهٔ \noun{بورک} استفاده شده است. \noun{دیکسون} به آنها می‌پیوندد. \noun{بلوم} عقب می‌افتد و از پرستار \noun{کالان} می‌خواهد جملهٔ محبت‌آمیزی به خانم \noun{پیورفوی} بگوید. از سبک \noun{تامس کارلایل}\LTRfootnote{\lr{Thomas Carlyle}} برای نشان دادن نیرومندی خانم \noun{پیورفوی} استفاده شده است.

    همچنان که مردان با عجله به میخانهٔ \noun{بورک} می‌روند، روایت داستان به تصویر آشفته‌ای از لهجه‌ها و اصطلاحات مختلف قرن بیستم تبدیل می‌شود. اول، \noun{استیون} برای همه مشروب می‌خرد. بحث مسابقات \noun{گلدکاپ} پیش می‌آید. \noun{استیون} دوباره برای همه \noun{ابسنث}\footnote{یا افسنطین یا خاراگوش نوعی نوشیدنی تقطیری با درصد بالای الکل می‌باشد که از تقطیر گیاه انیسون به دست می‌آید.} می‌خرد و \noun{الک بانون} بالاخره می‌فهمد که \noun{بلوم}، پدر \noun{میلی} است و با حالتی عصبی در می‌رود. متصدی نوشگاه، زمان را اعلام می‌کند و کسی راجع به مرد بارانی‌پوشی که در گوشه‌ای است، پچ‌پچ می‌کند. متصدی نوشگاه آنها را بیرون می‌اندازد و همزمان در خیابان، مأمورین آتش‌نشانی دارند می‌رود تا آتشی را خاموش کنند. کسی استفراغ می‌کند. \noun{استیون}، \noun{لینچ} را قانع می‌کند تا با او به محلهٔ فاحشه‌ها بروند. پوستری در آن حوالی که آمدن وزیری را اعلان می‌کند (همان آگهی که \noun{بلوم} در اپیزود \ref{ep:8} دریافت کرده بود)، الهام‌بخش آخرین گریز به سبک بازاری اونجلیسم آمریکایی است.

    \chapter[کیرکه]{کیرکه\protect\footnote{\lr{Circe}-\rl{در اسطوره‌های یونان، دختر هلیوس و پرسه است. جادوگری قدرتمند بود که دشمنانش را به حیوان تبدیل می‌کرد. پیکوس، سکولا و همراهان اولیس از جمله کسانی بودند که مورد خشم او قرار گرفتند.}}}\label{ep:15}
    اپیزود \ref{ep:15} به شکل نمایشنامه همراه با توضیحات کارگردانی و نام شخصیت‌های داستان در بالای دیالوگشان نوشته شده است. بخش عمدهٔ اپیزود \ref{ep:15} به شکل توهمات مستانه، نیمه هوشیار و همراه با تشویش رخ می‌دهد.

    نزدیک ورودی \noun{نایت‌تاون}\LTRfootnote{\lr{Nighttown}}، محلهٔ بدنام دوبلین، \noun{استیون} و \noun{لینچ} به سمت فاحشه‌خانه‌ای آشنا می‌روند. فوکوس روی \noun{بلوم} که در آن حوالی است می‌رود. \noun{بلوم}، \noun{استیون} و \noun{لینچ} را تا \noun{نایت‌تاون} تعقیب کرده ولی آنها را گم کرده است. او داخل مغازهٔ گوشت‌خوک‌فروشی می‌رود تا غذایی آخرشبی بخورد. \noun{بلوم} ناگهان نسبت به خرج و مخارج احساس گناه می‌کند و توهمی آغاز می‌شود که در آن والدین \noun{بلوم}، \noun{مالی} و \noun{گرتی مک‌داول‬} توهین‌های مختلفی را نثار \noun{بلوم} می‌کنند. سپس خانم \noun{برین} ظاهر می‌شود - او و \noun{بلوم} معاشقات قدیمی خود را تازه می‌کنند.

    در کنجی تاریک، \noun{بلوم} گوشتی را که خریده است به سگی گرسنه می‌دهد - این عملِ شک برانگیز، توهمی دیگر را سبب می‌شود که در آن، دو نگهبانِ شب از \noun{بلوم} که گناهکارانه پاسخ می‌دهد، بازجویی می‌کنند. عن‌قریب، \noun{بلوم} به اتهام دیوثی و بی‌غیرتی، آشوب‌طلبی، جعل، تعدد زوجات و جاکشی، در حال محاکمه علنی است. شاهدانی مانند \noun{مایلس کرافورد}\LTRfootnote{\lr{Myles Crawford}}، \noun{فیلیپ بوفوی} و \noun{پدی دیگنام} به شکل سگ ظاهر می‌شوند. \noun{ماری دریسکول}\LTRfootnote{\lr{Mary Driscoll}}، کلفت قبلی خانوادهٔ \noun{بلوم} شهادت می‌دهد که \noun{بلوم} زمانی به قصد سکس به او نزدیک شده بود.

    این صحنهٔ کابوسناک با نزدیک شدن \noun{زو هیگینزِ} فاحشه به \noun{بلوم} تمام می‌شود. \noun{زو} فکر می‌کند که \noun{بلوم} و \noun{استیون} که هر دو عزادارند، با هم هستند. او به \noun{بلوم} می‌گوید که \noun{استیون}، داخل است. \noun{زو} به شوخی، سیب‌زمینی \noun{بلوم} را از جیبش می‌دزدد سپس به \noun{بلوم} که او را دربارهٔ مضرات سیگار کشیدن نصحیت کرده بود، کنایه می‌زند. رویا و خیال دیگری شروع می‌شود که در آن نصیحت \noun{بلوم} درباره سیگار نکشیدن به یک سخنرانی انتخاباتی تبدیل می‌شود. سپس \noun{بلوم}، تحت محافظت ایرلندی‌ها و صهیونیست‌ها، به پادشاه «بلومشلیم\footnote{این را معادل \lr{Bloomusalem} گذاشته‌ام که به نظرم ترکیب «بلوم» \lr{(Bloom)} و «اورشلیم» \lr{(Jerusalem)} است.}» جدید تبدیل می‌شود. توهم میهن‌پرستانه با متهم کردن \noun{بلوم} به آزاداندیشی و بی‌بندوباری، تلخ می‌شود - \noun{باک مالیگان‬} پیش آمده و دربارهٔ ناهنجاری‌های جنسی \noun{بلوم} شهادت می‌دهد، سپس \noun{بلوم} را زن اعلام می‌کند. \noun{بلوم} هشت بچه به دنیا می‌آورد.

    این توهمات با ورود دوبارهٔ \noun{زو} پایان می‌یابد. از آخرین باری که حرف زده بود فقط یک «لحظهٔ واقعی» گذشته است. \noun{زو}، \noun{بلوم} را به داخل فاحشه‌خانهٔ \noun{بلا کوهن} هدایت می‌کند، جایی که \noun{استیون} و \noun{لینچ} با \noun{کیتی} و \noun{فلوریِ} فاحشه، گرم گرفته‌اند. \noun{استیون} مشغول فضل‌فروشی و نواختن پیانو است. \noun{فلوری} منظور \noun{استیون} را درست نمی‌فهمد و فکر می‌کند که او دارد کتاب مکاشفات یوحنا را موعظه می‌کند. \noun{استیون}، توهمی را دربارهٔ وعظ و موعظه آغاز می‌کند. توهمی دیگر توسط \noun{بلوم} آغاز می‌شود که در آن \noun{لیپوتی ویراگ}\LTRfootnote{\lr{Lipoti Virag}}، پدربزرگ \noun{بلوم} از راه می‌رسد و \noun{بلوم} را دربارهٔ سکس نصیحت می‌کند.

    وقتی خودِ \noun{بلا کوهن} وارد اتاق می‌شود، یک توهم طولانی آغاز می‌شود - \noun{بلا} به «بلو\LTRfootnote{\lr{Bello}}» تبدیل می‌شود، که برای رام کردن و آزار \noun{بلومِ} زن‌شده پیش می‌آید و به او دربارهٔ گناهان گذشته‌اش و مردانگی \noun{بویلان} طعنه می‌زند. \noun{بلو} معتقد است که خانهٔ \noun{بلوم} بدون او جای بهتر خواهد بود و \noun{بلوم} می‌میرد. توهم ادامه می‌یابد - احتمالاٌ در «دنیای پس از مرگ»ِ \noun{بلوم} - و در آن یک حوری (همان که تصویرش بالای تخت \noun{بلوم} بود) \noun{بلوم} را به خاطر فساد اخلاقی سرزنش می‌کند. این افسون وقتی پایان می‌یابد که \noun{بلوم}، حوری را با تمایلات جنسی خودش (خودِ حوری) مواجه می‌کند.

    \noun{بلوم} در می‌یابد که \noun{بلا کوهن} جلوی او ایستاده است - باز هم به نظر می‌رسد که در دنیای واقعی فقط چند لحظه از ورود او گذشته است. \noun{بلوم} سیب‌زمینی‌اش را از \noun{زو} پس می‌گیرد. \noun{بلا} از مردها می‌خواهد که پولشان را بدهند و \noun{استیون} بیشتر از میزان لازم پرداخت کرده و پول هر سه نفر را حساب می‌کند. \noun{بلوم} مقداری از پول خودش را برمی‌دارد و اضافه پرداخت \noun{استیون} را به او برمی‌گرداند، سپس از آنجا که \noun{استیون} مست است، مدیریت کل پول \noun{استیون} را برای شب به عهده می‌گیرد.

    \noun{زو} کف دست \noun{بلوم} را می‌خواند و او را «شوهری زن‌ذلیل» خطاب می‌کند. توهم دیگری آغاز می‌شود که در آن \noun{بلوم} دارد به صحنهٔ سکس \noun{بویلان} و \noun{مالی} نگاه می‌کند. گفتگو به ماجراهای پاریسیِ \noun{استیون} می‌کشد و \noun{استیون} با اشتیاق، داستان فرار از دست دشمنانش و پدرش را بازگو می‌کند.

    \noun{زو}، پیانولا را راه می‌اندازد و همه به جز \noun{بلوم} می‌رقصند. \noun{استیون} تندتر و تندتر می‌چرخد و نزدیک است که بیفتد. جسد متلاشی‌شده مادرش از زمین برمی‌خیزد. \noun{استیون} می‌ترسد و و غمگین و نادم است - او از روح مادرش می‌خواهد تأیید کند که او مسبب مرگش نبوده است. روح جوابی نمی‌دهد و از رحم و غضب خداوند سخن می‌گوید. بقیه متوجه گیجی و منگی \noun{استیون} می‌شوند و \noun{بلوم} پنجره را باز می‌کند. \noun{استیون} لجوجانه سعی در راندن روح و برطرف کردن ندامتش دارد و فریاد می‌زند که جلوی هر کسی که بخواهد قلبش را بشکند می‌ایستد. \noun{استیون} چوبدستی‌اش را به چلچراغ می‌کوبد. \noun{بلا}، پلیس را فرامی‌خواند و \noun{استیون} از در فرار می‌کند. \noun{بلوم} به سرعت بلا را آرام می‌کند و در پی \noun{استیون} می‌رود.

    \noun{بلوم} به \noun{استیون} می‌رسد که در محاصرهٔ جمعیتی قرار گرفته است و دارد برای سرباز \noun{کار}\LTRfootnote{\lr{Private Carr}} از ارتش انگلیس، دربارهٔ حضور ناخواندهٔ نظامی انگلیس در ایرلند رجزخوانی می‌کند. \noun{استیون} از قصد شخصی‌اش برای براندازی ذهنیِ کشیش و شاه سخن می‌گوید. \noun{بلوم} سعی در دخالت دارد. \noun{کار} که احساس می‌کند به شاهش توهین شده است، \noun{استیون} را تهدید به کتک و مجازات می‌کند. \noun{ادوارد هفتم}، \noun{شهروند‬}، پسر موبریده و «ننه‌بزرگ بی‌دندون پیر\LTRfootnote{\lr{Old Gummy Granny}}»، به عنوان تجسم‌های ایرلند، دعوا را تشویق می‌کنند با این حال \noun{استیون} از خشونت پرهیز می‌کند.

    \noun{لینچ} از روی بی‌طاقتی صحنه را ترک می‌کند. \noun{استیون}، \noun{لینچ} را «یهودا»ی خائن می‌نامد. \noun{کار}، \noun{استیون} را می‌زند. پلیس از راه می‌رسد. \noun{بلوم} متوجه \noun{کورنی کله‌هر} می‌شود که نزدیک مأمور پلیس است و برای پسرِ \noun{سایمون}  پادرمیانی می‌کند. \noun{کله‌هر} پلیس را راضی می‌کند و می‌رود. تنها در خیابان، \noun{بلوم} نزدیکِ \noun{استیون} تقریباً بیهوش می‌رود و روحِ \noun{رودی}، پسر \noun{بلوم} ظاهر می‌شود.

    \chapter[ائومایوس]{ائومایوس\protect\footnote{\lr{Eumaeus}-\rl{خوک‌چران اولیس. خدمتکار اولیس بود و او را در راه رسیدن به همسرش پنلوپه یاری داد. در اصل شاهزاده بود ولی کنیزی او را ربود و به لائرتس، پدر اولیس فروخت.}}}\label{ep:16}
    \noun{بلوم}، \noun{استیون} را بلند می‌کند و او را به استراحتگاه رانندگان تاکسی که در آن حوالی قرار دارد می‌برد تا غذایی بخورد. سر راه، \noun{بلوم}، \noun{استیون} را دربارهٔ خطرات \noun{نایت‌تاون} و شرابخواری با «رفقایی» که آدم را تنها می‌گذارند، نصیحت می‌کند. \noun{استیون} ساکت است. مردها از کنار \noun{گاملی}\LTRfootnote{\lr{Gumley}}، یکی از دوستان پدر \noun{استیون} عبور می‌کنند. در ادامهٔ مسیر، \noun{استیون} با \noun{کورلی}\LTRfootnote{\lr{Corley}}، یکی از آشنایان مفلسِ خود، مواجه می‌شود. \noun{استیون}، نیمه جدی- نیمه شوخی به \noun{کورلی} توصیه می‌کند تا برای موقعیت شغلی \noun{استیون} در مدرسهٔ \noun{دیزی} که به زودی خالی خواهد بود، درخواست کار دهد، سپس به او یک سکهٔ \noun{هاف‌کراون} می‌دهد. \noun{بلوم} از دست و دلبازی \noun{استیون} نگران است. همچنان که پیش می‌روند، \noun{بلوم} به \noun{استیون} یادآوری می‌کند حالا که \noun{باک}  و \noun{هینز}  او را ترک کرده‌اند، امشب جایی برای خوابیدن ندارد. \noun{بلوم}، خانهٔ پدر \noun{استیون} را پیشنهاد می‌کند و به \noun{استیون} قوت قلب می‌دهد که \noun{سایمون}  به او افتخار می‌کند. \noun{استیون} ساکت است و منظرهٔ افسردهٔ خانه را به یاد می‌آورد. \noun{بلوم} با خود فکر می‌کند که آیا در انتقاد کردن از \noun{باک}  تندروی کرده یا نه.

    \noun{بلوم} و \noun{استیون} وارد استراحتگاه رانندگان تاکسی می‌شود که شایعه شده نگهبان آن \noun{فیتس‌هریسِ}\LTRfootnote{\lr{Fitzharris}} «بز پوست‌کن» است، راننده تاکسی‌ای که در جریان جنایت \noun{فینکس پارک} دررفته بود. \noun{بلوم} سفارش قهوه و غذایی برای \noun{استیون} می‌دهد. ملوانی مو قرمز از \noun{استیون} نامش را می‌پرسد و سپس از او سوال می‌کند که آیا \noun{سایمون ددالوس‬} را می‌شناسد. \noun{بلوم} از جواب نامشخص \noun{استیون} گیج شده است. وقتی ملوان شروع به تعریف حکایت‌هایی محیرالعقول از \noun{سایمون ددالوس‬} می‌کند، \noun{بلوم} با خود فکر می‌کند که این فقط یک حسن تصادف است.

    ملوان، خود را \noun{دی.بی مورفی}\LTRfootnote{\lr{D.B. Murphy}} معرفی و شروع به تعریف ماجراهای سیر و سفرش می‌کند. او کارت پستالی از زنی ایلیاتی را دست به دست می‌کند. \noun{بلوم} با ظن و بدگمانی متوجه می‌شود که نام گیرنده، \noun{مورفی} نیست. حکایت‌های ملوان، \noun{بلوم} را به یاد نقشه‌های غیربلندپروازانهٔ خودش و بازار بکرِ سفرهای ازران‌قیمت برای آدم‌های معمولی می‌اندازد.

    ملوان می‌گوید که شخصی ایتالیایی را دیده که چاقویی در پشت مردی فرو کرده است. موقع اشاره به چاقو، کسی موضوع جنایات \noun{فینکس پارک} را پیش می‌کشد. وقتی مشتریان به جنایات پارک فکر می‌کنند، سکوتی حکمفرما می‌شود و با بدگمانی به نگهبان آنجا نگاه می‌کنند. \noun{مورفی}، خالکوبی‌هایش را نشان می‌دهد: یک لنگر، شمارهٔ 16، و نیمرخی از \noun{آنتونیو}\LTRfootnote{\lr{Antonio}}، دوستی که توسط کوسه‌ها خورده شد.

    \noun{بلوم} متوجه \noun{بریدی کلی} می‌شود که بیرون ایستاده و با خجالت سرک می‌کشد. وقتی که می‌رود، \noun{بلوم}، \noun{استیون} را در مورد فاحشه‌های منتقل کنندهٔ امراض نصیحت می‌کند. \noun{استیون} موضوع بحث را از قاچاق سکس به قاچاق روح می‌کشاند. بحثی گیج‌کننده آغاز می‌شود - \noun{بلوم} دربارهٔ سلول‌های خاکستری مغز و \noun{استیون} راجع به منازعات خداشناسانه دربارهٔ روح صحبت می‌کند.

    \noun{بلوم}، \noun{استیون} را وادار می‌کند چیزی بخورد و موضوع صحبت را به داستان ملون راجع به چاقوکش ایتالیایی برمی‌گرداند. \noun{بلوم} قبول دارد که مدیترانه‌ای‌ها زود جوش می‌آورند و می‌گوید که زنش نیمه اسپانیایی است. در این حین، مردان دیگر راجع به کشتیرانی ایرلند صحبت می‌کنند - نگهبان اصرار دارد که انگلیس، ثروت ایرلند را بالا می‌کشد. \noun{بلوم} معتقد است که قطع روابط با انگلستان احمقانه است ولی عاقلانه ساکت می‌ماند و چیزی نمی‌گوید. او برای \noun{استیون} صحنهٔ مشابهی را که با \noun{شهروند‬} داشته و جواب تند و تیزی را که راجع به یهودی بودنِ مسیح داده، شرح می‌دهد. با این حال \noun{بلوم} به \noun{استیون} اطمینان می‌دهد که واقعاً یهودی نیست. \noun{بلوم}، پاتک خود را به میهن‌پرستیِ خشونت‌آمیز \noun{شهروند‬} تشریح می‌کند: جامعه‌ای که در آن همه کار می‌کنند و در قبال آن درآمدی مکفی دارند. \noun{استیون} علاقه‌ای نشان نمی‌دهد و \noun{بلوم} تصریح می‌کند که در ایرلندِ \noun{بلوم}، کار ادبی هم شغل محسوب می‌شود. \noun{استیون}، نقشهٔ \noun{بلوم} را که به عقیدهٔ \noun{استیون} حقیر به نظر می‌رسد، مسخره می‌کند - \noun{استیون} با تکبر بحث را بر عکس می‌کند و می‌گوید که ایرلند مهم است چون «آن» (ایرلند) به« او» (\noun{استیون}) تعلق دارد نه برعکس.

    \noun{بلوم} بی‌ادبی و رفتار نامتعادل \noun{استیون} را با در نظر گرفتن مست بودن و زندگی شخصی دشواری که دارد، می‌بخشد. \noun{بلوم} دوباره به تقدیر و مشیت دیدارشان در آنجا فکر می‌کند و به این می‌اندیشد که نمایشنامه‌ای با عنوان «تجربیات من در استراحتگاه رانندگان تاکسی» برای مجلهٔ \noun{تیت‌بیتس} بنویسد. نگاه \noun{بلوم} به روزنامهٔ \noun{ایونینگ تلگراف} متوجه می‌شود که مطلبی راجع به برنده شدن \noun{ثرواوی} در \noun{گلدکاپ} و مطلب دیگری راجع به خاکسپاری \noun{دیگنام} نوشته شده و در آن اسم \noun{استیون} و «بارانی‌پوش» به عنوان حاضرین آمده و اسم خودش اشتباهاً به صورت \noun{ل. بوم} چاپ شده است. \noun{استیون} دنبال نوشتهٔ \noun{دیزی} می‌گردد.

    گفتگوها در استراحتگاه به \noun{پارنل} می‌کشد و اینکه او احتمالاً نمرده بلکه فقط تبعید شده است. \noun{بلوم} به زمانی فکر می‌کند که در شلوغی و ازدحامی، کلاه پرت شده \noun{پارنل} را به او برگردانده بود. \noun{بلوم} به موضوع بازگشت گمشده و فرد متقلبی که خود را گمشده جا زده فکر می‌کند. در این حین، نگهبان با خشونت، \noun{کیتی اُشی}\LTRfootnote{\lr{Kitty O’Shea}} - رفیقهٔ محصنهٔ \noun{پارنل} - را برای سقوط و تباهی \noun{پارنل} سرزنش می‌کند. \noun{بلوم} با \noun{اُشی} و \noun{پارنل} احساس همدردی می‌کند - شوهرِ \noun{کیتی اُشی} آشکارا بدرد نخور بود.

    \noun{بلوم} به \noun{استیون}، تصویری از \noun{مالی} را نشان می‌دهد. \noun{بلوم} درخفا آرزو دارد که \noun{استیون} عادت خانم‌بازی خود را ترک کند و سر و سامانی به زندگی‌اش بدهد. \noun{بلوم} خود را مثل \noun{استیون} می‌بیند و ایده‌های سوسیالیستی دوران جوانی‌اش را به خاطر می‌آورد. \noun{بلوم} که در سرش نقشه‌های زیادی برای هر دو نفرشان دارد، \noun{استیون} را برای فنجانی کاکائوی داغ به خانه‌اش دعوت می‌کند. \noun{بلوم} پول غذایی را که \noun{استیون} نخورد پرداخت می‌کند و بازوان \noun{استیون} را می‌گیرد چرا که \noun{استیون} هنوز بیحال به نظر می‌رسد. آنها به سمت خانه راه می‌افتند و دربارهٔ موسیقی، غاصبین قدرت و حوری‌های دریایی گپ می‌زنند. \noun{استیون}، ترانه‎‌ای غمگین برای \noun{بلوم} می‌خواند و \noun{بلوم} به این فکر می‌کند که این استعداد آوازِ \noun{استیون} چه موفقیت اقتصادی‌ای می‌توانست برای او به ارمغان آورد. این اپیزود از زاویه دید رفتگری که دو مرد را بازو به بازو در دل شب می‌بیند که با هم می‌روند، تمام می‌شود.

    \chapter[ایتاکا]{ایتاکا\protect\footnote{\lr{Ithaca}-\rl{شهری که اولیس در آن زندگی می‌کرد.}}}\label{ep:17}
    اپیزود \ref{ep:17} به صورت سوم‌شخص روایت می‌شود و مشتمل بر 309 سوال، و جواب‌های مفصل و اسلوب‌مند آنها به سبک سوال و جواب دینی یا مکالمات سقراطی است.

    \noun{بلوم} و \noun{استیون} به سمت خانه می‌روند و دربارهٔ موسیقی و سیاست گپ می‌زنند. وقتی به خانه می‌رسند، \noun{بلوم} با ناراحتی درمی‌یابد که کلیدهایش را فراموش کرده است. او از روی نرده‌ها می‌پرد، وارد آشپزخانه می‌شود و جلوی دروازه می‌آید تا \noun{استیون} را به داخل راه دهد. در آشپزخانه، \noun{بلوم}، کتری را روشن می‌کند. \noun{استیون}، پیشنهاد \noun{بلوم} را برای شستشو رد می‌کند چرا که دچار آب‌هراسی است. محتویات آشپزخانه \noun{بلوم} مرور می‌شود منجمله چیزهایی که حضور قبلی \noun{بویلان} در خانه را آشکار می‌سازند - سبد هدیه و بلیط‌های شرط‌بندی. بلیط‌های شرط‌بندی، \noun{بلوم} را به یاد \noun{گلدکاپ} می‌اندازد و سوتفاهمش با \noun{بانتام لاینز} (در اپیزود \ref{ep:5}) برایش آشکار می‌شود.

    \noun{بلوم} برای خود و \noun{استیون}، کاکائوی داغ می‌ریزد و آنها در سکوت آن را می‌نوشند. \noun{بلوم} به \noun{استیونِ} متفکر نگاه می‌کند و تمایلات شاعریِ جوانی خودش را به یاد می‌آورد. راوی فاش می‌کند که \noun{بلوم} و \noun{استیون} قبلاً دوبار همدیگر را دیده‌اند - یکبار وقتی \noun{استیون} پنج سالش بود و بار دیگر وقتی ده ساله بود. در دفعهٔ دوم، \noun{استیون}، \noun{بلوم} را به نهار در منزل \noun{ددالوس} دعوت کرده بود و \noun{بلوم} مؤدبانه تقاضایش را رد کرده بود. سرگذشت شخصی و خلق و خوی آنها مقایسه می‌شود - خلق و خوی \noun{استیون} هنری است در حالیکه خلق و خوی \noun{بلوم} به علوم کاربردی، اختراعات و تبلیغات متمایل است.

    دو مرد برای هم حکایت تعریف می‌کنند و \noun{بلوم} امکان انتشار مجموعه داستان‌های \noun{استیون} را بررسی می‌کند. آنها برای هم از زبان ایرلندی و عبری نقل قول می‌کنند و می‌نویسند. \noun{استیون}، گذشته را در \noun{بلوم} و \noun{بلوم} آینده را در \noun{استیون} می‌بیند. \noun{استیون} می‌خواهد داستان ضدسامی قرون وسطاییِ «هری هیوز کوچک\LTRfootnote{\lr{Little Harry Hughes}}» را بخواند که در آن یک پسر مسیحی سرش را توسط دخترِ فردی یهودی از دست می‌دهد. نحوهٔ بیان \noun{استیون} از داستان این را تداعی می‌کند که او خودش و \noun{بلوم} را به صورت پسر مسیحی کوچک داستان در نظر گرفته است. ولی \noun{بلوم} احساسات متضادی دارد و ناگهان به، میلیسنت، «دختر فردی یهودی» فکر می‌کند. \noun{بلوم} لحظه‌هایی از کودکی \noun{میلی} را به خاطر می‌آورد و با فکر وصلت بالقوهٔ \noun{استیون} و \noun{میلی} (یا \noun{مالی})، از \noun{استیون} می‌خواهد که شب را بماند. \noun{استیون}، حق‌شناسانه دعوت او را رد می‌کند. \noun{بلوم}، پول \noun{استیون} را به او برمی‌گرداند که روی هم یک پنی می‌شود و پیشنهاد می‌دهد در آینده تعاملات بیشتری با هم داشته باشند. \noun{استیون} مردد به نظر می‌رسد و \noun{بلوم} بدبین می‌شود. به نظر می‌رسد \noun{استیون} در حس افسردگی \noun{بلوم} شریک است.

    \noun{بلوم}، \noun{استیون} را به بیرون راهنمایی می‌کند و آنها در حال نگاه به آسمان شب که شهاب ثاقبی ناگهان در آن ظاهر می‌شود، با هم در حیاط می‌شاشند. \noun{بلوم}، \noun{استیون} را بدرقه می‌کند و آنها با هم دست می‌دهند و در این حین ناقوس کلیسا به صدا در می‌آید. \noun{بلوم} به صدای قدم‌های \noun{استیون} گوش می‌دهد و احساس تنهایی می‌کند.

    \noun{بلوم} به داخل باز می‌گردد. با ورود به اتاق نشیمن، سرش به اسباب و اثاثیه‌ای می‌خورد که جابجا شده‌اند. او می‌نشیند و شروع به درآوردن لباس‌هایش می‌کند. محتویات اتاق و هزینهٔ \noun{بلوم} برای کل روز (به غیر از پولی که به \noun{بلا کوهن} داده بود) فهرست می‌شوند. آرزوی \noun{بلوم} برای داشتن خانهٔ ییلاقی ساده‌ای در حومهٔ شهر، شرح داده می‌شود. \noun{بلوم}، نامهٔ \noun{مارتا} را در کشوی قفل‌دار گنجه‌اش می‌گذارد و با لذت به تعاملات خوشایندی که امروز با خانم \noun{برین}، پرستار \noun{کالان} و \noun{گرتی مک‌داول‬} داشت فکر می‌کند. محتویات کشوی دوم شامل اسناد مختلف خانوادگی منجمله یادداشت خودکشی پدر \noun{بلوم} است. \noun{بلوم} احساس غمگینی و ندامت می‌کند بیشتر بدین خاطر که او از اعتقادات و رویه‌های پدرش مثل تهیه غذای حلال طبق سنت یهود، پیروی نکرده است. \noun{بلوم} از پدرش برای میراثی که برای او بجا گذشته و او را از بدبختی نجات داده، سپاسگزار است - در اینجا \noun{بلوم} رویای خودِ تحقق‌نیافتهٔ سرگردانش را در سر می‌پروراند که در کل جهان می‌گردد و ستاره‌ها راهنمای راهش هستند.

    خیالات \noun{بلوم} تمام می‌شود و او به سمت اتاق خوابش می‌رود و به کارهای کرده و نکردهٔ امروزش فکر می‌کند. با ورود به اتاق خواب، \noun{بلوم} متوجه شواهد بیشتری از حضور \noun{بویلان} می‌شود. ذهن \noun{بلوم}، سرسری به سمت فهرست فرضیِ بیست و پنج عاشق قبلی \noun{مالی} که \noun{بویلان} فقط آخرین آنهاست می‌رود. \noun{بلوم}، ابتدا از روی حسادت و سپس از روی تسلیم به \noun{بویلان} فکر می‌کند.

    \noun{بلوم}، کفلِ \noun{مالی} را که نزدیک صورتش است می‌بوسد، چرا که او (\noun{بلوم}) با صورت در پای تخت دراز کشیده است. \noun{مالی} بیدار می‌شود و \noun{بلوم} با دروغ و حذف موارد بسیار، داستان امروزش را برای او تعریف می‌کند. او به \noun{مالی} راجع به \noun{استیون} می‌گوید و او را به صورت استاد دانشگاه و نویسنده معرفی می‌کند. \noun{مالی} تلویحاً آگاه است  که او و \noun{بلوم} ده سال است که با هم سکس نداشته‌اند. \noun{بلوم} تلویحاً از پیچیدگی روابطشان از زمان بلوغِ \noun{میلی} آگاه است. همچنان که اپیزود به پایان خود نزدیک می‌شود، \noun{مالی} به صورت یک «گایا\footnote{گایا: در اسطوره‌های یونان، ایزدبانوی زمین است.}-ترا\footnote{ترا یا مادر ترا یا تلوس: در اساطیر رومی، ایزدبانویی بود که به زمین تجسم می‌بخشید. اسامی مادر ترا و مادر تلوس هر دو در لاتین به معنی «مادر زمین» هستند.}»، یا مادر زمین توصیف می‌شود در حالیکه \noun{بلوم} همزمان کودکی در رحم و ملوانی است که از سیر و سفرهایش برگشته و در حال استراحت است. یک نقطهٔ (.) تمثیلی، اپیزود را به پایان می‌برد و نشانگر استراحتگاه \noun{بلوم} است.

    \chapter[پنلوپه]{پنلوپه\protect\footnote{\lr{Penelope}-\rl{در اسطوره‌های یونان، همسر اولیس و مادر تلماخوس است. او در جوانی به دلیل زیبایی‌اش خواستگاران زیادی داشت. پدرش برای جلوگیری از دعوا و کشمکش مسابقه‌ای ترتیب داد تا برندهٔ آن را به دامادی انتخاب کند. در این مسابقه اولیس سرافراز بیرون آمد و با پنلوپه ازدواج کرد.}}}\label{ep:18}
    اولین جمله از جملات هشتگانهٔ طویل \noun{مالی} که مونولوگ درونی او را تشکیل می‌دهند با دلخوری و تعجب او از درخواست \noun{بلوم} برای آوردن صبحانه‌اش در رختخواب آغاز می‌شود. \noun{مالی} فهمیده که \noun{بلوم} امروز به ارگاسم رسیده و به هوسبازی‌های گذشتهٔ او با زنانِ دیگر فکر می‌کند. او به سکس بعدازظهرش با \noun{بویلانِ} خشن که آلت بزرگی هم داشت فکر می‌کند - تغییری فرح‌بخش بعد از روش‌های عشق‌بازی عجیب و غریب \noun{بلوم}. از طرفی دیگر \noun{مالی} معتقد است که \noun{بلوم} مردانه‌تر از \noun{بویلان} است و به یاد می‌آورد که وقتی نامزدبازی می‌کردند، \noun{بلوم} چقدر خوش‌تیپ بود. با فکر به ازدواج \noun{جوسی} و \noun{دنیس برین}، \noun{مالی} حس می‌کند که خودش و \noun{بلوم} احتمالاً خوشبخت هستند.

    در جملهٔ دومِ \noun{مالی}، او عشاق مختلفش را از نظر می‌گذراند: \noun{بویلان} که پاهای \noun{مالی} را دوست دارد؛ خوانندهٔ تِنور، \noun{بارتل دارسی}\LTRfootnote{\lr{Bartell D'Arcy}} که \noun{مالی} را در کلیسا بوسیده بود؛ ستوان \noun{گاردنر}\LTRfootnote{\lr{Gardner}} که در نبرد \noun{بوئر}\LTRfootnote{\lr{Boer}} از تب مرده بود. \noun{مالی} به تمایل جنسی \noun{بلوم} به لباس زیر فکر می‌کند. \noun{مالی} که تحریک شده، به ملاقات با \noun{بویلان} در دوشنبه و سفر آتی و تنهایشان به بلفاست فکر می‌کند. فکر \noun{مالی} موقتاً به دنیای آواز و کنسرت، خواننده‌های دختروار چندش‌آور دوبلینی و کمک \noun{بلوم} به کار حرفه‌ای خود پر می‌کشد. \noun{مالی} به یاد عصبانیت \noun{بویلان} نسبت به شرط بندی بازندهٔ \noun{لنه‌هان} در مسابقات \noun{گلدکاپ} می‌افتد. \noun{مالی} فکر می‌کند که \noun{لنه‌هان} آدم عجیب و غریبی است. با فکر به دیدارهای آینده با \noun{بویلان}، \noun{مالی} سعی دارد وزن کم کند و امیدوار است پول بیشتری داشته باشد تا شیک‌تر لباس بپوشد. \noun{بلوم} باید \noun{فریمن} را ترک کند و شغلی پرسود در دفتری برای خود بیابد. \noun{مالی} به خاطر می‌آورد که نزد آقای \noun{کاف}\LTRfootnote{\lr{Cuffe}} رفته بود تا راضی‌اش کند \noun{بلوم} را که اخراج شده بود بر سر کارش برگرداند - آقای \noun{کاف} به پستانهایش زل زده بود و مودبانه درخواست او را رد کرده بود.

    در جملهٔ سومِ \noun{مالی}، او به پستان‌های زیبای زنانه و آلت احمقانهٔ مردانه فکر می‌کند. او زمانی را به یاد می‌آورد که \noun{بلوم} به او پیشنهاد داده بود جلوی عکاسی برهنه شود تا از این راه پول در بیاورد. او تصاویر پورنوگرافیک را مانند عکس حوری‌ای می‌داند که \noun{بلوم} از آن برای توضیح مزخرفانهٔ تناسخ در صبح امروز استفاده کرده بود. با فکر دوباره به پستان‌ها، به خاطر می‌آورد که \noun{بلوم} یک بار پیشنهاد داده بود شیر اضافهٔ پستان او را در چای بریزند. قبل از اینکه افکارش دوباره به \noun{بویلان} و ارگاسم قدرتمندی که (\noun{مالی}) امروز بعدازظهر داشت برگردد، \noun{مالی} به این می‌اندیشد که همه تفکرات اعصاب خردکن \noun{بلوم} را در کتابی گردآوری کند.

    جملهٔ چهارم \noun{مالی} با صدای سوت قطاری آغاز می‌شود. فکر به موتور داغ قطار، افکار او را به کودکی‌هایش در \noun{جبل‌الطارق}، دوستی‌اش با \noun{هستر استنهوپ}\LTRfootnote{\lr{Hester Stanhope}} و «ووگر\LTRfootnote{\lr{Wogger}}» شوهر \noun{هستر} می‌برد و اینکه بعد از رفتن آنها، زندگی‌اش چقدر خسته کننده شده بود - او بعد از آن به خودش نامه می‌نوشت. \noun{مالی} به این فکر می‌کند که \noun{میلی} صبح امروز برای او فقط یک کارت ولی برای \noun{بلوم} یک نامهٔ کامل فرستاده بود. \noun{مالی} با خود می‌اندیشد که آیا \noun{بویلان} برایش نامهٔ عاشقانه‌ای خواهد فرستاد یا نه.

    جملهٔ پنجمِ \noun{مالی} با تجدید خاطره نسبت به اولین نامهٔ عاشقانه‌اش شروع می‌شود - از طرف ستوان \noun{مالوی}\LTRfootnote{\lr{Mulvey}}، که (\noun{مالی}) او را زیر دیوار \noun{موریش}\LTRfootnote{\lr{Moorish}} در \noun{جبل‌الطارق} بوسیده بود. او به این فکر می‌کند که ستوان الان در چه وضعی است. صدای سوت قطار دیگری می‌آید و \noun{مالی} را به یاد «آواز دلنشین قدیمی عشق» و اجرای قریب‌الوقوعش می‌اندازد. او باز نسبت به دختران ابله خواننده احساس تنفر می‌کند - \noun{مالی} خود را نسبت به آنان حرفه‌ای‌تر و باتجربه‌تر می‌داند. با درنظر گرفتن چهرهٔ سبزهٔ اسپانیایی‌اش که از مادرش به ارث برده است، \noun{مالی} معتقد است که اگر با \noun{بلوم} ازدواج نکرده بود الان ستارهٔ صحنه بود. \noun{مالی} به تختخواب می‌خزد تا آهسته بادی از خود رها کند که صدایش با سوت قطاری دیگر در هم می‌آمیزد.

    در جملهٔ ششم‌اش، ذهن \noun{مالی} از کودکی‌اش در \noun{جبل‌الطارق} به \noun{میلی} منحرف می‌شود. \noun{مالی} اکنون دوست ندارد شبها در خانه تنها بماند - این ایدهٔ \noun{بلوم} بود که \noun{میلی} را برای آموختن عکاسی به \noun{مالینگار} بفرستند چرا که او متوجه رابطهٔ قریب‌الوقوع \noun{مالی} و \noun{بویلان} شده بود. \noun{مالی} به رابطهٔ نزدیک ولی پیچیده‌اش با \noun{میلی} فکر می‌کند که مثل جوانی‌های \noun{مالی}، وحشی و زیبا شده است. \noun{مالی} با ناراحتی متوجه می‌شود که دارد پریود می‌شود و به دستشویی می‌رود. او متوجه می‌شود که \noun{بویلان} او را آبستن نکرده است. صحنه‌های بعدازظهر در ذهنش مرور می‌شوند.

    در جملهٔ هفتمش، \noun{مالی} به آرامی به تختخواب برمی‌گردد و دوباره به اسباب‌کشی‌های متعددی که به خاطر وضعیت مالیِ متزلزل \noun{بلوم} داشتند فکر می‌کند. \noun{مالی} نگران این است که مبادا \noun{بلوم} امروز برای زنی یا برای خانوادهٔ \noun{دیگنام} پول خرج کرده باشد. \noun{مالی} به مردهایی که در مراسم \noun{دیگنام} بودند فکر می‌کند - آنها خوبند، ولی \noun{مالی} از تحقیرهایی که نسبت به \noun{بلوم} روا می‌دارند متنفر است. \noun{مالی} به یاد استعداد آواز \noun{سایمون ددالوس‬} می‌افتد و به پسر \noun{سایمون}  فکر می‌کند. \noun{مالی} به خاطر می‌آورد که \noun{استیون} را وقتی بچه بود دیده و تصور می‌کند که احتمالاً گستاخ و مغرور نیست و فقط جوان و معصوم است. \noun{مالی} برنامه دارد که قبل از آمدن دوبارهٔ او کتاب بخواند و مطالعه کند تا او (\noun{استیون}) فکر نکند که \noun{مالی} احمق است.

    در جملهٔ هشتمش، \noun{مالی} به این فکر می‌کند که \noun{بلوم} هیچ وقت او را در آغوش نمی‌گیرد و به جای آن، به طرزی عجیب کفلش را می‌بوسد. \noun{مالی} به این فکر می‌کند که اگر دنیا توسط زن‌ها اداره می‌شد چقدر جای بهتری می‌بود. با در نظرگرفتن اهمیت مادران، او دوباره به \noun{استیون} که مادرش به تازگی درگذشته است و به مرگ \noun{رودی} فکر می‌کند، سپس از ترس افسرده شدن، این خط فکری را متوقف می‌سازد. \noun{مالی} به این فکر می‌کند که فردا صبح \noun{بلوم} را بیدار کند و به سردی، رابطه‌اش با \noun{بویلان} را به او بگوید تا متوجه تقصیرات خودش بشود. \noun{مالی} نقشه می‌کشد که فردا اگر \noun{استیون} آمد، گل بخرد. با فکر به گل‌ها و طبیعت، خلقت‌های خدا، او عاشقانه به روزی که او و \noun{بلوم} در \noun{هاوث} گذرانده بودند، خواستگاری‌ \noun{بلوم} از او و جواب مثبتِ سریع و مؤکدش فکر می‌کند.
\end{document}