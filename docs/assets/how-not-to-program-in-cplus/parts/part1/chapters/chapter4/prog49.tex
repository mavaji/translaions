\section[به سادگیِ گسستن یک ارتباط]{به سادگیِ گسستن یک ارتباط \protect\LTRfootnote{\lr{It's As Easy As Falling Off  a Link}} (راهنمایی \ref{hint:186}، جواب \ref{answer:77})}
\paragraph{}\label{prog:49}
چرا برنامهٔ زیر بعضی مواقع ... ؟

\begin{LTR}
    %@formatter:off
    \begin{lstlisting}[style=C++Style]
         #include <iostream>
         #include <string>
         /************************************************
         * linked_list -- Class to handle a linked list *
         * containing a list of strings. *
         * *
         * Member functions: *
         * add -- Add an item to the list *
         * is_in -- Check to see if a string is *
         * in the list. *
         ************************************************/
         class linked_list {
         	private:
         		/*
         		* Node in the list
         		*/
         		struct node {
         			// String in this node
         			std::string data;

         			// Pointer to next node
         			struct node *next;
         		};
         		//First item in the list
         		struct node *first;
         	public:
         		// Constructor
         		linked_list(void): first(NULL) {};
         		// Destructor
         		~linked_list();
         	private:
         		// No copy constructor
         		linked_list(const linked_list &);

         		// No assignment operator
         		linked_list& operator = (const linked_list &);
         	public:
         		// Add an item to the list
         		void add(
         			// Item to add
         			const std::string &what
         		) {
         			// Create a node to add
         			struct node *new_ptr = new node;

         			// Add the node
         			new_ptr->next = first;
         			new_ptr->data = what;
         			first = new_ptr;
         		}
         		bool is_in(const std::string &what);
         };
         /************************************************
         * is_in -- see if a string is in a *
         * linked list. *
         * *
         * Returns true if string's on the list, *
         * otherwise false. *
         ************************************************/
         bool linked_list::is_in(
         	// String to check for
         	const std::string &what
         ) {
         	/* current structure we are looking at */
         	struct node *current_ptr;

         	current_ptr = first;

         	while (current_ptr != NULL) {
         		if (current_ptr->data == what)
         		return (true);

         		current_ptr = current_ptr->next;
         	}
         	return (false);
         }

         /************************************************
         * linked_list::~linked_list -- Delete the *
         * data in the linked list. *
         ************************************************/
         linked_list::~linked_list(void) {
         	while (first != NULL) {
         		delete first;
         		first = first->next;
         	}
         }

         int main() {
         	linked_list list; // A list to play with

         	list.add("Sam");
         	list.add("Joe");
         	list.add("Mac");

         	if (list.is_in("Harry"))
         		std::cout << "Harry is on the list\n";
         	else
         		std::cout << "Could not find Harry\n";
         	return (0);
         }
    \end{lstlisting}
    %@formatter:on
\end{LTR}

\begin{tcolorbox}
    یک بانوی نظافتچی، جای خراشی در کف اتاق کامپیوتر یافت و تصمیم گرفت که آن را پاک کند. اول واکس را امتحان کرد، سپس پاک‌کنندهٔ آمونیاک و نهایتاً به عنوان راهِ آخر از سیم ظرفشویی استفاده نمود. ترکیب این مواد، نه تنها برای جای خراش بلکه برای کامپیوترها هم کشنده بود.

    روز بعد، وقتی اعضای اتاق، سر کار آمدند، تمام دستگاه‌هایشان از کار افتاده بود. با باز کردنِ دستگاه‌ها، متوجه شدند که اتصال‌کوتاه‌های بسیار زیادی در تمام مَدارات رخ داده است.

    چه اتفاقی افتاده بود؟ بانوی نظافتچی، ابتدا لایه‌ای از واکس را به کف اتاق کشیده بود. آمونیاک، واکس را تبخیر کرده بود، که توسط هواکش‌های کامپیوترها به درون مکیده شده بود. لذا تمام مدارات توسط یک لایه از واکسِ چسبناک پوشیده شده بود. تا اینجا زیاد بد نبود ولی بعد نوبت سیم ظرفشویی بود. بُراده‌های سیم ظرفشویی به درون دستگاه‌ها کشیده شده بود و به واکس موجود روی مدارات چسبیده بود و موجب اتصال کوتاه شده بود.
\end{tcolorbox}