\section[دوباره سلام]{دوباره سلام \protect\LTRfootnote{\lr{Hello Again}} (راهنمایی \ref{hint:214}، جواب \ref{answer:50})}
\paragraph{}\label{prog:97}
برنامهٔ زیر چه چیزی را چاپ می‌کند؟

\begin{LTR}
    %@formatter:off
        \begin{lstlisting}[style=C++Style]
             /************************************************
             * Normally I would put in a comment explaining *
             * what this program is nominally used for. *
             * But in this case I can figure out no *
             * practical use for this program. *
             ************************************************/
             #include <stdio.h>
             #include <unistd.h>
             #include <stdlib.h>

             int main()
             {
             	printf("Hello ");
             	fork();
             	printf("\n");
             	exit(0);
             }
        \end{lstlisting}
    %@formatter:on
\end{LTR}

\begin{tcolorbox}
    شکسپیر، سؤالی قدیمی پرسیده است: \lr{"To be or not to be?"} علوم کامپیوتر جواب آن را به ما داده است:
    \LTR
    \lr{\texttt{0x2B | \textasciitilde0x2B = 0xFF}}
    \RTL
    وقتی این طنز را به افراد غیرفنی می‌گویم با تعجب به من می‌نگرند. افراد فنی، دقیقه‌ای فکر می‌کنند و سپس می‌گویند: «حق با توست». فقط یک نفر از صد نفر واقعاً می‌خندد.
\end{tcolorbox}