\section[سرعت می‌کُشد]{سرعت می‌کُشد \protect\LTRfootnote{\lr{Speed Kills}} (راهنمایی \ref{hint:305}، جواب \ref{answer:56})}
\paragraph{}\label{prog:94}
فراخوانی توابع \lr{\texttt{new}} و \lr{\texttt{delete}} هزینه‌بر هستند. اگر بخواهید برنامهٔ خود را تسریع ببخشید و بدانید که چه کار دارید می‌کنید می‌توانید آنها را بازنویسی\LTRfootnote{\lr{Override}} کنید و \lr{\texttt{new}} و \lr{\texttt{delete}} مخصوصِ کلاسِ خود را ایجاد نمایید. این کاری است که برنامه‌نویس انجام داده است. الگوریتم تخصیص، بسیار ساده است، ولی به طریقی حافظه خراب می‌شود. چرا؟

\begin{LTR}
    %@formatter:off
        \begin{lstlisting}[style=C++Style]
             /***********************************************
             * bit_test -- Test out our new high speed *
             * bit_array. *
             ***********************************************/
             #include <iostream>
             #include <memory.h>

             // The size of a fast bit_array.
             // (Private to fast bit array)
             const int BIT_ARRAY_MAX = 64; // Size in bits

             // Number of bits in a byte
             const int BITS_PER_BYTE = 8;
             /************************************************
             * fast_bit_array -- A bit array using fast *
             * allocate technology. *
             * *
             * Member functions: *
             * get -- Get an element from the *
             * array. *
             * set -- Set the value of an element *
             * in the array. *
             * *
             * new -- used to quickly allocate a bit *
             * array. *
             * delete -- used to quickly deallocate *
             * a bit array. *
             ************************************************/
             class fast_bit_array
             {
             	protected:
             		// Array data
             		unsigned char
             		data[BIT_ARRAY_MAX/BITS_PER_BYTE];

             	public:
             		fast_bit_array(void)
             		{
             			memset(data, '\0', sizeof(data));
             		}
             		// Destructor defaults
             	private:
             		// No copy constructor
             		fast_bit_array(const fast_bit_array &);

             		// No assignment operator
             		fast_bit_array & operator =
             			(const fast_bit_array &);
             	public:
             		// Set the value on an item
             		void set(
             			// Index into the array
             			const unsigned int index,

             			// Value to put in the array
             			const unsigned int value
             		)
             		{
             			// Index into the bit in the byte
             			unsigned int bit_index = index % 8;

             			// Byte in the array to use
             			unsigned int byte_index = index / 8;

             			if (value)
             			{
             				data[byte_index] |=
             					(1 << bit_index);
             			}
             			else
             			{
             				data[byte_index] &=
             					~(1 << bit_index);
             			}
             		}
             		// Return the value of an element
             		int get(unsigned int index)
             		{
             			// Index into the bit in the byte
             			unsigned int bit_index = index % 8;
             			// Byte in the array to use
             			unsigned int byte_index = index / 8;

             			return (
             				(data[byte_index] &
             				(1 << bit_index)) != o);
             		}
             		// Allocate a new fast_bit_array
             		void *operator new(const size_t);

             		// Delete a fast bit array.
             		void operator delete(void *ptr);
             };

             /************************************************
             * The following routines handle the local *
             * new/delete for the fast_bit_array. *
             ************************************************/
             // Max number of fast_bit_arrays we can use at once
             const int N_FAST_BIT_ARRAYS = 30;

             // If true, the bit array slot is allocated
             // false indicates a free slot
             static bool
             bit_array_used[N_FAST_BIT_ARRAYS] = {false};

             // Space for our fast bit arrays.
             static char
             bit_array_mem[N_FAST_BIT_ARRAYS]
             [sizeof(fast_bit_array)];

             // Handle new for "fast_bit_array".
             // (This is much quicker than the
             // system version of new)
             /************************************************
             * fast_bit_array -- new *
             * *
             * This is a high speed allocation routine for *
             * the fast_bit_array class. The method used *
             * for this is simple, but we know that only *
             * a few bit_arrays will be allocated. *
             * *
             * Returns a pointer to the new memory. *
             ************************************************/
             void *fast_bit_array::operator new(const size_t)
             {
             	int i; // Index into the bit array slots

             	// Look for a free slot
             	for (i = 0; i < N_FAST_BIT_ARRAYS; ++i)
             	{
             		if (!bit_array_used[i])
             		{
             			// Free slot found, allocate the space
             			bit_array_used[i] = true;
             			return(bit_array_mem[i]);
             		}
             	}
             	std::cout << "Error: Out of local memory\n";
             	exit (8);
             }

             /************************************************
             * fast_bit_array -- delete *
             * *
             * Quickly free the space used by a *
             * fast bit array. *
             ************************************************/
             void fast_bit_array::operator delete(
             	void *ptr // Pointer to the space to return
             )
             {
             	int i; // Slot index

             	for (i = 0; i < N_FAST_BIT_ARRAYS; ++i)
             	{
             		// Is this the right slot
             		if (ptr == bit_array_mem[i])
             		{
             			// Right slot, free it
             			bit_array_used[i] = false;
             			return;
             		}
             	}
             	std::cout <<
             		"Error: Freed memory we didn't have\n";
             	exit (8);
             }


             /************************************************
             * safe_bit_array -- A safer bit array. *
             * *
             * Like bit array, but with error checking. *
             ************************************************/
             class safe_bit_array : public fast_bit_array
             {
             	public:
             		// Sequence number generator
             		static int bit_array_counter;

             		// Our bit array number
             		int sequence;

             		safe_bit_array(void)
             		{
             			sequence = bit_array_counter;
             			++bit_array_counter;
             		};
             		// Destructor defaults
             	private:
             		// No copy constructor
             		safe_bit_array(const safe_bit_array &);

             		// No assignment operator
             		safe_bit_array & operator = (
             				const safe_bit_array &);
             	public:
             		// Set the value on an item
             		void set(
             			// Where to put the item
             			const unsigned int index,
             			// Item to put
             			const unsigned int value
             		)
             		{
             			if (index >= (BIT_ARRAY_MAX-1))
             			{
             				std::cout <<
             					"Bit array set error "
             					"for bit array #"
             					<< sequence << "\n";
             				exit (8);
             			}
             			fast_bit_array::set(index, value);
             		}
             		// Return the value of an element
             		int get(unsigned int index)
             		{
             			if (index >= (BIT_ARRAY_MAX-1))
             			{
             				std::cout <<
             					"Bit array get error "
             					"for bit array #"
             					<< sequence << "\n";
             				exit (8);
             			}
             			return (fast_bit_array::get(index));
             		}
             };

             // Sequence information
             int safe_bit_array::bit_array_counter = 0;

             int main()
             {
             	// Create a nice new safe bit array
             	safe_bit_array *a_bit_array =
             	new safe_bit_array;

             	a_bit_array->set(5, 1);
             	// Return the bit_array to the system
             	delete a_bit_array;
             	return (0);
             }
        \end{lstlisting}
    %@formatter:on
\end{LTR}

\begin{tcolorbox}
    یک تکنولوژیِ سطح‌بالا از جادو قابل تمییز نیست.
    \LTR
    \rl{آرتور س. کلارک}
\end{tcolorbox}